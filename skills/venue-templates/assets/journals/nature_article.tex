% Nature Journal Article Template
% For submission to Nature family journals
% Last updated: 2024

\documentclass[12pt]{article}

% Packages
\usepackage[margin=2.5cm]{geometry}
\usepackage{times}
\usepackage{graphicx}
\usepackage{amsmath}
\usepackage{amssymb}
\usepackage{hyperref}
\usepackage{lineno}  % Line numbers for review
\usepackage[super]{natbib}  % Superscript citations

% Line numbering (required for submission)
\linenumbers

% Title and Authors
\title{Insert Your Title Here: Concise and Descriptive}

\author{
First Author\textsuperscript{1}, Second Author\textsuperscript{1,2}, Third Author\textsuperscript{2,*}
}

\date{}

\begin{document}

\maketitle

% Affiliations
\noindent
\textsuperscript{1}Department Name, Institution Name, City, State/Province, Postal Code, Country \\
\textsuperscript{2}Second Department/Institution \\
\textsuperscript{*}Correspondence: [email protected]

% Abstract
\begin{abstract}
\noindent
Write a concise abstract of 150-200 words summarizing the main findings, significance, and conclusions of your work. The abstract should be self-contained and understandable without reading the full paper. Focus on what you did, what you found, and why it matters. Avoid jargon and abbreviations where possible.
\end{abstract}

% Main Text
\section*{Introduction}
% 2-3 paragraphs setting the context
Provide background on the research area, establish the importance of the problem, and identify the knowledge gap your work addresses. Nature papers should emphasize broad significance beyond a narrow specialty.

State your main research question or objective clearly.

Briefly preview your approach and key findings.

\section*{Results}
% Primary results section
% Organize by finding, not by experiment
% Reference figures/tables as you describe results

\subsection*{First major finding}
Describe your first key result. Reference Figure~\ref{fig:example} to support your findings.

\begin{figure}[ht]
\centering
% Include your figure here
% \includegraphics[width=0.7\textwidth]{figure1.pdf}
\caption{{\bf Figure title in bold.} Detailed figure caption explaining what is shown, experimental conditions, sample sizes (n), statistical tests, and significance levels. Panels should be labeled (a), (b), etc. if multiple panels are present.}
\label{fig:example}
\end{figure}

\subsection*{Second major finding}
Describe your second key result objectively, without interpretation.

\subsection*{Third major finding}
Describe additional results as needed.

\section*{Discussion}
% Interpret results, compare to literature, acknowledge limitations

\subsection*{Main findings and interpretation}
Summarize your key findings and explain their significance. How do they advance our understanding?

\subsection*{Comparison to previous work}
Compare and contrast your results with existing literature\cite{example2023}.

\subsection*{Implications}
Discuss the broader implications of your work for the field and beyond.

\subsection*{Limitations and future directions}
Honestly acknowledge limitations and suggest future research directions.

\section*{Conclusions}
Provide a concise conclusion summarizing the main take-home messages of your work.

\section*{Methods}
% Detailed methods allowing reproducibility
% Can be placed after main text in Nature

\subsection*{Experimental design}
Describe overall experimental design, including controls.

\subsection*{Sample preparation}
Detail procedures for sample collection, preparation, and handling.

\subsection*{Data collection}
Describe instrumentation, measurement procedures, and data collection protocols.

\subsection*{Data analysis}
Explain analytical methods, statistical tests, and software used. State sample sizes, replication, and significance thresholds.

\subsection*{Ethical approval}
Include relevant ethical approval statements (human subjects, animal use, biosafety).

\section*{Data availability}
State where data supporting the findings can be accessed (repository, supplementary files, available on request).

\section*{Code availability}
If applicable, provide information on code availability (GitHub, Zenodo, etc.).

\section*{Acknowledgements}
Acknowledge funding sources, technical assistance, and other contributions. List grant numbers.

\section*{Author contributions}
Describe contributions of each author using CRediT taxonomy or similar (conceptualization, methodology, investigation, writing, etc.).

\section*{Competing interests}
Declare any financial or non-financial competing interests. If none, state "The authors declare no competing interests."

% References
\bibliographystyle{naturemag}  % Nature bibliography style
\bibliography{references}  % Your .bib file

% Alternatively, manually format references:
\begin{thebibliography}{99}

\bibitem{example2023}
Smith, J. D., Jones, M. L. \& Williams, K. R. Groundbreaking discovery in the field. \textit{Nature} \textbf{600}, 123--130 (2023).

\bibitem{author2022}
Author, A. A. \& Coauthor, B. B. Another important paper. \textit{Nat. Methods} \textbf{19}, 456--

460 (2022).

% Add more references as needed

\end{thebibliography}

% Figure Legends (if not included with figures)
\section*{Figure Legends}

\textbf{Figure 1 | Figure title.} Comprehensive figure legend describing all panels, experimental conditions, sample sizes, and statistical analyses.

\textbf{Figure 2 | Second figure title.} Another detailed legend.

% Extended Data Figures (optional - supplementary figures)
\section*{Extended Data}

\textbf{Extended Data Figure 1 | Supplementary data title.} Description of supplementary figure supporting main findings.

\end{document}

% Notes for Authors:
% 1. Nature articles are typically ~3,000 words excluding Methods, References, Figure Legends
% 2. Use superscript numbered citations (1, 2, 3)
% 3. Figures should be high resolution (300+ dpi for photos, 1000 dpi for line art)
% 4. Submit figures as separate files (TIFF, EPS, or PDF)
% 5. Double-space the manuscript for review
% 6. Include line numbers using \linenumbers
% 7. Follow Nature's specific author guidelines for your target journal
% 8. Methods section can be quite detailed and placed after main text
% 9. Check word limits and specific requirements for your Nature family journal

