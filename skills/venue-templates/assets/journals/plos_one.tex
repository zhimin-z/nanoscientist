% PLOS ONE Article Template
% For submission to PLOS ONE and other PLOS journals
% Last updated: 2024

\documentclass[10pt,letterpaper]{article}

% Packages
\usepackage[top=0.85in,left=2.75in,footskip=0.75in]{geometry}
\usepackage{amsmath,amssymb}
\usepackage{changepage}
\usepackage[utf8]{inputenc}
\usepackage{textcomp,marvosym}
\usepackage{cite}
\usepackage{nameref,hyperref}
\usepackage[right]{lineno}
\usepackage{microtype}
\usepackage{graphicx}
\usepackage[table]{xcolor}
\usepackage{array}
\usepackage{authblk}

% Line numbering
\linenumbers

% Set up authblk for PLOS format
\renewcommand\Authfont{\fontsize{12}{14}\selectfont}
\renewcommand\Affilfont{\fontsize{9}{11}\selectfont}

% Title
\title{Your Article Title Here: Concise and Descriptive}

% Authors and Affiliations
\author[1]{First Author}
\author[1,2]{Second Author}
\author[2,$\dagger$]{Third Author}

\affil[1]{Department of Biology, University Name, City, State, Country}
\affil[2]{Institute of Research, Institution Name, City, Country}

% Corresponding author
\affil[$\dagger$]{Corresponding author. E-mail: [email protected]}

\date{}

\begin{document}

\maketitle

% Abstract
\begin{abstract}
\noindent
Write a structured or unstructured abstract of 250-300 words. The abstract should be accessible to a broad readership and should clearly state: (1) background/rationale, (2) objectives, (3) methods, (4) principal findings with key data, and (5) conclusions and significance. Do not include citations in the abstract.
\end{abstract}

% Introduction
\section*{Introduction}

Provide background and context for your study. The introduction should:
\begin{itemize}
    \item Present the rationale for your study
    \item Clearly state what is currently known about the topic
    \item Identify the knowledge gap your study addresses
    \item State your research objectives or hypotheses
    \item Explain the significance of the research
\end{itemize}

Review relevant literature \cite{smith2023,jones2022}, setting your work in context.

State your main research question or objective at the end of the introduction.

% Materials and Methods
\section*{Materials and Methods}

Provide sufficient detail to allow reproduction of your work.

\subsection*{Study Design}
Describe the overall study design (e.g., prospective cohort, randomized controlled trial, observational study, etc.).

\subsection*{Participants/Samples}
Describe your study population, sample collection, or experimental subjects:
\begin{itemize}
    \item Sample size and how it was determined (power analysis)
    \item Inclusion and exclusion criteria
    \item Demographic information
    \item For animal studies: species, strain, age, sex, housing conditions
\end{itemize}

\subsection*{Procedures}
Detail all experimental procedures, measurements, and interventions. Include:
\begin{itemize}
    \item Equipment and reagents (with manufacturer, catalog numbers)
    \item Protocols and procedures (step-by-step if novel)
    \item Controls used
    \item Blinding and randomization (if applicable)
\end{itemize}

\subsection*{Data Collection}
Describe how data were collected, including instruments, assays, and measurements.

\subsection*{Statistical Analysis}
Clearly describe statistical methods used:
\begin{itemize}
    \item Software and version (e.g., R 4.3.0, Python 3.9 with scipy 1.9.0)
    \item Statistical tests performed (e.g., t-tests, ANOVA, regression)
    \item Significance level ($\alpha$, typically 0.05)
    \item Corrections for multiple testing
    \item Sample size justification
\end{itemize}

\subsection*{Ethical Approval}
Include relevant ethical approval statements:
\begin{itemize}
    \item Human subjects: IRB approval, protocol number, consent procedures
    \item Animal research: IACUC approval, protocol number, welfare considerations
    \item Field studies: Permits and permissions
\end{itemize}

Example: "This study was approved by the Institutional Review Board of University Name (Protocol \#12345). All participants provided written informed consent."

% Results
\section*{Results}

Present your findings in a logical sequence. Refer to figures and tables as you describe results. Do not interpret results in this section (save for Discussion).

\subsection*{First Major Finding}
Describe your first key result. Statistical results should include effect sizes and confidence intervals in addition to p-values.

As shown in Figure~\ref{fig:results1}, we observed a significant increase in [outcome variable] (mean $\pm$ SD: 45.2 $\pm$ 8.3 vs. 32.1 $\pm$ 6.9; t = 7.42, df = 48, p < 0.001).

\begin{figure}[!ht]
\centering
% \includegraphics[width=0.75\textwidth]{figure1.png}
\caption{{\bf Figure 1. Title of first figure.}
Detailed figure legend describing what is shown. Include: (A) Description of panel A. (B) Description of panel B. Sample sizes (n), error bars represent [SD, SEM, 95\% CI], and statistical significance indicated by asterisks (* p < 0.05, ** p < 0.01, *** p < 0.001). Statistical test used should be stated.}
\label{fig:results1}
\end{figure}

\subsection*{Second Major Finding}
Describe your second key result, referencing Table~\ref{tab:results1}.

\begin{table}[!ht]
\centering
\caption{{\bf Table 1. Title of table.}}
\label{tab:results1}
\begin{tabular}{lccc}
\hline
\textbf{Condition} & \textbf{Measurement 1} & \textbf{Measurement 2} & \textbf{p-value} \\
\hline
Control & 25.3 $\pm$ 3.1 & 48.2 $\pm$ 5.4 & -- \\
Treatment A & 32.7 $\pm$ 2.8 & 55.1 $\pm$ 4.9 & 0.003 \\
Treatment B & 41.2 $\pm$ 3.5 & 62.8 $\pm$ 6.2 & < 0.001 \\
\hline
\end{tabular}
\begin{flushleft}
Values shown as mean $\pm$ standard deviation (n = 20 per group). P-values from one-way ANOVA with Tukey's post-hoc test comparing to control.
\end{flushleft}
\end{table}

\subsection*{Additional Results}
Present additional findings as needed.

% Discussion
\section*{Discussion}

Interpret your results and place them in the context of existing literature.

\subsection*{Principal Findings}
Summarize your main findings concisely.

\subsection*{Interpretation}
Interpret your findings and explain their significance. How do they advance understanding of the topic? Compare and contrast with previous studies \cite{brown2021,williams2020}.

\subsection*{Strengths and Limitations}
Discuss both strengths and limitations of your study honestly:

\textbf{Strengths:}
\begin{itemize}
    \item Large sample size with adequate statistical power
    \item Rigorous methodology with appropriate controls
    \item Novel approach or finding
\end{itemize}

\textbf{Limitations:}
\begin{itemize}
    \item Cross-sectional design limits causal inference
    \item Generalizability may be limited to [specific population]
    \item Potential confounding variables not measured
\end{itemize}

\subsection*{Implications}
Discuss the practical or theoretical implications of your findings.

\subsection*{Future Directions}
Suggest directions for future research.

% Conclusions
\section*{Conclusions}

Provide a concise conclusion summarizing the main findings and their significance. Avoid repeating the abstract.

% Acknowledgments
\section*{Acknowledgments}

Acknowledge individuals who contributed but do not meet authorship criteria, technical assistance, and writing assistance. Example: "We thank Dr. Jane Doe for technical assistance with microscopy and Dr. John Smith for helpful discussions."

% References
\section*{References}

% Using BibTeX
\bibliographystyle{plos2015}
\bibliography{references}

% Or manually formatted (Vancouver style, numbered):
\begin{thebibliography}{99}

\bibitem{smith2023}
Smith JD, Johnson ML, Williams KR. Title of article. Journal Abbrev. 2023;45(3):301-318. doi:10.1371/journal.pone.1234567.

\bibitem{jones2022}
Jones AB, Brown CD. Another article title. PLoS ONE. 2022;17(8):e0234567. doi:10.1371/journal.pone.0234567.

\bibitem{brown2021}
Brown EF, Davis GH, Wilson IJ, Taylor JK. Comprehensive study title. Nat Commun. 2021;12:1234. doi:10.1038/s41467-021-12345-6.

\bibitem{williams2020}
Williams LM, Anderson NO. Previous work on topic. Science. 2020;368(6489):456-460. doi:10.1126/science.abc1234.

\end{thebibliography}

% Supporting Information
\section*{Supporting Information}

List all supporting information files (captions provided separately during submission):

\paragraph{S1 Fig.}
\textbf{Title of supplementary figure 1.} Brief description.

\paragraph{S2 Fig.}
\textbf{Title of supplementary figure 2.} Brief description.

\paragraph{S1 Table.}
\textbf{Title of supplementary table 1.} Brief description.

\paragraph{S1 Dataset.}
\textbf{Raw data.} Complete dataset used in analysis (CSV format).

\paragraph{S1 File.}
\textbf{Supplementary methods.} Additional methodological details.

% Author Contributions (CRediT taxonomy recommended)
\section*{Author Contributions}

Use CRediT (Contributor Roles Taxonomy):
\begin{itemize}
    \item \textbf{Conceptualization:} FA, SA
    \item \textbf{Data curation:} FA
    \item \textbf{Formal analysis:} FA, SA
    \item \textbf{Funding acquisition:} TA
    \item \textbf{Investigation:} FA, SA
    \item \textbf{Methodology:} FA, SA, TA
    \item \textbf{Project administration:} TA
    \item \textbf{Resources:} TA
    \item \textbf{Software:} FA
    \item \textbf{Supervision:} TA
    \item \textbf{Validation:} FA, SA
    \item \textbf{Visualization:} FA
    \item \textbf{Writing – original draft:} FA
    \item \textbf{Writing – review \& editing:} FA, SA, TA
\end{itemize}

(FA = First Author, SA = Second Author, TA = Third Author)

% Data Availability Statement (REQUIRED)
\section*{Data Availability}

Choose one of the following:

\textbf{Option 1 (Public repository):} 
All data are available in the [repository name] repository at [URL/DOI].

\textbf{Option 2 (Supporting Information):}
All relevant data are within the paper and its Supporting Information files.

\textbf{Option 3 (Available on request):}
Data cannot be shared publicly because of [reason]. Data are available from the [institution/contact] (contact via [email]) for researchers who meet the criteria for access to confidential data.

\textbf{Option 4 (Third-party):}
Data are available from [third party] (contact: [details]) for researchers who meet criteria for access.

% Funding Statement (REQUIRED)
\section*{Funding}

State all funding sources including grant numbers. If no funding, state "The authors received no specific funding for this work."

Example: "This work was supported by the National Science Foundation (NSF) [grant number 123456 to TA] and the National Institutes of Health (NIH) [grant number R01-234567 to TA]. The funders had no role in study design, data collection and analysis, decision to publish, or preparation of the manuscript."

% Competing Interests (REQUIRED)
\section*{Competing Interests}

Declare any financial or non-financial competing interests. If none, state: "The authors have declared that no competing interests exist."

If competing interests exist, declare them explicitly: "Author TA is a consultant for Company X. This does not alter our adherence to PLOS ONE policies on sharing data and materials."

\end{document}

% Notes for Authors:
% 1. PLOS ONE has no length limit - be concise but thorough
% 2. Use Vancouver style for citations [1], [2], [3]
% 3. Figures: TIFF or EPS format, 300-600 dpi
% 4. All data must be made available (data availability statement required)
% 5. Include line numbers for review
% 6. PLOS ONE focuses on scientific rigor, not novelty or impact
% 7. Reporting guidelines encouraged (CONSORT, STROBE, PRISMA, etc.)
% 8. Ethical approval required for human/animal studies
% 9. All authors must agree to submission
% 10. Submit via PLOS online submission system

