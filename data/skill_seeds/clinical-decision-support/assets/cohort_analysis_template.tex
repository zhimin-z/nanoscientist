\documentclass[10pt,letterpaper]{article}

% Packages
\usepackage[margin=0.5in]{geometry}
\usepackage[utf8]{inputenc}
\usepackage[T1]{fontenc}
\usepackage{helvet}
\renewcommand{\familydefault}{\sfdefault}
\usepackage{xcolor}
\usepackage{tcolorbox}
\usepackage{array}
\usepackage{tabularx}
\usepackage{booktabs}
\usepackage{enumitem}
\usepackage{titlesec}
\usepackage{fancyhdr}
\usepackage{multicol}
\usepackage{graphicx}
\usepackage{float}

% Color definitions
\definecolor{headerblue}{RGB}{0,102,204}
\definecolor{highlightgreen}{RGB}{0,153,76}
\definecolor{warningred}{RGB}{204,0,0}
\definecolor{highlightgray}{RGB}{240,240,240}
\definecolor{biomarkerblue}{RGB}{51,102,204}

% Section formatting - compact
\titleformat{\section}{\normalfont\fontsize{11}{12}\bfseries\color{headerblue}}{\thesection}{0.5em}{}
\titlespacing*{\section}{0pt}{4pt}{2pt}

\titleformat{\subsection}{\normalfont\fontsize{10}{11}\bfseries}{\thesubsection}{0.5em}{}
\titlespacing*{\subsection}{0pt}{3pt}{1pt}

% List formatting - ultra compact
\setlist[itemize]{leftmargin=*,itemsep=0pt,parsep=0pt,topsep=1pt}
\setlist[enumerate]{leftmargin=*,itemsep=0pt,parsep=0pt,topsep=1pt}

% Remove paragraph indentation
\setlength{\parindent}{0pt}
\setlength{\parskip}{2pt}

% Header/footer
\pagestyle{fancy}
\fancyhf{}
\fancyhead[L]{\footnotesize \textbf{Clinical Decision Support: [COHORT NAME]}}
\fancyhead[R]{\footnotesize Page \thepage}
\renewcommand{\headrulewidth}{0.5pt}
\fancyfoot[C]{\footnotesize Confidential Medical Document - For Professional Use Only}

\begin{document}

% Title block - compact
\begin{center}
{\fontsize{14}{16}\selectfont\bfseries\color{headerblue} PATIENT COHORT ANALYSIS REPORT}\\[2pt]
{\fontsize{12}{14}\selectfont\bfseries [Cohort Description - e.g., NSCLC Patients Stratified by PD-L1 Expression]}\\[2pt]
{\fontsize{10}{12}\selectfont [Institution/Study Name]}\\[1pt]
{\fontsize{9}{11}\selectfont Report Date: [Date]}
\end{center}

\vspace{4pt}

% Executive Summary Box
\begin{tcolorbox}[colback=highlightgray,colframe=headerblue,title=\textbf{Executive Summary},fonttitle=\bfseries\small,coltitle=black]
{\small
\textbf{Cohort}: [n=XX] patients with [disease] stratified by [biomarker/characteristic]

\textbf{Key Findings}:
\begin{itemize}
\item [Primary finding - e.g., Biomarker+ patients had significantly longer PFS]
\item [Secondary finding - e.g., ORR 45\% vs 30\%, p=0.023]
\item [Safety finding - e.g., Similar toxicity profiles between groups]
\end{itemize}

\textbf{Clinical Implications}: [Treatment recommendations based on findings]
}
\end{tcolorbox}

\vspace{2pt}

\section{Cohort Characteristics}

\subsection{Patient Demographics}

[Narrative description of cohort composition, inclusion/exclusion criteria, time period]

\begin{table}[H]
\centering
\small
\begin{tabular}{lccc}
\toprule
\textbf{Characteristic} & \textbf{Group A (n=XX)} & \textbf{Group B (n=XX)} & \textbf{p-value} \\
\midrule
Age, years (median [IQR]) & XX [XX-XX] & XX [XX-XX] & X.XX \\
Sex, n (\%) & & & \\
\quad Male & XX (XX\%) & XX (XX\%) & X.XX \\
\quad Female & XX (XX\%) & XX (XX\%) & \\
ECOG PS, n (\%) & & & \\
\quad 0-1 & XX (XX\%) & XX (XX\%) & X.XX \\
\quad 2 & XX (XX\%) & XX (XX\%) & \\
Disease Stage, n (\%) & & & \\
\quad III & XX (XX\%) & XX (XX\%) & X.XX \\
\quad IV & XX (XX\%) & XX (XX\%) & \\
Prior Lines of Therapy & & & \\
\quad 0 (treatment-naïve) & XX (XX\%) & XX (XX\%) & X.XX \\
\quad 1-2 & XX (XX\%) & XX (XX\%) & \\
\quad $\geq$3 & XX (XX\%) & XX (XX\%) & \\
\bottomrule
\end{tabular}
\caption{Baseline patient demographics and clinical characteristics}
\end{table}

\subsection{Biomarker Profile}

\begin{tcolorbox}[colback=biomarkerblue!10,colframe=biomarkerblue,title=\textbf{Biomarker Stratification},fonttitle=\bfseries\small]
{\small
\textbf{Classification Method}: [e.g., IHC for PD-L1 expression, NGS for mutations, gene expression clustering]

\textbf{Group Definitions}:
\begin{itemize}
\item \textbf{Group A (Biomarker+)}: [n=XX] - [Definition, e.g., PD-L1 TPS $\geq$50\%, or Mesenchymal-Immune-Active subtype]
\item \textbf{Group B (Biomarker-)}: [n=XX] - [Definition, e.g., PD-L1 TPS <50\%]
\end{itemize}

\textbf{Molecular Features of Group A}:
\begin{itemize}
\item [Feature 1]: XX\% (n=XX) - [Clinical significance]
\item [Feature 2]: XX\% (n=XX) - [Clinical significance]
\item [Feature 3]: Elevated/decreased [marker] (median [value])
\end{itemize}
}
\end{tcolorbox}

\section{Treatment Exposures}

\begin{table}[H]
\centering
\small
\begin{tabular}{lcc}
\toprule
\textbf{Treatment Received} & \textbf{Group A, n (\%)} & \textbf{Group B, n (\%)} \\
\midrule
[Treatment regimen 1] & XX (XX\%) & XX (XX\%) \\
[Treatment regimen 2] & XX (XX\%) & XX (XX\%) \\
[Treatment regimen 3] & XX (XX\%) & XX (XX\%) \\
Median cycles received (range) & X (X-X) & X (X-X) \\
\bottomrule
\end{tabular}
\caption{Treatment exposures by biomarker group}
\end{table}

\section{Treatment Outcomes}

\subsection{Response Rates}

\begin{table}[H]
\centering
\small
\begin{tabular}{lccc}
\toprule
\textbf{Response Category} & \textbf{Group A (n=XX)} & \textbf{Group B (n=XX)} & \textbf{p-value} \\
\midrule
Objective Response Rate (ORR) & XX\% [95\% CI] & XX\% [95\% CI] & X.XXX \\
\quad Complete Response (CR) & XX (XX\%) & XX (XX\%) & \\
\quad Partial Response (PR) & XX (XX\%) & XX (XX\%) & \\
Disease Control Rate (DCR) & XX\% [95\% CI] & XX\% [95\% CI] & X.XXX \\
\quad Stable Disease (SD) & XX (XX\%) & XX (XX\%) & \\
Progressive Disease (PD) & XX (XX\%) & XX (XX\%) & \\
\midrule
Median Duration of Response (months) & X.X (95\% CI X.X-X.X) & X.X (95\% CI X.X-X.X) & X.XXX \\
\bottomrule
\end{tabular}
\caption{Best overall response by biomarker group (RECIST v1.1 criteria)}
\end{table}

\subsection{Survival Outcomes}

\textbf{Progression-Free Survival (PFS)}:
\begin{itemize}
\item Group A: Median X.X months (95\% CI X.X-X.X), 12-month PFS rate: XX\%
\item Group B: Median X.X months (95\% CI X.X-X.X), 12-month PFS rate: XX\%
\item Hazard Ratio: X.XX (95\% CI X.XX-X.XX), log-rank p = X.XXX
\item \textit{[Interpretation: Group A had XX\% reduction in risk of progression compared to Group B]}
\end{itemize}

\textbf{Overall Survival (OS)}:
\begin{itemize}
\item Group A: Median XX.X months (95\% CI XX.X-XX.X), 12-month OS rate: XX\%
\item Group B: Median XX.X months (95\% CI XX.X-XX.X), 12-month OS rate: XX\%
\item Hazard Ratio: X.XX (95\% CI X.XX-X.XX), log-rank p = X.XXX
\item \textit{[Interpretation: XX\% reduction in risk of death for Group A]}
\end{itemize}

% Note: Include Kaplan-Meier curves as figures if available
% \begin{figure}[H]
% \centering
% \includegraphics[width=0.9\textwidth]{figures/pfs_by_biomarker.pdf}
% \caption{Progression-free survival by biomarker status}
% \end{figure}

\section{Safety and Tolerability}

\begin{table}[H]
\centering
\small
\begin{tabular}{lcccc}
\toprule
\multirow{2}{*}{\textbf{Adverse Event}} & \multicolumn{2}{c}{\textbf{Any Grade, n (\%)}} & \multicolumn{2}{c}{\textbf{Grade 3-4, n (\%)}} \\
\cmidrule(lr){2-3} \cmidrule(lr){4-5}
& Group A & Group B & Group A & Group B \\
\midrule
[AE 1 - e.g., Fatigue] & XX (XX\%) & XX (XX\%) & X (X\%) & X (X\%) \\
[AE 2 - e.g., Nausea] & XX (XX\%) & XX (XX\%) & X (X\%) & X (X\%) \\
[AE 3 - e.g., Neutropenia] & XX (XX\%) & XX (XX\%) & X (X\%) & X (X\%) \\
[AE 4 - e.g., Diarrhea] & XX (XX\%) & XX (XX\%) & X (X\%) & X (X\%) \\
[AE 5 - immune-related] & XX (XX\%) & XX (XX\%) & X (X\%) & X (X\%) \\
\midrule
Treatment discontinuation & XX (XX\%) & XX (XX\%) & \multicolumn{2}{c}{-} \\
Dose reductions & XX (XX\%) & XX (XX\%) & \multicolumn{2}{c}{-} \\
\bottomrule
\end{tabular}
\caption{Treatment-emergent adverse events by biomarker group (CTCAE v5.0)}
\end{table}

\section{Statistical Analysis}

\subsection{Methods}

\textbf{Study Design}: [Retrospective cohort analysis / Prospective cohort / Post-hoc analysis of clinical trial]

\textbf{Statistical Tests}:
\begin{itemize}
\item Continuous variables: [t-test / Mann-Whitney U test], reported as [mean $\pm$ SD / median [IQR]]
\item Categorical variables: Chi-square test or Fisher's exact test (if expected count <5)
\item Survival analysis: Kaplan-Meier method, log-rank test, Cox proportional hazards regression
\item Significance level: Two-sided p<0.05 considered statistically significant
\item Software: [R version X.X.X, survival package / SAS / Stata / Python lifelines]
\end{itemize}

\subsection{Multivariable Analysis}

Cox regression model adjusting for baseline prognostic factors:

\begin{table}[H]
\centering
\small
\begin{tabular}{lccc}
\toprule
\textbf{Variable} & \textbf{Hazard Ratio} & \textbf{95\% CI} & \textbf{p-value} \\
\midrule
Biomarker+ (vs Biomarker-) & X.XX & X.XX-X.XX & X.XXX \\
Age (per 10 years) & X.XX & X.XX-X.XX & X.XXX \\
ECOG PS 2 (vs 0-1) & X.XX & X.XX-X.XX & X.XXX \\
Stage IV (vs III) & X.XX & X.XX-X.XX & X.XXX \\
[Additional variable] & X.XX & X.XX-X.XX & X.XXX \\
\bottomrule
\end{tabular}
\caption{Multivariable Cox regression for progression-free survival}
\end{table}

\textbf{Interpretation}: After adjusting for age, performance status, and disease stage, [biomarker status] remained an independent predictor of [PFS/OS] (HR X.XX, 95\% CI X.XX-X.XX, p=X.XXX).

\section{Clinical Implications}

\begin{tcolorbox}[colback=highlightgreen!10,colframe=highlightgreen,title=\textbf{Treatment Recommendations},fonttitle=\bfseries\small]
{\small
\textbf{For Biomarker-Positive Patients (Group A)}:

\textbf{Preferred Regimen} (GRADE 1A):
\begin{itemize}
\item [Specific treatment based on biomarker]
\item Evidence: [Trial name/data showing benefit in biomarker+ population]
\item Expected outcomes: ORR XX\%, median PFS XX months
\end{itemize}

\textbf{Monitoring}:
\begin{itemize}
\item Imaging every [X weeks] for response assessment
\item [Specific lab monitoring for biomarker+ patients]
\item Watch for [specific toxicities more common in this group]
\end{itemize}

\textbf{For Biomarker-Negative Patients (Group B)}:

\textbf{Standard Regimen} (GRADE 1B):
\begin{itemize}
\item [Standard therapy for biomarker- population]
\item Expected outcomes: ORR XX\%, median PFS XX months
\item Consider [alternative approaches or clinical trial enrollment]
\end{itemize}
}
\end{tcolorbox}

\section{Subgroup Analyses}

\textbf{Interaction Testing}: Treatment effect by biomarker subgroup (p-interaction = X.XXX)

[Describe whether treatment benefit differs by biomarker status - i.e., predictive biomarker]

Additional exploratory subgroups:
\begin{itemize}
\item Age <65 vs $\geq$65 years
\item Sex (male vs female)
\item Prior lines of therapy (0 vs 1+ prior treatments)
\item Disease burden (high vs low tumor burden)
\end{itemize}

\section{Strengths and Limitations}

\subsection{Strengths}
\begin{itemize}
\item [e.g., Biomarker-stratified analysis with prospectively defined groups]
\item [e.g., Adequate sample size for statistical power]
\item [e.g., Standardized response assessment using RECIST v1.1]
\item [e.g., Multivariable analysis adjusting for confounders]
\end{itemize}

\subsection{Limitations}
\begin{itemize}
\item [e.g., Retrospective design with potential selection bias]
\item [e.g., Single-institution cohort may limit generalizability]
\item [e.g., Biomarker testing not available for all patients (XX\% tested)]
\item [e.g., Limited follow-up for OS (median X months)]
\item [e.g., Heterogeneous treatment regimens across cohort]
\end{itemize}

\section{Conclusions}

[Paragraph summarizing key findings]

[Biomarker-positive patients demonstrated [significantly better/worse] outcomes compared to biomarker-negative patients, with [outcome metric] of [values] (HR X.XX, p=X.XXX). These findings support [biomarker-guided therapy selection / routine biomarker testing / specific treatment approach].]

[Future directions: Prospective validation in independent cohort, investigation of mechanisms, clinical trial design implications]

\section{References}

\begin{enumerate}
\item [Reference 1 - Key clinical trial]
\item [Reference 2 - Biomarker validation study]
\item [Reference 3 - Guideline reference (NCCN, ASCO, ESMO)]
\item [Reference 4 - Statistical methods reference]
\item [Reference 5 - Additional supporting evidence]
\end{enumerate}

\vspace{10pt}

\hrule
\vspace{4pt}
{\footnotesize
\textbf{Report Prepared By}: [Name, Title]\\
\textbf{Date}: [Date]\\
\textbf{Contact}: [Email/Phone]\\
\textbf{Institutional Review}: [IRB approval number if applicable]\\
\textbf{Data Cut-Off Date}: [Date]\\
\textbf{Confidentiality}: This document contains proprietary clinical data. Distribution restricted to authorized personnel only.
}

\end{document}

