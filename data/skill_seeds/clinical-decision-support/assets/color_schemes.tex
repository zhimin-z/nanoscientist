% Clinical Decision Support Color Schemes
% For use in LaTeX documents

% ============================================================================
% PRIMARY THEME COLORS
% ============================================================================

% Header and structural elements
\definecolor{headerblue}{RGB}{0,102,204}        % Section headers, titles
\definecolor{highlightgray}{RGB}{240,240,240}   % Background boxes

% ============================================================================
% RECOMMENDATION STRENGTH COLORS
% ============================================================================

% Strong recommendations (benefits clearly outweigh risks)
\definecolor{stronggreen}{RGB}{0,153,76}        % Grade 1A, 1B
\definecolor{strongdark}{RGB}{0,120,60}         % Darker variant for emphasis

% Conditional recommendations (trade-offs exist)
\definecolor{conditionalyellow}{RGB}{255,193,7} % Grade 2A, 2B, 2C
\definecolor{conditionalamber}{RGB}{255,160,0}  % Darker variant

% Research/Investigational (insufficient evidence)
\definecolor{researchblue}{RGB}{33,150,243}     % Clinical trials
\definecolor{researchdark}{RGB}{25,118,210}     % Darker variant

% Not recommended / Contraindicated
\definecolor{warningred}{RGB}{204,0,0}          % Strong recommendation against
\definecolor{dangerred}{RGB}{220,20,60}         % Critical warnings, urgent actions

% ============================================================================
% URGENCY LEVELS (Clinical Pathways)
% ============================================================================

\definecolor{urgentred}{RGB}{220,20,60}         % Immediate action (<1 hour)
\definecolor{semiurgent}{RGB}{255,152,0}        % Action within 24 hours  
\definecolor{routineblue}{RGB}{100,181,246}     % Routine care (>24 hours)
\definecolor{actiongreen}{RGB}{0,153,76}        % Standard interventions

% ============================================================================
% BIOMARKER CATEGORIES
% ============================================================================

% Alteration types
\definecolor{mutationred}{RGB}{244,67,54}       % Point mutations, SNVs
\definecolor{amplificationblue}{RGB}{33,150,243} % Copy number gains
\definecolor{deletionpurple}{RGB}{156,39,176}   % Copy number losses
\definecolor{fusionpurple}{RGB}{156,39,176}     % Gene fusions/rearrangements
\definecolor{expressionorange}{RGB}{255,152,0}  % Expression alterations

% Actionability tiers
\definecolor{tier1green}{RGB}{0,153,76}         % FDA-approved therapy
\definecolor{tier2orange}{RGB}{255,152,0}       % Clinical trial/off-label
\definecolor{tier3gray}{RGB}{158,158,158}       % VUS, no action

% ============================================================================
% STATISTICAL SIGNIFICANCE
% ============================================================================

\definecolor{significant}{RGB}{0,153,76}        % p < 0.05, statistically significant
\definecolor{trending}{RGB}{255,193,7}          % p = 0.05-0.10, trending
\definecolor{nonsignificant}{RGB}{158,158,158}  % p > 0.10, not significant

% ============================================================================
% OUTCOME CATEGORIES
% ============================================================================

% Response assessment (RECIST)
\definecolor{completeresponse}{RGB}{0,153,76}   % CR (complete response)
\definecolor{partialresponse}{RGB}{76,175,80}   % PR (partial response)
\definecolor{stabledisease}{RGB}{255,193,7}     % SD (stable disease)
\definecolor{progressivedisease}{RGB}{244,67,54} % PD (progressive disease)

% Survival outcomes
\definecolor{survivedgreen}{RGB}{0,153,76}      % Patient alive
\definecolor{eventred}{RGB}{244,67,54}          % Event occurred (death, progression)
\definecolor{censoredgray}{RGB}{158,158,158}    % Censored observation

% ============================================================================
% ADVERSE EVENT SEVERITY (CTCAE)
% ============================================================================

\definecolor{grade1}{RGB}{255,235,59}           % Mild
\definecolor{grade2}{RGB}{255,193,7}            % Moderate
\definecolor{grade3}{RGB}{255,152,0}            % Severe
\definecolor{grade4}{RGB}{244,67,54}            % Life-threatening
\definecolor{grade5}{RGB}{198,40,40}            % Fatal

% ============================================================================
% COLORBLIND-SAFE PALETTE (Okabe-Ito)
% ============================================================================
% Use these for graphs/figures to ensure accessibility

\definecolor{okabe1}{RGB}{230,159,0}            % Orange
\definecolor{okabe2}{RGB}{86,180,233}           % Sky blue
\definecolor{okabe3}{RGB}{0,158,115}            % Bluish green
\definecolor{okabe4}{RGB}{240,228,66}           % Yellow
\definecolor{okabe5}{RGB}{0,114,178}            % Blue
\definecolor{okabe6}{RGB}{213,94,0}             % Vermillion
\definecolor{okabe7}{RGB}{204,121,167}          % Reddish purple

% ============================================================================
% USAGE EXAMPLES
% ============================================================================

% Example 1: Strong recommendation box
% \begin{tcolorbox}[enhanced,colback=stronggreen!10,colframe=stronggreen,
%   title={\textbf{STRONG RECOMMENDATION} \hfill \textbf{GRADE: 1A}}]
%   We recommend osimertinib for EGFR-mutated NSCLC...
% \end{tcolorbox}

% Example 2: Conditional recommendation box
% \begin{tcolorbox}[enhanced,colback=conditionalyellow!10,colframe=conditionalyellow,
%   title={\textbf{CONDITIONAL RECOMMENDATION} \hfill \textbf{GRADE: 2B}}]
%   We suggest considering maintenance therapy...
% \end{tcolorbox}

% Example 3: Biomarker alteration
% \colorbox{mutationred!60}{\textcolor{white}{\textbf{MUTATION}}}

% Example 4: Statistical significance in table
% \cellcolor{significant!20} p < 0.001

% Example 5: Adverse event severity
% \textcolor{grade3}{Grade 3} or \colorbox{grade3!30}{Grade 3}

% ============================================================================
% ACCESSIBILITY NOTES
% ============================================================================

% 1. Always use sufficient color contrast (4.5:1 ratio for normal text)
% 2. Do not rely on color alone - use symbols/text as well
% 3. Test in grayscale to ensure readability
% 4. Use Okabe-Ito palette for colorblind accessibility in figures
% 5. Add text labels to colored boxes ("STRONG", "CONDITIONAL", etc.)

% ============================================================================
% STYLE CONSISTENCY
% ============================================================================

% Font: Helvetica (sans-serif) for clinical documents
% Margins: 0.5 inches for compact professional appearance
% Font sizes: 10pt body, 11pt subsections, 12-14pt headers
% Line spacing: Compact (minimal whitespace for dense information)
% Boxes: tcolorbox with rounded corners, colored backgrounds at 10-20% opacity

% End of color scheme definitions

