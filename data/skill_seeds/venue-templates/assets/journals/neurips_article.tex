% NeurIPS Conference Paper Template
% For submission to Neural Information Processing Systems (NeurIPS)
% Last updated: 2024
% Note: Use the official neurips_2024.sty file from the conference website

\documentclass{article}

% Required packages (neurips_2024.sty provides most formatting)
\usepackage{neurips_2024}  % Official NeurIPS style file (download from conference site)

% Recommended packages
\usepackage{amsmath}
\usepackage{amssymb}
\usepackage{amsthm}
\usepackage{graphicx}
\usepackage{algorithm}
\usepackage{algorithmic}
\usepackage{hyperref}
\usepackage{url}
\usepackage{booktabs}  % For better tables
\usepackage{multirow}
\usepackage{microtype}  % Improved typography

% Theorems, lemmas, etc.
\newtheorem{theorem}{Theorem}
\newtheorem{lemma}{Lemma}
\newtheorem{proposition}{Proposition}
\newtheorem{corollary}{Corollary}
\newtheorem{definition}{Definition}

% Title and Authors
\title{Your Paper Title: Concise and Descriptive \\ (Maximum Two Lines)}

% Authors - ANONYMIZED for initial submission
% For initial submission (double-blind review):
\author{
  Anonymous Authors \\
  Anonymous Institution(s) \\
}

% For camera-ready version (after acceptance):
% \author{
%   First Author \\
%   Department of Computer Science \\
%   University Name \\
%   City, State, Postal Code \\
%   \texttt{first.author@university.edu} \\
%   \And
%   Second Author \\
%   Company/Institution Name \\
%   Address \\
%   \texttt{second.author@company.com} \\
%   \And
%   Third Author \\
%   Institution \\
%   \texttt{third.author@institution.edu}
% }

\begin{document}

\maketitle

\begin{abstract}
Write a concise abstract (150-250 words) summarizing your contributions. The abstract should clearly state: (1) the problem you address, (2) your approach/method, (3) key results/findings, and (4) significance/implications. Make it accessible to a broad machine learning audience.
\end{abstract}

\section{Introduction}
\label{sec:introduction}

Introduce the problem you're addressing and its significance in machine learning or AI. Motivate why this problem is important and challenging.

\subsection{Background and Motivation}
Provide necessary background for understanding your work. Explain the gap in current methods or knowledge.

\subsection{Contributions}
Clearly state your main contributions as a bulleted list:
\begin{itemize}
    \item First contribution: e.g., "We propose a novel architecture for..."
    \item Second contribution: e.g., "We provide theoretical analysis showing..."
    \item Third contribution: e.g., "We demonstrate state-of-the-art performance on..."
\end{itemize}

\subsection{Paper Organization}
Briefly describe the structure of the remainder of the paper.

\section{Related Work}
\label{sec:related}

Discuss relevant prior work and how your work differs. Organize by themes or approaches rather than chronologically. Be fair and accurate in describing others' work.

Cite key papers \cite{lecun2015deep, vaswani2017attention, devlin2019bert} and explain how your work builds upon or differs from them.

\section{Problem Formulation}
\label{sec:problem}

Formally define the problem you're solving. Include mathematical notation and definitions.

\subsection{Notation}
Define your notation clearly. For example:
\begin{itemize}
    \item $\mathcal{X}$: input space
    \item $\mathcal{Y}$: output space
    \item $f: \mathcal{X} \rightarrow \mathcal{Y}$: function to learn
    \item $\mathcal{D} = \{(x_i, y_i)\}_{i=1}^n$: training dataset
\end{itemize}

\subsection{Objective}
State your learning objective formally, e.g.:
\begin{equation}
\min_{\theta} \mathbb{E}_{(x,y) \sim \mathcal{D}} \left[ \mathcal{L}(f_\theta(x), y) \right]
\end{equation}
where $\mathcal{L}$ is the loss function and $\theta$ are model parameters.

\section{Method}
\label{sec:method}

Describe your proposed method in detail. This is the core technical contribution of your paper.

\subsection{Model Architecture}
Describe the architecture of your model with sufficient detail for reproduction. Include figures if helpful.

\begin{figure}[t]
\centering
% \includegraphics[width=0.8\textwidth]{architecture.pdf}
\caption{Model architecture diagram. Describe the key components and data flow. Use colorblind-safe colors.}
\label{fig:architecture}
\end{figure}

\subsection{Training Procedure}
Explain how you train the model, including:
\begin{algorithm}[t]
\caption{Training Algorithm}
\label{alg:training}
\begin{algorithmic}[1]
\STATE \textbf{Input:} Training data $\mathcal{D}$, learning rate $\alpha$
\STATE \textbf{Output:} Trained parameters $\theta$
\STATE Initialize $\theta$ randomly
\FOR{epoch $= 1$ to $T$}
    \FOR{batch $(x, y)$ in $\mathcal{D}$}
        \STATE Compute loss: $\mathcal{L} = \mathcal{L}(f_\theta(x), y)$
        \STATE Update: $\theta \leftarrow \theta - \alpha \nabla_\theta \mathcal{L}$
    \ENDFOR
\ENDFOR
\RETURN $\theta$
\end{algorithmic}
\end{algorithm}

\subsection{Key Components}
Describe key technical innovations or components in detail.

\section{Theoretical Analysis}
\label{sec:theory}

If applicable, provide theoretical analysis of your method.

\begin{theorem}
\label{thm:main}
State your main theoretical result here.
\end{theorem}

\begin{proof}
Provide proof or sketch of proof. Full proofs can go in the appendix.
\end{proof}

\section{Experiments}
\label{sec:experiments}

Describe your experimental setup and results.

\subsection{Experimental Setup}
\textbf{Datasets:} Describe datasets used (e.g., ImageNet, CIFAR-10, etc.).

\textbf{Baselines:} List baseline methods for comparison.

\textbf{Implementation Details:} Provide implementation details including hyperparameters, hardware, training time.

\textbf{Evaluation Metrics:} Define metrics used (accuracy, F1, AUC, etc.).

\subsection{Main Results}
Present your main experimental results.

\begin{table}[t]
\centering
\caption{Performance comparison on benchmark datasets. Bold indicates best performance. Results reported as mean ± std over 3 runs.}
\label{tab:main_results}
\begin{tabular}{lcccc}
\toprule
Method & Dataset 1 & Dataset 2 & Dataset 3 & Average \\
\midrule
Baseline 1 & 85.3 ± 0.5 & 72.1 ± 0.8 & 90.2 ± 0.3 & 82.5 \\
Baseline 2 & 87.2 ± 0.4 & 74.5 ± 0.6 & 91.1 ± 0.5 & 84.3 \\
\textbf{Our Method} & \textbf{91.7 ± 0.3} & \textbf{79.8 ± 0.5} & \textbf{94.3 ± 0.2} & \textbf{88.6} \\
\bottomrule
\end{tabular}
\end{table}

\subsection{Ablation Studies}
Conduct ablation studies to understand which components contribute to performance.

\subsection{Analysis}
Provide deeper analysis of results, failure cases, limitations, etc.

\section{Discussion}
\label{sec:discussion}

Discuss your findings, limitations, and broader implications.

\subsection{Limitations}
Honestly acknowledge limitations of your work.

\subsection{Broader Impacts}
Discuss potential positive and negative societal impacts (required by NeurIPS).

\section{Conclusion}
\label{sec:conclusion}

Summarize your main contributions and findings. Suggest future research directions.

% Acknowledgments (add after acceptance, not in submission version)
\section*{Acknowledgments}
Thank collaborators, funding sources (with grant numbers), and compute resources. Not included in double-blind submission.

% References
\bibliographystyle{plainnat}  % or other NeurIPS-compatible style
\bibliography{references}  % Your .bib file

% Appendix (optional, unlimited pages)
\appendix

\section{Additional Proofs}
\label{app:proofs}

Provide full proofs of theorems here.

\section{Additional Experimental Results}
\label{app:experiments}

Include additional experiments, more ablations, qualitative results, etc.

\section{Hyperparameters}
\label{app:hyperparameters}

List all hyperparameters used in experiments for reproducibility.

\begin{table}[h]
\centering
\caption{Hyperparameters used in all experiments}
\begin{tabular}{ll}
\toprule
Hyperparameter & Value \\
\midrule
Learning rate & 0.001 \\
Batch size & 64 \\
Optimizer & Adam \\
Weight decay & 0.0001 \\
Epochs & 100 \\
\bottomrule
\end{tabular}
\end{table}

\section{Code and Data}
\label{app:code}

Provide links to code repository (anonymized for review, e.g., anonymous GitHub):
\begin{itemize}
    \item Code: \url{https://anonymous.4open.science/r/project-XXXX}
    \item Data: Available upon request / at [repository]
\end{itemize}

\end{document}

% Notes for Authors:
% 1. Main paper: 8 pages maximum (excluding references and appendix)
% 2. References: unlimited pages
% 3. Appendix: unlimited pages (reviewed at discretion of reviewers)
% 4. Use double-blind anonymization for initial submission
% 5. Include broader impact statement
% 6. Code submission strongly encouraged (anonymous for review)
% 7. Use official neurips_2024.sty file (download from NeurIPS website)
% 8. Font: Times, 10pt (enforced by style file)
% 9. Figures should be colorblind-friendly
% 10. Ensure reproducibility: report seeds, hyperparameters, dataset splits

