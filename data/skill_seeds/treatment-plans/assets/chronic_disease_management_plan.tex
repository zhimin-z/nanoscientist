% Chronic Disease Management Plan Template
% For long-term management of multiple chronic conditions
% Last updated: 2025

\documentclass[11pt,letterpaper]{article}

% Packages
\usepackage[top=1in,bottom=1in,left=1in,right=1in]{geometry}
\usepackage{amsmath,amssymb}
\usepackage[utf8]{inputenc}
\usepackage{graphicx}
\usepackage{array}
\usepackage{longtable}
\usepackage{booktabs}
\usepackage{enumitem}
\usepackage{xcolor}
\usepackage{fancyhdr}
\usepackage{lastpage}
\usepackage{tabularx}
\usepackage[most]{tcolorbox}

% Header and footer
\pagestyle{fancy}
\fancyhf{}
\lhead{Chronic Disease Management Plan}
\rhead{Page \thepage\ of \pageref{LastPage}}
\lfoot{Date Created: \today}
\rfoot{Confidential Patient Information}

% Title formatting
\usepackage{titlesec}
\titleformat{\section}{\large\bfseries}{\thesection}{1em}{}
\titleformat{\subsection}{\normalsize\bfseries}{\thesubsection}{1em}{}

\begin{document}

% Title
\begin{center}
{\Large\bfseries CHRONIC DISEASE MANAGEMENT PLAN}\\[0.5em]
{\large Comprehensive Long-Term Care Coordination}\\[0.5em]
\rule{\textwidth}{1pt}
\end{center}

\vspace{1em}

% ===== TREATMENT PLAN HIGHLIGHTS (Foundation Medicine Model) =====
\begin{tcolorbox}[colback=orange!5!white,colframe=orange!75!black,title=\textbf{TREATMENT PLAN HIGHLIGHTS},fonttitle=\bfseries\large]

\textbf{Key Diagnoses:} [Primary chronic conditions - e.g., Type 2 Diabetes, CHF (NYHA II), CKD Stage 3]

\vspace{0.3em}
\textbf{Primary Treatment Goals:}
\begin{itemize}[leftmargin=*,itemsep=0pt]
    \item [Goal 1 - e.g., Maintain HbA1c $<$7.5\% and prevent diabetic complications]
    \item [Goal 2 - e.g., Optimize heart failure management, prevent hospitalizations]
    \item [Goal 3 - e.g., Slow CKD progression, maintain eGFR $>$45 mL/min]
\end{itemize}

\vspace{0.3em}
\textbf{Main Interventions:}
\begin{itemize}[leftmargin=*,itemsep=0pt]
    \item \textit{Medications:} [Core regimen - e.g., Metformin, Lisinopril, Furosemide, statin therapy]
    \item \textit{Lifestyle:} [Key modifications - e.g., Low-sodium diet, fluid restriction, regular exercise]
    \item \textit{Monitoring:} [Essential tracking - e.g., Daily weights, BP, glucose; quarterly labs]
\end{itemize}

\vspace{0.3em}
\textbf{Timeline:} [Care model - e.g., Monthly visits initially, then quarterly; annual comprehensive review]

\end{tcolorbox}

\vspace{1em}

% ===== SECTION 1: PATIENT INFORMATION =====
\section*{1. Patient Information and Problem List}

\textbf{HIPAA Notice}: De-identify all protected health information before sharing.

\vspace{0.5em}

\begin{tabularx}{\textwidth}{|l|X|}
\hline
\textbf{Patient ID} & [De-identified code, e.g., CDM-001] \\ \hline
\textbf{Age Range} & [e.g., 60-65 years] \\ \hline
\textbf{Sex} & [Male/Female/Other] \\ \hline
\textbf{Date of Plan} & [Month/Year only] \\ \hline
\textbf{Primary Care Provider} & [Name, MD/DO, Credentials] \\ \hline
\textbf{Care Coordinator} & [Name, RN/NP/PA, if applicable] \\ \hline
\textbf{Facility/System} & [Healthcare organization] \\ \hline
\end{tabularx}

\vspace{1em}

\subsection*{Active Problem List (Prioritized)}

\begin{longtable}{|c|p{4cm}|c|p{3cm}|p{3.5cm}|}
\hline
\textbf{\#} & \textbf{Condition} & \textbf{ICD-10} & \textbf{Status} & \textbf{Specialists} \\ \hline
\endfirsthead
\hline
\textbf{\#} & \textbf{Condition} & \textbf{ICD-10} & \textbf{Status} & \textbf{Specialists} \\ \hline
\endhead
1 & Type 2 Diabetes Mellitus & E11.65 & Suboptimal control (HbA1c 8.2\%) & Endocrinology \\ \hline
2 & Chronic Heart Failure (HFrEF) & I50.22 & Stable, NYHA Class II & Cardiology \\ \hline
3 & Chronic Kidney Disease Stage 3b & N18.31 & Stable, eGFR 38 & Nephrology (as needed) \\ \hline
4 & Hypertension & I10 & Well-controlled on meds & PCP \\ \hline
5 & Hyperlipidemia & E78.5 & On statin, LDL at goal & PCP \\ \hline
6 & Obstructive Sleep Apnea & G47.33 & On CPAP, adherent & Sleep Medicine \\ \hline
7 & Obesity & E66.9 & BMI 34, stable weight & PCP, Nutrition \\ \hline
8 & Osteoarthritis, bilateral knees & M17.0 & Managed conservatively & Ortho (prn) \\ \hline
[Add rows] & & & & \\ \hline
\end{longtable}

\subsection*{Current Medication List}

\textit{Reconciled as of [Date]. Total: [X] medications}

\begin{longtable}{|p{3cm}|p{2cm}|p{1.8cm}|p{3cm}|p{3.5cm}|}
\hline
\textbf{Medication} & \textbf{Dose} & \textbf{Frequency} & \textbf{Indication} & \textbf{Prescriber} \\ \hline
\endfirsthead
\hline
\textbf{Medication} & \textbf{Dose} & \textbf{Frequency} & \textbf{Indication} & \textbf{Prescriber} \\ \hline
\endhead
Metformin ER & 1000mg & BID & Diabetes & PCP \\ \hline
Insulin glargine & 24 units & QHS & Diabetes & Endocrinology \\ \hline
Carvedilol & 12.5mg & BID & Heart failure, HTN & Cardiology \\ \hline
Lisinopril & 40mg & Daily & Heart failure, HTN, CKD protection & Cardiology \\ \hline
Furosemide & 40mg & Daily & Heart failure (diuresis) & Cardiology \\ \hline
Atorvastatin & 40mg & QHS & Hyperlipidemia, ASCVD prevention & PCP \\ \hline
Aspirin & 81mg & Daily & ASCVD prevention & PCP \\ \hline
[Continue list] & & & & \\ \hline
\end{longtable}

\subsection*{Care Team and Specialists}

\begin{itemize}[leftmargin=*]
    \item \textbf{Primary Care Provider}: [Name, practice] - Coordinates overall care
    \item \textbf{Cardiology}: [Name] - Heart failure management
    \item \textbf{Endocrinology}: [Name] - Diabetes optimization
    \item \textbf{Nephrology}: [Name if engaged] - CKD monitoring
    \item \textbf{Care Coordinator/Navigator}: [Name] - Appointment coordination, patient education
    \item \textbf{Pharmacist}: [Clinical pharmacist if available] - Medication reconciliation, optimization
    \item \textbf{Registered Dietitian}: [Name] - Medical nutrition therapy
    \item \textbf{Social Worker}: [Name if engaged] - Psychosocial support, resources
\end{itemize}

% ===== SECTION 2: DISEASE-SPECIFIC ASSESSMENTS =====
\section*{2. Disease-Specific Assessments and Status}

\subsection*{2.1 Type 2 Diabetes Mellitus}

\textbf{Current Status}: Suboptimal control
\begin{itemize}[leftmargin=*]
    \item \textbf{HbA1c}: 8.2\% (target $<$7\%)
    \item \textbf{Fasting Glucose}: Average 165 mg/dL (target 80-130)
    \item \textbf{Time in Range}: Approximately 55\% (target $>$70\%)
    \item \textbf{Hypoglycemia}: Infrequent, 1-2 episodes/month (BG 65-70)
    \item \textbf{Duration}: 12 years
    \item \textbf{Complications Screening}:
    \begin{itemize}
        \item Retinopathy: Mild NPDR, followed by ophthalmology
        \item Nephropathy: CKD stage 3b, urine ACR 180 mg/g (albuminuria)
        \item Neuropathy: Mild peripheral neuropathy, no foot ulcers
        \item Cardiovascular: History of heart failure
    \end{itemize}
\end{itemize}

\subsection*{2.2 Chronic Heart Failure (HFrEF)}

\textbf{Current Status}: Stable, NYHA Class II
\begin{itemize}[leftmargin=*]
    \item \textbf{Ejection Fraction}: 35\% (reduced, HFrEF)
    \item \textbf{Etiology}: Ischemic cardiomyopathy (prior MI 5 years ago)
    \item \textbf{NYHA Class}: II - Slight limitation, comfortable at rest, symptoms with ordinary activity
    \item \textbf{Symptoms}: Mild dyspnea on exertion, no orthopnea/PND, occasional LE edema
    \item \textbf{Weight}: Stable, patient monitors daily
    \item \textbf{GDMT Status}:
    \begin{itemize}
        \item ACE inhibitor: Lisinopril 40mg daily (at target dose)
        \item Beta-blocker: Carvedilol 12.5mg BID (target 25mg BID - limited by fatigue)
        \item Diuretic: Furosemide 40mg daily
        \item Need to consider: SGLT2 inhibitor (also beneficial for diabetes), ARNI
    \end{itemize}
    \item \textbf{Device Therapy}: No ICD/CRT currently, discussed with cardiology
\end{itemize}

\subsection*{2.3 Chronic Kidney Disease Stage 3b}

\textbf{Current Status}: Stable
\begin{itemize}[leftmargin=*]
    \item \textbf{eGFR}: 38 mL/min/1.73m² (Stage 3b, moderate-severe decrease)
    \item \textbf{Creatinine}: 1.8 mg/dL (stable)
    \item \textbf{Urine Albumin}: ACR 180 mg/g (albuminuria, from diabetes)
    \item \textbf{Etiology}: Diabetic nephropathy, hypertensive nephropathy
    \item \textbf{Progression Risk}: Moderate-high (diabetes, albuminuria)
    \item \textbf{Complications}: Anemia (Hgb 11.2), managed with iron supplementation
    \item \textbf{Renal Protection}: ACE inhibitor, BP control, glucose control, limit nephrotoxins
\end{itemize}

\subsection*{2.4 Additional Conditions Summary}

\begin{itemize}[leftmargin=*]
    \item \textbf{Hypertension}: Well-controlled, average home BP 128/78 mmHg
    \item \textbf{Hyperlipidemia}: LDL 65 mg/dL (at goal $<$70 for ASCVD), on statin
    \item \textbf{Obstructive Sleep Apnea}: On CPAP nightly, AHI reduced from 32 to 4, good adherence
    \item \textbf{Obesity}: BMI 34, weight stable, difficulty with weight loss due to HF exercise limitations
    \item \textbf{Osteoarthritis}: Bilateral knee pain, managed with acetaminophen, PT, avoid NSAIDs (CKD)
\end{itemize}

% ===== SECTION 3: INTEGRATED GOALS =====
\section*{3. Integrated Treatment Goals (SMART Format)}

\subsection*{3.1 Short-Term Goals (3-6 months)}

\textbf{Diabetes Goals}:
\begin{enumerate}[leftmargin=*]
    \item Reduce HbA1c from 8.2\% to $<$7.5\% within 3 months by optimizing insulin dosing and medication adherence.
    \item Improve fasting glucose to 100-140 mg/dL range through medication adjustment and dietary changes within 3 months.
    \item Complete annual diabetic eye exam and foot exam within 1 month.
\end{enumerate}

\textbf{Heart Failure Goals}:
\begin{enumerate}[leftmargin=*]
    \item Maintain NYHA Class II status (no worsening) with daily weight monitoring and adherence to fluid/sodium restrictions.
    \item Add SGLT2 inhibitor for dual diabetes and heart failure benefit within 1 month.
    \item Improve exercise tolerance: Walk 15 minutes daily without dyspnea within 3 months.
\end{enumerate}

\textbf{CKD Goals}:
\begin{enumerate}[leftmargin=*]
    \item Maintain eGFR stability ($\pm$5 mL/min from baseline 38) over 6 months.
    \item Reduce urine albumin-to-creatinine ratio from 180 to $<$100 mg/g with BP and glucose control.
    \item Avoid nephrotoxic agents (NSAIDs, contrast without prophylaxis).
\end{enumerate}

\textbf{Cross-Cutting Goals}:
\begin{enumerate}[leftmargin=*]
    \item Medication adherence $>$90\% measured by refill rates and pill counts within 3 months.
    \item Weight loss of 5\% body weight (10 lbs) through diet modification within 6 months.
    \item Blood pressure maintenance at $<$130/80 mmHg (home average).
\end{enumerate}

\subsection*{3.2 Long-Term Goals (6-12 months)}

\begin{enumerate}[leftmargin=*]
    \item \textbf{Diabetes}: Achieve HbA1c $<$7\% and prevent progression of microvascular complications.
    \item \textbf{Heart Failure}: Optimize GDMT, prevent hospitalizations, maintain functional status.
    \item \textbf{CKD}: Slow progression (goal: $<$2 mL/min/year eGFR decline), delay need for dialysis.
    \item \textbf{Quality of Life}: Maintain independence in ADLs, engage in meaningful activities (gardening, grandchildren visits).
    \item \textbf{Prevention}: Up-to-date with all preventive care (vaccinations, cancer screenings).
    \item \textbf{Coordination}: Seamless care transitions, all providers aware of care plan, no conflicting treatments.
\end{enumerate}

\subsection*{3.3 Patient-Centered Priorities}

\begin{itemize}[leftmargin=*]
    \item \textbf{Priority 1}: "I don't want to end up on dialysis like my brother"
    \item \textbf{Priority 2}: "I want to keep up with my grandkids"
    \item \textbf{Priority 3}: "I want to reduce my medications if possible" (pill burden concern)
    \item \textbf{Priority 4}: "I want to avoid being hospitalized again"
\end{itemize}

% ===== SECTION 4: COMPREHENSIVE INTERVENTIONS =====
\section*{4. Comprehensive Interventions}

\subsection*{4.1 Medication Management and Optimization}

\textbf{Current Regimen Optimization}:

\begin{enumerate}[leftmargin=*]
    \item \textbf{ADD: Empagliflozin (Jardiance) 10mg daily}
    \begin{itemize}
        \item \textit{Rationale}: SGLT2 inhibitor provides dual benefit - improves diabetes control AND reduces HF hospitalizations/mortality (EMPEROR-Reduced trial). Also slows CKD progression.
        \item \textit{Monitoring}: eGFR (hold if $<$20), volume status, UTI symptoms, DKA risk (low in T2DM)
        \item \textit{Expected benefit}: HbA1c reduction 0.5-0.8\%, reduced HF events 25-30\%
    \end{itemize}
    
    \item \textbf{TITRATE: Insulin glargine}
    \begin{itemize}
        \item \textit{Current}: 24 units QHS, fasting BG averaging 165
        \item \textit{Plan}: Increase by 2 units every 3 days until fasting BG 100-130, patient to self-titrate with daily phone/portal check-ins
        \item \textit{Expected dose}: Likely 30-36 units
    \end{itemize}
    
    \item \textbf{OPTIMIZE: Beta-blocker (carvedilol)}
    \begin{itemize}
        \item \textit{Current}: 12.5mg BID (patient reports fatigue at higher doses)
        \item \textit{Plan}: Trial slow up-titration to 18.75mg BID, monitor for tolerance
        \item \textit{Goal}: Target dose 25mg BID for HFrEF mortality benefit
        \item \textit{Alternative}: Consider switching to different beta-blocker if intolerable
    \end{itemize}
    
    \item \textbf{CONTINUE}: ACE inhibitor (lisinopril 40mg) - at target dose
    
    \item \textbf{CONSIDER FUTURE}: Sacubitril/valsartan (Entresto) to replace lisinopril if HF symptoms progress
\end{enumerate}

\textbf{Medication Safety}:
\begin{itemize}[leftmargin=*]
    \item \textbf{Polypharmacy Review}: Current medication count [X], review quarterly for deprescribing opportunities
    \item \textbf{Renal Dosing}: All medications reviewed for CKD Stage 3b, adjust as needed
    \item \textbf{Drug Interactions}: Monitor K+ with ACE + diuretic, avoid NSAIDs (CKD, HF)
    \item \textbf{Adherence Support}: Pill organizer, medication list wallet card, automatic refills, pharmacy synchronization
\end{itemize}

\subsection*{4.2 Lifestyle and Self-Management Interventions}

\textbf{Dietary Management}:
\begin{itemize}[leftmargin=*]
    \item \textbf{Diabetes}:
    \begin{itemize}
        \item Carbohydrate consistency: 45-60g per meal
        \item Mediterranean diet pattern
        \item Limit refined sugars and processed carbohydrates
    \end{itemize}
    \item \textbf{Heart Failure}:
    \begin{itemize}
        \item Sodium restriction: $<$2000mg daily (low-sodium products, avoid processed foods)
        \item Fluid restriction: 1.5-2L daily if needed for volume management
    \end{itemize}
    \item \textbf{CKD}:
    \begin{itemize}
        \item Moderate protein intake: 0.8-1.0 g/kg/day
        \item Phosphorus and potassium awareness (but not severely restricted at Stage 3b)
    \end{itemize}
    \item \textbf{Weight Loss}: 500 kcal/day deficit for gradual weight loss
    \item \textbf{Referral}: Registered dietitian for medical nutrition therapy
\end{itemize}

\textbf{Physical Activity}:
\begin{itemize}[leftmargin=*]
    \item \textbf{Goal}: 150 min/week moderate activity (walking, swimming)
    \item \textbf{Heart Failure Considerations}: Start with 10-15 min sessions, gradually increase, monitor symptoms
    \item \textbf{Diabetes Benefits}: Improves insulin sensitivity, glucose control
    \item \textbf{Cardiac Rehabilitation}: Consider referral if not previously completed
    \item \textbf{Progression}: Track with pedometer/activity tracker, goal 7000-10,000 steps daily
\end{itemize}

\textbf{Self-Monitoring}:
\begin{itemize}[leftmargin=*]
    \item \textbf{Daily}:
    \begin{itemize}
        \item Weight (same time, same scale) - report gain $>$2-3 lbs in 2 days
        \item Blood glucose: Fasting and pre-dinner
        \item Blood pressure: Morning and evening
    \end{itemize}
    \item \textbf{Weekly}:
    \begin{itemize}
        \item Symptom check (dyspnea, edema, chest pain, hypoglycemia frequency)
        \item Medication adherence review
    \end{itemize}
    \item \textbf{Recording}: Use logbook or smartphone app (MyChart, Apple Health)
\end{itemize}

\textbf{Other Lifestyle Factors}:
\begin{itemize}[leftmargin=*]
    \item \textbf{CPAP Adherence}: Continue nightly use, download compliance data quarterly
    \item \textbf{Smoking}: [If applicable - cessation interventions]
    \item \textbf{Alcohol}: Limit to $\leq$1 drink/day (heart failure, diabetes management)
    \item \textbf{Stress Management}: Mindfulness, adequate sleep, social engagement
\end{itemize}

\subsection*{4.3 Disease-Specific Monitoring and Screening}

\textbf{Diabetes Monitoring}:
\begin{itemize}[leftmargin=*]
    \item HbA1c every 3 months until at goal, then every 6 months
    \item Lipid panel annually
    \item Urine albumin-to-creatinine ratio annually
    \item Comprehensive foot exam every visit, monofilament testing annually
    \item Dilated eye exam annually (ophthalmology)
    \item Dental exam every 6 months (periodontal disease link)
\end{itemize}

\textbf{Heart Failure Monitoring}:
\begin{itemize}[leftmargin=*]
    \item Daily weights, report significant changes
    \item BNP or NT-proBNP when symptoms change
    \item Echocardiogram annually or if clinical change
    \item EKG annually
    \item Functional assessment (6-minute walk test) periodically
\end{itemize}

\textbf{CKD Monitoring}:
\begin{itemize}[leftmargin=*]
    \item eGFR and creatinine every 3-6 months
    \item Urine ACR annually
    \item CBC (anemia), CMP (electrolytes, calcium, phosphorus) every 6 months
    \item Vitamin D, PTH if indicated
    \item Bone density scan (increased fracture risk)
\end{itemize}

\textbf{Preventive Care}:
\begin{itemize}[leftmargin=*]
    \item Influenza vaccine annually
    \item Pneumococcal vaccines (PCV20 or PCV15+PPSV23) per ACIP guidelines
    \item COVID-19 vaccination per current recommendations
    \item Zoster vaccine (Shingrix)
    \item Colorectal cancer screening per age guidelines
    \item [Other age/sex-appropriate screenings]
\end{itemize}

% ===== SECTION 5: CARE COORDINATION =====
\section*{5. Care Coordination and Communication}

\subsection*{Provider Communication Plan}

\begin{tabularx}{\textwidth}{|l|X|X|}
\hline
\textbf{Provider} & \textbf{Visit Frequency} & \textbf{Communication/Coordination} \\ \hline
Primary Care & Every 3 months & Care plan coordinator, medication reconciliation, preventive care \\ \hline
Cardiology & Every 4-6 months & HF medication optimization, EF monitoring, device consideration \\ \hline
Endocrinology & Every 3-4 months & Diabetes management, insulin titration, complications screening \\ \hline
Nephrology & As needed (if eGFR $<$30 or rapid decline) & CKD management, dialysis planning if needed \\ \hline
Dietitian & Monthly x3, then quarterly & Nutrition counseling, meal planning \\ \hline
Pharmacist & Quarterly & Medication review, adherence counseling, cost optimization \\ \hline
Care Coordinator & Monthly phone check-in & Appointment scheduling, barrier identification, education \\ \hline
\end{tabularx}

\subsection*{Information Sharing}

\begin{itemize}[leftmargin=*]
    \item Shared EHR access for all providers in health system
    \item Medication reconciliation after each specialist visit
    \item Lab results shared via patient portal and provider notifications
    \item Care plan accessible to all team members
    \item Patient carries medication list and problem list
\end{itemize}

\subsection*{Care Transitions Management}

\textbf{Hospital Discharge Protocol}:
\begin{itemize}[leftmargin=*]
    \item PCP notified within 24 hours of admission and discharge
    \item Follow-up appointment within 7 days of discharge
    \item Medication reconciliation at discharge and first follow-up
    \item Red flags review: HF exacerbation signs, hyperglycemia, AKI
\end{itemize}

\textbf{Specialty Referral Coordination}:
\begin{itemize}[leftmargin=*]
    \item Care coordinator ensures specialist appointments scheduled
    \item Specialist notes reviewed by PCP within 1 week
    \item Treatment recommendations integrated into care plan
    \item Conflicting recommendations discussed among providers
\end{itemize}

% ===== SECTION 6: MONITORING AND OUTCOMES =====
\section*{6. Monitoring Parameters and Quality Measures}

\subsection*{Clinical Outcomes Dashboard}

\begin{longtable}{|p{3.5cm}|p{2.5cm}|p{2cm}|p{2cm}|p{3cm}|}
\hline
\textbf{Parameter} & \textbf{Baseline} & \textbf{Target} & \textbf{Current} & \textbf{Frequency} \\ \hline
\endfirsthead
\hline
\textbf{Parameter} & \textbf{Baseline} & \textbf{Target} & \textbf{Current} & \textbf{Frequency} \\ \hline
\endhead
HbA1c & 8.2\% & $<$7\% & [update] & Q3-6 months \\ \hline
Fasting Glucose & 165 mg/dL & 100-130 & [update] & Daily (patient), labs Q3mo \\ \hline
Blood Pressure & 142/86 & $<$130/80 & [update] & Daily (patient), each visit \\ \hline
LDL Cholesterol & 65 mg/dL & $<$70 & At goal & Annually \\ \hline
eGFR & 38 mL/min & Stable ($\pm$5) & [update] & Every 3-6 months \\ \hline
Urine ACR & 180 mg/g & $<$100 & [update] & Annually \\ \hline
Weight & [baseline] lbs & -10 lbs (5\%) & [update] & Daily (patient), each visit \\ \hline
BNP/NT-proBNP & [if available] & Stable & [update] & When symptomatic \\ \hline
Ejection Fraction & 35\% & Monitor & [date of last echo] & Annually or if change \\ \hline
\end{longtable}

\subsection*{Quality Measure Tracking (HEDIS/CMS)}

\begin{itemize}[leftmargin=*]
    \item ✓ Diabetes HbA1c testing (every 6 months)
    \item ☐ Diabetes HbA1c control ($<$8\%) - \textit{Target: achieve}
    \item ✓ Diabetes eye exam (annual dilated)
    \item ☐ Diabetes medical attention for nephropathy (urine ACR) - \textit{Due [month]}
    \item ✓ Blood pressure control ($<$140/90 for diabetes)
    \item ✓ Statin therapy for ASCVD
    \item ✓ ACE/ARB therapy for diabetes with hypertension
    \item ✓ Beta-blocker for HFrEF
    \item ☐ Flu vaccine current year - \textit{Due [month]}
    \item ✓ Pneumococcal vaccine
\end{itemize}

% ===== SECTION 7: PATIENT EDUCATION AND ACTIVATION =====
\section*{7. Patient Education and Self-Management Support}

\subsection*{Disease Education Completed}

\begin{itemize}[leftmargin=*]
    \item \textbf{Diabetes}: Pathophysiology, complications, importance of glucose control, hypoglycemia recognition
    \item \textbf{Heart Failure}: How heart failure affects body, medication importance, fluid/sodium restrictions, warning signs
    \item \textbf{CKD}: Kidney function, progression risk, renal protection strategies, medication precautions
    \item \textbf{Medication Purposes}: Why each medication is prescribed, expected benefits
    \item \textbf{Lifestyle Impact}: How diet, exercise, weight loss benefit all conditions
\end{itemize}

\subsection*{Self-Management Skills Training}

\begin{itemize}[leftmargin=*]
    \item ✓ Blood glucose monitoring technique
    \item ✓ Insulin injection technique and storage
    \item ✓ Home blood pressure monitoring
    \item ✓ Daily weight tracking and interpretation
    \item ✓ Symptom recognition (HF exacerbation, hypoglycemia, hyperglycemia)
    \item ✓ Medication organization (pill box use)
    \item ☐ Dietary skills: Carb counting, label reading, low-sodium food selection
    \item ☐ Sick day management (when to call, medication adjustments)
\end{itemize}

\subsection*{Warning Signs - When to Call Provider}

\textbf{Call office same day or go to ED if}:
\begin{itemize}[leftmargin=*]
    \item Weight gain $>$2-3 lbs in 2 days or 5 lbs in 1 week (heart failure)
    \item Increased shortness of breath, cannot lie flat, new leg swelling
    \item Chest pain or pressure
    \item Blood glucose consistently $>$300 or $<$60 mg/dL
    \item Decreased urine output, dark urine, swelling
    \item Dizziness, lightheadedness, syncope
\end{itemize}

\subsection*{Resources and Support}

\begin{itemize}[leftmargin=*]
    \item Diabetes self-management education program (DSMES)
    \item Cardiac rehabilitation program
    \item Patient portal for lab results, messaging, educational materials
    \item American Diabetes Association (diabetes.org) resources
    \item American Heart Association (heart.org) HF information
    \item National Kidney Foundation (kidney.org) CKD education
    \item Local support groups [if available]
\end{itemize}

% ===== SECTION 8: CONTINGENCY PLANNING =====
\section*{8. Contingency Planning and Risk Mitigation}

\subsection*{Hospital Readmission Prevention}

\textbf{High-Risk Period}: 30 days post-discharge

\textbf{Prevention Strategies}:
\begin{itemize}[leftmargin=*]
    \item Early follow-up appointment (within 7 days)
    \item Medication reconciliation and adherence check
    \item Symptom monitoring escalation
    \item Care coordinator phone call within 48 hours of discharge
    \item Access to nurse advice line 24/7
\end{itemize}

\subsection*{Disease Progression Planning}

\textbf{If CKD progresses to Stage 4-5}:
\begin{itemize}[leftmargin=*]
    \item Nephrology referral for CKD education and dialysis planning
    \item Vascular access planning if eGFR $<$20
    \item Medication adjustments for reduced renal clearance
    \item Anemia management optimization (ESA if needed)
    \item Advance care planning discussions
\end{itemize}

\textbf{If HF worsens to NYHA Class III-IV}:
\begin{itemize}[leftmargin=*]
    \item Consider ICD/CRT device evaluation
    \item Advanced therapies discussion (LVAD, transplant evaluation if appropriate)
    \item Palliative care consultation for symptom management
    \item Home health nursing for weight/symptom monitoring
\end{itemize}

\subsection*{Advance Care Planning}

\begin{itemize}[leftmargin=*]
    \item Goals of care discussion: [Patient preferences documented]
    \item Healthcare proxy: [Name, relationship] designated
    \item Advance directive: ☐ Completed / ☐ To complete
    \item CPR preferences: [Discussed, documented in chart]
    \item Dialysis preferences: Patient expresses desire to avoid if possible
\end{itemize}

% ===== SECTION 9: FOLLOW-UP SCHEDULE =====
\section*{9. Follow-Up and Reassessment Schedule}

\subsection*{Appointment Calendar}

\begin{longtable}{|l|l|p{7cm}|}
\hline
\textbf{Timeframe} & \textbf{Provider} & \textbf{Purpose} \\ \hline
\endfirsthead
\hline
\textbf{Timeframe} & \textbf{Provider} & \textbf{Purpose} \\ \hline
\endhead
Week 2 & Care Coordinator (phone) & Check medication tolerability, answer questions, reinforce education \\ \hline
Month 1 & PCP & Add empagliflozin, assess insulin titration, review home monitoring logs \\ \hline
Month 2 & Dietitian & Nutrition counseling, meal planning, sodium/carb education \\ \hline
Month 3 & PCP & HbA1c check, labs (CMP, lipids), medication review, preventive care update \\ \hline
Month 3-4 & Cardiology & HF assessment, beta-blocker titration, consider ARNI \\ \hline
Month 3-4 & Endocrinology & Diabetes management review, complications screening \\ \hline
Month 6 & PCP & Comprehensive reassessment, all labs, update care plan, goal review \\ \hline
Ongoing & Quarterly PCP & Chronic disease management visits \\ \hline
\end{longtable}

\subsection*{Plan Reassessment}

This care plan will be formally reassessed and updated:
\begin{itemize}[leftmargin=*]
    \item Every 6 months (routine)
    \item After hospitalization or ED visit
    \item With significant change in clinical status
    \item When new diagnoses are added
    \item When treatment goals are achieved or modified
    \item At patient or provider request
\end{itemize}

% ===== SECTION 10: SIGNATURES =====
\vspace{2em}

\section*{10. Provider Signature and Attestation}

This comprehensive chronic disease management plan has been reviewed with the patient. The patient demonstrates understanding of all chronic conditions, treatment goals, medications, lifestyle recommendations, self-monitoring requirements, warning signs, and when to seek care. Patient's values and preferences have been incorporated through shared decision-making.

\vspace{1em}

\begin{tabular}{ll}
Provider Signature: & \rule{7cm}{0.5pt} \\[1em]
Provider Name/Credentials: & \rule{7cm}{0.5pt} \\[1em]
Date: & \rule{4cm}{0.5pt} \\[2em]
\end{tabular}

\subsection*{Care Team Acknowledgment (Optional)}

Care team members have reviewed this integrated care plan and will coordinate care accordingly.

\vspace{0.5em}

\textit{[Additional signature lines for cardiologist, endocrinologist, care coordinator as appropriate]}

\vspace{2em}
\begin{center}
\rule{\textwidth}{1pt}\\
\textbf{End of Chronic Disease Management Plan}\\
This document contains confidential patient information protected by HIPAA.
\end{center}

\end{document}

% ========== NOTES FOR USERS ==========
%
% KEY FEATURES:
% - Integrates multiple chronic conditions into unified plan
% - Addresses medication interactions and contraindications across conditions
% - Coordinates care across multiple specialistsUtilizes shared goals when conditions overlap (e.g., SGLT2i for DM + HF + CKD)
% - Emphasizes patient self-management and activation
% - Tracks quality measures and outcomes
%
% CUSTOMIZATION:
% - Adjust problem list based on patient's specific conditions
% - Modify goals for disease severity and patient capabilities
% - Adapt medication regimen to formulary and patient tolerance
% - Coordinate specialist involvement based on availability and need
%
% COMPILATION:
% pdflatex chronic_disease_management_plan.tex

