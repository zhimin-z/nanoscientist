% Perioperative Care Plan Template
% For surgical and procedural patient management
% Last updated: 2025

\documentclass[11pt,letterpaper]{article}

% Packages
\usepackage[top=1in,bottom=1in,left=1in,right=1in]{geometry}
\usepackage[utf8]{inputenc}
\usepackage{array}
\usepackage{longtable}
\usepackage{booktabs}
\usepackage{enumitem}
\usepackage{xcolor}
\usepackage{fancyhdr}
\usepackage{lastpage}
\usepackage{tabularx}
\usepackage[most]{tcolorbox}

% Header and footer
\pagestyle{fancy}
\fancyhf{}
\lhead{Perioperative Care Plan}
\rhead{Page \thepage\ of \pageref{LastPage}}
\lfoot{Date Created: \today}
\rfoot{Confidential Patient Information}

% Title formatting
\usepackage{titlesec}
\titleformat{\section}{\large\bfseries}{\thesection}{1em}{}
\titleformat{\subsection}{\normalsize\bfseries}{\thesubsection}{1em}{}

\begin{document}

% Title
\begin{center}
{\Large\bfseries PERIOPERATIVE CARE PLAN}\\[0.5em]
{\large Surgical \& Procedural Patient Management}\\[0.5em]
\rule{\textwidth}{1pt}
\end{center}

\vspace{1em}

% ===== TREATMENT PLAN HIGHLIGHTS (Foundation Medicine Model) =====
\begin{tcolorbox}[colback=red!5!white,colframe=red!75!black,title=\textbf{TREATMENT PLAN HIGHLIGHTS},fonttitle=\bfseries\large]

\textbf{Procedure:} [Planned surgery/procedure - e.g., Laparoscopic cholecystectomy for symptomatic cholelithiasis]

\vspace{0.3em}
\textbf{Primary Perioperative Goals:}
\begin{itemize}[leftmargin=*,itemsep=0pt]
    \item [Goal 1 - e.g., Safe completion of procedure with minimal complications]
    \item [Goal 2 - e.g., Discharge within 24 hours (outpatient procedure)]
    \item [Goal 3 - e.g., Return to normal activities within 2 weeks]
\end{itemize}

\vspace{0.3em}
\textbf{Key Perioperative Elements:}
\begin{itemize}[leftmargin=*,itemsep=0pt]
    \item \textit{Preoperative:} [Optimization - e.g., ASA class II, medical clearance obtained, NPO after midnight]
    \item \textit{Intraoperative:} [Approach - e.g., General anesthesia, standard laparoscopic technique]
    \item \textit{Postoperative:} [Recovery - e.g., Early mobilization, multimodal analgesia, same-day discharge]
\end{itemize}

\vspace{0.3em}
\textbf{Timeline:} [Schedule - e.g., Surgery date [XX/XX], follow-up at 2 weeks, full recovery 4-6 weeks]

\end{tcolorbox}

\vspace{1em}

% ===== SECTION 1: PATIENT AND PROCEDURE INFORMATION =====
\section*{1. Patient and Procedure Information}

\textbf{HIPAA Notice}: De-identify all protected health information before sharing.

\vspace{0.5em}

\begin{tabularx}{\textwidth}{|l|X|}
\hline
\textbf{Patient ID} & [De-identified code, e.g., SURG-001] \\ \hline
\textbf{Age Range} & [e.g., 65-70 years] \\ \hline
\textbf{Sex} & [Male/Female/Other] \\ \hline
\textbf{Date of Plan} & [Month/Year only] \\ \hline
\textbf{Surgeon} & [Name, MD, Specialty] \\ \hline
\textbf{Anesthesiologist} & [Name, MD or assigned team] \\ \hline
\textbf{Planned Procedure} & [e.g., Elective total knee arthroplasty, right] \\ \hline
\textbf{CPT Code} & [e.g., 27447] \\ \hline
\textbf{Scheduled Date} & [Month/Year or "Within 2-4 weeks"] \\ \hline
\textbf{Facility} & [Hospital/Surgery center name] \\ \hline
\textbf{Expected LOS} & [e.g., 2-3 days] \\ \hline
\end{tabularx}

\vspace{1em}

\subsection*{Surgical Indication}

\textbf{Primary Diagnosis}: [e.g., Severe osteoarthritis, right knee] (ICD-10: [M17.11])

\textbf{Indication for Surgery}:
[e.g., Patient has severe right knee pain (8/10) limiting mobility and function despite conservative management including physical therapy, weight loss, and analgesics. Radiographs demonstrate bone-on-bone contact, osteophytes, and joint space narrowing. Failed conservative treatment for 12+ months. Patient desires surgical intervention to improve quality of life and function.]

\textbf{Previous Treatments}:
\begin{itemize}[leftmargin=*]
    \item Physical therapy (6 months, minimal benefit)
    \item Weight loss (15 lbs, ongoing)
    \item NSAIDs, acetaminophen (limited efficacy)
    \item Intra-articular corticosteroid injections (3 injections, temporary relief only)
\end{itemize}

\subsection*{Medical History and Comorbidities}

\textbf{Active Medical Conditions}:
\begin{itemize}[leftmargin=*]
    \item \textbf{Hypertension}: Well-controlled on lisinopril 20mg daily
    \item \textbf{Type 2 Diabetes}: HbA1c 6.8\%, well-controlled on metformin
    \item \textbf{Hyperlipidemia}: On atorvastatin 40mg
    \item \textbf{Obesity}: BMI 32 (down from 35 with weight loss efforts)
    \item [List additional conditions]
\end{itemize}

\textbf{Current Medications}:
\begin{longtable}{|p{3cm}|p{2cm}|p{2cm}|p{6cm}|}
\hline
\textbf{Medication} & \textbf{Dose} & \textbf{Frequency} & \textbf{Perioperative Plan} \\ \hline
Lisinopril & 20mg & Daily & Hold day of surgery, resume POD 1 if BP stable \\ \hline
Metformin & 1000mg & BID & Hold 24 hours before surgery, resume when eating \\ \hline
Atorvastatin & 40mg & QHS & Continue through surgery \\ \hline
Aspirin & 81mg & Daily & Discuss with surgeon - likely continue \\ \hline
Ibuprofen & 600mg & PRN & Discontinue 5-7 days before surgery \\ \hline
[Add medications] & & & \\ \hline
\end{longtable}

\textbf{Allergies}: [NKDA or list medication allergies and reactions]

\subsection*{Preoperative Risk Assessment}

\textbf{ASA Physical Status Classification}: [e.g., ASA Class II - Mild systemic disease (HTN, DM)]

\textbf{Cardiac Risk} (Revised Cardiac Risk Index - RCRI):
\begin{itemize}[leftmargin=*]
    \item High-risk surgery: ☐ Yes ☑ No (orthopedic is intermediate-risk)
    \item Ischemic heart disease: ☐ Yes ☑ No
    \item Heart failure: ☐ Yes ☑ No
    \item Cerebrovascular disease: ☐ Yes ☑ No
    \item Diabetes on insulin: ☐ Yes ☑ No
    \item Creatinine $>$2 mg/dL: ☐ Yes ☑ No
    \item \textbf{RCRI Score}: 0 (Low risk $<$1\% cardiac event)
\end{itemize}

\textbf{Pulmonary Risk}:
\begin{itemize}[leftmargin=*]
    \item No active pulmonary disease
    \item No smoking history
    \item Room air oxygen saturation 98\%
    \item Low risk for postoperative pulmonary complications
\end{itemize}

\textbf{VTE Risk} (Caprini Score):
\begin{itemize}[leftmargin=*]
    \item Age 65-70: 2 points
    \item Major surgery ($>$45 min): 2 points
    \item BMI $>$30: 1 point
    \item \textbf{Total Score}: 5 (Moderate-high risk)
    \item \textbf{Prophylaxis Plan}: Pharmacologic (enoxaparin) + mechanical (SCDs)
\end{itemize}

\textbf{Bleeding Risk}: Low (no anticoagulation, normal coagulation studies)

% ===== SECTION 2: PREOPERATIVE OPTIMIZATION =====
\section*{2. Preoperative Optimization and Preparation}

\subsection*{2.1 Medical Optimization}

\textbf{Diabetes Management}:
\begin{itemize}[leftmargin=*]
    \item \textbf{Goal}: HbA1c $<$7-8\% for elective surgery (current 6.8\% - optimized)
    \item \textbf{Preop Day}: Hold metformin 24 hours before surgery
    \item \textbf{Morning of Surgery}: NPO, no oral hypoglycemics
    \item \textbf{Glucose Monitoring}: Check fasting glucose morning of surgery, target 100-180 mg/dL
    \item \textbf{Perioperative Protocol}: Insulin sliding scale if glucose $>$180 mg/dL
\end{itemize}

\textbf{Hypertension Management}:
\begin{itemize}[leftmargin=*]
    \item \textbf{Goal}: BP $<$140/90 preoperatively (current 128/76 - controlled)
    \item \textbf{Medication Plan}: Hold lisinopril morning of surgery (avoid intraop hypotension)
    \item \textbf{Beta-blockers}: [If on beta-blocker, continue through surgery]
    \item \textbf{Postop}: Resume home BP medications when tolerating oral intake
\end{itemize}

\textbf{Cardiac Clearance}:
\begin{itemize}[leftmargin=*]
    \item \textbf{Assessment}: Low cardiac risk (RCRI 0), intermediate-risk surgery
    \item \textbf{Functional Capacity}: $>$4 METs (can climb 1 flight of stairs)
    \item \textbf{EKG}: Normal sinus rhythm, no acute changes
    \item \textbf{Additional Testing}: Not needed (low risk, good functional capacity)
    \item \textbf{Cardiology Consultation}: Not indicated
    \item \textbf{Cleared for Surgery}: Yes
\end{itemize}

\textbf{Pulmonary Optimization}:
\begin{itemize}[leftmargin=*]
    \item \textbf{Smoking Cessation}: N/A (non-smoker)
    \item \textbf{Incentive Spirometry}: Education provided, will use postoperatively
    \item \textbf{Pulmonary Function Tests}: Not indicated (no pulmonary disease)
\end{itemize}

\textbf{Nutritional Status}:
\begin{itemize}[leftmargin=*]
    \item \textbf{Albumin}: [e.g., 4.0 g/dL - normal]
    \item \textbf{BMI}: 32 (obese, but weight loss of 15 lbs achieved)
    \item \textbf{Nutritional Optimization}: Adequate, no protein supplementation needed
\end{itemize}

\textbf{Anemia Screening and Management}:
\begin{itemize}[leftmargin=*]
    \item \textbf{Preop Hemoglobin}: [e.g., 13.2 g/dL - normal]
    \item \textbf{Iron Studies}: [If low Hgb - check iron, ferritin, TIBC]
    \item \textbf{Optimization}: No anemia present, no intervention needed
    \item \textbf{Transfusion Threshold}: Hgb $<$7-8 g/dL postoperatively (restrictive strategy)
\end{itemize}

\subsection*{2.2 Medication Management}

\textbf{Medications to Continue}:
\begin{itemize}[leftmargin=*]
    \item Statin (atorvastatin)
    \item Aspirin 81mg (after surgeon confirmation - typically continued for orthopedic)
    \item [Other chronic medications per anesthesia recommendations]
\end{itemize}

\textbf{Medications to Hold}:
\begin{itemize}[leftmargin=*]
    \item \textbf{NSAIDs}: Discontinue 5-7 days before surgery (ibuprofen)
    \item \textbf{ACE Inhibitors}: Hold day of surgery (lisinopril)
    \item \textbf{Metformin}: Hold 24 hours before, resume when eating normally
    \item \textbf{[Other medications]}: [Specific instructions]
\end{itemize}

\textbf{Anticoagulation Management}:
\begin{itemize}[leftmargin=*]
    \item Not applicable (patient not on anticoagulation)
    \item [If on warfarin: bridge with LMWH, target INR $<$1.5]
    \item [If on DOAC: hold 24-48 hours based on renal function]
\end{itemize}

\subsection*{2.3 Preoperative Testing and Clearance}

\textbf{Laboratory Tests}:
\begin{itemize}[leftmargin=*]
    \item CBC: [Results - Hgb, platelets]
    \item BMP: [Results - creatinine, glucose, electrolytes]
    \item HbA1c: 6.8\% (within 3 months)
    \item Coagulation studies (PT/INR, PTT): [If indicated]
    \item Type and screen: [Completed, blood available if needed]
\end{itemize}

\textbf{Imaging}:
\begin{itemize}[leftmargin=*]
    \item Chest X-ray: [If indicated - age $>$50 with cardiac/pulmonary disease]
    \item Preop knee X-rays: Confirm diagnosis, surgical planning
\end{itemize}

\textbf{Medical Clearance}: ☑ Cleared for surgery by PCP [Date]

\subsection*{2.4 Enhanced Recovery After Surgery (ERAS) Protocol}

\textbf{Preoperative ERAS Elements}:
\begin{itemize}[leftmargin=*]
    \item \textbf{Patient Education}: Provided ERAS booklet, reviewed expectations
    \item \textbf{Nutritional Optimization}: Carbohydrate loading (clear carb drink 2 hours before surgery)
    \item \textbf{Fasting Guidelines}: NPO solid food 6 hours, clear liquids until 2 hours before
    \item \textbf{Preoperative Bathing}: Chlorhexidine shower night before and morning of surgery
    \item \textbf{No Premedication}: Avoid long-acting sedatives (faster recovery)
\end{itemize}

% ===== SECTION 3: PERIOPERATIVE GOALS =====
\section*{3. Perioperative Goals}

\subsection*{3.1 Immediate Perioperative Goals (Day 0-1)}

\begin{enumerate}[leftmargin=*]
    \item \textbf{Pain Control}: Achieve pain $\leq$4/10 at rest, $\leq$6/10 with movement using multimodal analgesia by POD 0.
    
    \item \textbf{Early Mobilization}: Out of bed to chair within 4-6 hours post-surgery (day of surgery if morning case).
    
    \item \textbf{Nausea/Vomiting Prevention}: No or minimal PONV with multimodal antiemetic prophylaxis.
    
    \item \textbf{Glucose Control}: Maintain blood glucose 100-180 mg/dL perioperatively.
    
    \item \textbf{Hemodynamic Stability}: Maintain BP within 20\% of baseline, avoid hypo/hypertension.
\end{enumerate}

\subsection*{3.2 Early Postoperative Goals (POD 1-3)}

\begin{enumerate}[leftmargin=*]
    \item \textbf{Mobilization}: Ambulate with physical therapy 50+ feet with walker by POD 1, progress to 150 feet by POD 2.
    
    \item \textbf{ROM}: Achieve knee flexion $>$70 degrees and full extension by POD 2.
    
    \item \textbf{Pain Management}: Transition to oral multimodal analgesia, pain $\leq$5/10, minimize opioid use.
    
    \item \textbf{Diet Advancement}: Resume regular diet POD 1, adequate oral intake.
    
    \item \textbf{Bowel Function}: Return of bowel sounds, pass flatus by POD 2.
    
    \item \textbf{Urinary Function}: Foley catheter removed POD 0-1, spontaneous void within 6-8 hours.
    
    \item \textbf{Prevent Complications}: No surgical site infection, DVT, PE, or other major complications.
\end{enumerate}

\subsection*{3.3 Discharge Goals (POD 2-3)}

\begin{enumerate}[leftmargin=*]
    \item \textbf{Functional Mobility}: Independent transfers, ambulate 150+ feet with assistive device, negotiate stairs if needed for home.
    
    \item \textbf{Pain Control}: Adequate pain control on oral medications, pain $<$5/10.
    
    \item \textbf{Safety}: Patient/family demonstrate understanding of precautions, medications, wound care.
    
    \item \textbf{Discharge Readiness}: Stable vital signs, no complications, safe for discharge home (with home health if needed).
\end{enumerate}

% ===== SECTION 4: INTRAOPERATIVE MANAGEMENT =====
\section*{4. Intraoperative Management Plan}

\subsection*{Anesthesia Plan}

\textbf{Anesthesia Type}: [e.g., Spinal anesthesia + sedation] (surgeon/anesthesia preference)

\textbf{Alternatives Discussed}:
\begin{itemize}[leftmargin=*]
    \item General anesthesia
    \item Regional anesthesia (spinal/epidural)
    \item Peripheral nerve block (femoral, adductor canal block)
\end{itemize}

\textbf{Multimodal Analgesia - Intraoperative}:
\begin{itemize}[leftmargin=*]
    \item Regional anesthesia (spinal/block) as primary analgesic
    \item IV acetaminophen 1g intraoperatively
    \item Ketorolac 15-30mg IV (if no contraindication)
    \item Local anesthetic infiltration at surgical site (surgeon)
    \item Minimize intraop opioids (opioid-sparing approach)
\end{itemize}

\textbf{PONV Prophylaxis}:
\begin{itemize}[leftmargin=*]
    \item Ondansetron 4mg IV
    \item Dexamethasone 4-8mg IV
    \item Scopolamine patch (if high PONV risk)
    \item Avoid volatile anesthetics if possible (TIVA preferred)
\end{itemize}

\subsection*{Surgical Approach}

\textbf{Procedure}: Total knee arthroplasty, cemented components

\textbf{Antibiotic Prophylaxis}:
\begin{itemize}[leftmargin=*]
    \item Cefazolin 2g IV within 60 minutes of incision (3g if weight $>$120 kg)
    \item Redose if surgery $>$4 hours or blood loss $>$1500 mL
    \item Discontinue within 24 hours post-surgery
\end{itemize}

\textbf{VTE Prophylaxis - Intraoperative}:
\begin{itemize}[leftmargin=*]
    \item Sequential compression devices (SCDs) applied before induction
    \item Continue SCDs throughout hospitalization and at rest at home
\end{itemize}

\textbf{Surgical Site Infection Prevention}:
\begin{itemize}[leftmargin=*]
    \item Chlorhexidine-alcohol skin prep
    \item Maintain normothermia (goal temp $>$36°C)
    \item Glucose control (intraop glucose $<$180 mg/dL)
    \item Surgical time minimize (planned $<$2 hours)
\end{itemize}

\textbf{Blood Management}:
\begin{itemize}[leftmargin=*]
    \item Tranexamic acid 1-2g IV (reduce blood loss)
    \item Cell saver if appropriate
    \item Restrictive transfusion strategy (Hgb $<$7-8 g/dL)
\end{itemize}

% ===== SECTION 5: POSTOPERATIVE MANAGEMENT =====
\section*{5. Postoperative Management Plan}

\subsection*{5.1 Pain Management (Multimodal Analgesia)}

\textbf{ERAS Pain Protocol} (opioid-minimizing):

\begin{longtable}{|p{3.5cm}|p{2.5cm}|p{7cm}|}
\hline
\textbf{Medication} & \textbf{Dose/Frequency} & \textbf{Instructions} \\ \hline
\textbf{Acetaminophen} & 1000mg Q6H & Scheduled (not PRN), around-the-clock for 48 hours \\ \hline
\textbf{Celecoxib} or \textbf{Meloxicam} & 200mg BID or 15mg daily & NSAID (if no contraindication), scheduled x 7-14 days \\ \hline
\textbf{Gabapentin} & 300mg TID & Neuropathic pain adjuvant, start preop or POD 0 \\ \hline
\textbf{Ice therapy} & Q2H while awake & Local cooling, reduces swelling and pain \\ \hline
\textbf{Oxycodone} & 5mg Q4H PRN & Breakthrough pain only, goal minimize use \\ \hline
\end{longtable}

\textbf{Pain Assessment}: Numeric rating scale (0-10) every 4 hours, before and after ambulation

\textbf{Pain Goals}: $\leq$4/10 at rest, $\leq$6/10 with PT/activity

\subsection*{5.2 Early Mobilization and Physical Therapy}

\textbf{ERAS Mobility Protocol}:

\begin{itemize}[leftmargin=*]
    \item \textbf{POD 0 (Day of Surgery)}: Out of bed to chair 4-6 hours post-op, stand at bedside
    \item \textbf{POD 1}:
    \begin{itemize}
        \item PT evaluation and gait training
        \item Ambulate 50+ feet with walker x2
        \item Begin ROM exercises (CPM machine or therapist-assisted)
        \item Stair practice if needed for home
    \end{itemize}
    \item \textbf{POD 2}:
    \begin{itemize}
        \item Ambulate 150+ feet with walker x2-3
        \item ROM: Goal flexion $>$90 degrees
        \item Independent bed mobility and transfers
        \item Stairs if required
    \end{itemize}
    \item \textbf{Discharge Criteria}: Ambulate 150 feet, transfers independently, stairs if applicable
\end{itemize}

\textbf{Fall Precautions}: High risk post-surgery - bed alarm, non-slip socks, walker, call for assist

\subsection*{5.3 Nausea and Vomiting Management}

\textbf{Multimodal Antiemetic Protocol}:
\begin{itemize}[leftmargin=*]
    \item Ondansetron 4mg IV/PO Q6H PRN
    \item Metoclopramide 10mg IV Q6H PRN (if ondansetron insufficient)
    \item Scopolamine patch (continue 72 hours if applied)
    \item Non-pharmacologic: Ginger ale, acupressure bands, avoid rapid position changes
\end{itemize}

\subsection*{5.4 Nutrition and Diet Advancement}

\textbf{ERAS Nutrition}:
\begin{itemize}[leftmargin=*]
    \item Resume diet as tolerated POD 0-1 (no prolonged NPO)
    \item Protein-rich diet (wound healing)
    \item Adequate hydration
    \item No routine NG tube
\end{itemize}

\subsection*{5.5 VTE Prophylaxis}

\textbf{Pharmacologic} (High-risk orthopedic surgery):
\begin{itemize}[leftmargin=*]
    \item \textbf{Enoxaparin 40mg SC daily} starting POD 1, continue 10-14 days
    \item \textit{Alternative}: Apixaban 2.5mg BID x 12 days (extended prophylaxis)
    \item Hold first dose if neuraxial anesthesia (spinal/epidural) until catheter removal + 12 hours
\end{itemize}

\textbf{Mechanical}:
\begin{itemize}[leftmargin=*]
    \item SCDs while in bed throughout hospitalization
    \item Early mobilization (most important)
\end{itemize}

\textbf{Duration}: Minimum 10-14 days, consider up to 35 days for high-risk patients

\subsection*{5.6 Urinary Catheter Management}

\begin{itemize}[leftmargin=*]
    \item \textbf{Foley Catheter}: Typically placed intraoperatively
    \item \textbf{Removal}: POD 0 or POD 1 morning (early removal to prevent CAUTI)
    \item \textbf{Voiding Trial}: Must void within 6-8 hours of catheter removal
    \item \textbf{Retention Protocol}: If unable to void or bladder scan $>$400 mL, straight cath or replace Foley temporarily
\end{itemize}

\subsection*{5.7 Wound Care and Drain Management}

\textbf{Surgical Drain}:
\begin{itemize}[leftmargin=*]
    \item Hemovac or JP drain typically placed
    \item Monitor output, remove when $<$30 mL/8 hours (usually POD 1-2)
\end{itemize}

\textbf{Dressing}:
\begin{itemize}[leftmargin=*]
    \item Keep clean and dry
    \item First dressing change POD 2 or per surgeon
    \item Assess for signs of infection daily
\end{itemize}

\subsection*{5.8 Glycemic Control}

\textbf{Postoperative Glucose Management}:
\begin{itemize}[leftmargin=*]
    \item Target glucose 100-180 mg/dL
    \item Check glucose Q6H while NPO or on IV fluids
    \item Insulin sliding scale (SSI) if glucose $>$180 mg/dL
    \item Resume metformin when tolerating regular diet and creatinine stable
\end{itemize}

\subsection*{5.9 Complication Surveillance}

\textbf{Monitor for}:
\begin{itemize}[leftmargin=*]
    \item \textbf{Surgical site infection}: Fever, wound erythema, purulent drainage, increased pain
    \item \textbf{DVT/PE}: Unilateral leg swelling, chest pain, dyspnea, hypoxia
    \item \textbf{Acute kidney injury}: Decreased UOP, rising creatinine
    \item \textbf{Cardiovascular events}: Chest pain, EKG changes, troponin elevation
    \item \textbf{Delirium}: Especially in elderly, multimodal prevention
\end{itemize}

% ===== SECTION 6: DISCHARGE PLANNING =====
\section*{6. Discharge Planning and Criteria}

\subsection*{Discharge Criteria (Typically POD 2-3)}

Patient ready for discharge when ALL met:
\begin{itemize}[leftmargin=*]
    \item ☐ Adequate pain control on oral medications (pain $<$5/10)
    \item ☐ Functional mobility: Ambulate 150+ feet, transfers, stairs if needed
    \item ☐ Tolerating regular diet, adequate oral intake
    \item ☐ Voiding spontaneously without catheter
    \item ☐ Stable vital signs, no fever $>$38.5°C x 24 hours
    \item ☐ No complications requiring continued hospitalization
    \item ☐ Adequate home support and DME arranged
    \item ☐ Patient/family education completed, demonstrate understanding
\end{itemize}

\subsection*{Discharge Medications}

\begin{longtable}{|p{3cm}|p{2cm}|p{2cm}|p{6cm}|}
\hline
\textbf{Medication} & \textbf{Dose} & \textbf{Frequency} & \textbf{Duration/Instructions} \\ \hline
Oxycodone & 5mg & Q4-6H PRN & Pain, 20 tablets (minimize use) \\ \hline
Acetaminophen & 1000mg & Q6H & Scheduled x 2 weeks \\ \hline
Meloxicam & 15mg & Daily & x 2 weeks (NSAID) \\ \hline
Enoxaparin & 40mg SC & Daily & x 10-14 days (VTE prophylaxis) \\ \hline
Colace & 100mg & BID & Constipation prevention while on opioids \\ \hline
[Resume home meds] & & & Resume lisinopril, metformin, atorvastatin \\ \hline
\end{longtable}

\subsection*{Durable Medical Equipment (DME)}

\begin{itemize}[leftmargin=*]
    \item Walker (front-wheeled, standard adult)
    \item Raised toilet seat with arms
    \item Shower chair or bath bench
    \item Reacher (32-inch)
    \item Ice machine or ice packs (for knee)
    \item Long-handled shoe horn (hip precautions if applicable)
\end{itemize}

\subsection*{Home Services}

\begin{itemize}[leftmargin=*]
    \item \textbf{Home Health Physical Therapy}: 2-3x/week x 2-3 weeks, then transition to outpatient PT
    \item \textbf{Home Health Nursing}: PRN for wound check, drain removal if not removed before discharge, medication teaching (enoxaparin injections)
    \item [If high needs: Home health aide for ADL assistance]
\end{itemize}

\subsection*{Patient Education Completed}

\begin{itemize}[leftmargin=*]
    \item ✓ Wound care and dressing changes
    \item ✓ Signs of infection (fever, redness, drainage, increased pain)
    \item ✓ Pain medication use and weaning plan
    \item ✓ Enoxaparin self-injection technique (or family member trained)
    \item ✓ DVT/PE warning signs (leg swelling, chest pain, shortness of breath)
    \item ✓ Activity restrictions and precautions
    \item ✓ Home exercise program
    \item ✓ Use of DME (walker, raised toilet seat, etc.)
    \item ✓ When to call surgeon (fever $>$101.5°F, severe pain, wound concerns)
    \item ✓ Follow-up appointments scheduled
\end{itemize}

\subsection*{Activity Restrictions}

\begin{itemize}[leftmargin=*]
    \item Use walker for ambulation x 2-4 weeks (per PT recommendation)
    \item No driving until off opioid pain medications and cleared by surgeon (typically 2-4 weeks)
    \item No prolonged sitting $>$30-45 min without getting up and moving
    \item Avoid kneeling on operative knee
    \item Gradual return to activities as tolerated
\end{itemize}

\subsection*{Follow-Up Appointments}

\begin{tabularx}{\textwidth}{|l|l|X|}
\hline
\textbf{Provider} & \textbf{Timing} & \textbf{Purpose} \\ \hline
Surgeon & 10-14 days & Wound check, staple/suture removal, assess progress \\ \hline
Surgeon & 6 weeks & X-ray, functional assessment, advance activities \\ \hline
Surgeon & 3 months, 6 months, 1 year & Long-term follow-up, outcomes \\ \hline
PCP & 1-2 weeks & Resume chronic disease management, BP/DM check \\ \hline
PT (outpatient) & After home health complete & Continue strengthening, ROM, return to function \\ \hline
\end{tabularx}

% ===== SECTION 7: EMERGENCY PROCEDURES =====
\section*{7. Postoperative Emergency Procedures}

\textbf{Call surgeon immediately or go to ED if}:
\begin{itemize}[leftmargin=*]
    \item Fever $>$101.5°F (38.6°C)
    \item Severe uncontrolled pain ($>$7/10 despite medications)
    \item Wound: Excessive drainage, purulent discharge, wound dehiscence, foul odor
    \item Increased redness, warmth, or swelling at surgical site
    \item DVT symptoms: Unilateral leg swelling, pain, warmth, redness
    \item PE symptoms: Sudden chest pain, shortness of breath, rapid heart rate
    \item Numbness, tingling, or weakness in leg (nerve injury concern)
    \item Inability to urinate
    \item Excessive bleeding from surgical site
\end{itemize}

\textbf{Call 911 for}:
\begin{itemize}[leftmargin=*]
    \item Chest pain or pressure
    \item Severe shortness of breath
    \item Loss of consciousness
    \item Signs of stroke (facial droop, arm weakness, speech difficulty)
\end{itemize}

\textbf{Surgeon Contact Information}:
\begin{itemize}[leftmargin=*]
    \item Office: [Phone number]
    \item After-hours/Emergency: [On-call service number]
\end{itemize}

% ===== SECTION 8: REHABILITATION AND RECOVERY =====
\section*{8. Rehabilitation Plan and Expected Recovery}

\subsection*{Recovery Timeline}

\begin{tabularx}{\textwidth}{|l|X|}
\hline
\textbf{Timeframe} & \textbf{Expected Progress} \\ \hline
Week 1-2 & Wound healing, pain decreasing, ambulation with walker improving, ROM exercises \\ \hline
Week 3-6 & Transition from walker to cane, ROM improving (goal flexion $>$100°), less pain \\ \hline
Week 6-12 & Progress to independent ambulation (no assistive device), ROM 110-120° flexion, strengthening phase \\ \hline
3-6 months & Return to most activities, continued strengthening, ROM optimization, minimal pain \\ \hline
6-12 months & Full recovery, return to all desired activities, final ROM achieved \\ \hline
\end{tabularx}

\subsection*{Physical Therapy Goals}

\textbf{Short-term} (0-6 weeks):
\begin{itemize}[leftmargin=*]
    \item ROM: Flexion $>$90° by week 2, $>$110° by week 6, full extension
    \item Strength: Quadriceps, hamstrings, hip abductors
    \item Ambulation: Progress from walker to cane to independent
    \item Stairs: Negotiate safely
\end{itemize}

\textbf{Long-term} (6 weeks - 3 months):
\begin{itemize}[leftmargin=*]
    \item ROM: Maximum flexion (goal 120-125°)
    \item Strength: Near-normal lower extremity strength
    \item Function: Return to ADLs, hobbies, light sports
    \item Gait: Normal gait pattern without assistive device
\end{itemize}

\subsection*{Home Exercise Program}

\textit{Provided by PT, to be performed 2-3x daily}:
\begin{itemize}[leftmargin=*]
    \item Ankle pumps
    \item Quad sets
    \item Straight leg raises
    \item Hamstring curls
    \item Hip abduction
    \item Knee flexion/extension ROM exercises
    \item Heel slides
    \item Stationary bike (when cleared)
\end{itemize}

% ===== SECTION 9: INFORMED CONSENT =====
\section*{9. Informed Consent Documentation}

\textbf{Risks and Benefits Discussed}:

\textbf{Benefits}:
\begin{itemize}[leftmargin=*]
    \item Pain relief (90\% significant improvement)
    \item Improved function and mobility
    \item Enhanced quality of life
    \item Return to desired activities
\end{itemize}

\textbf{Risks}:
\begin{itemize}[leftmargin=*]
    \item Infection ($<$2\%)
    \item DVT/PE (2-3\% despite prophylaxis)
    \item Bleeding, hematoma
    \item Nerve or blood vessel injury (rare)
    \item Stiffness, limited ROM
    \item Implant loosening, wear (long-term)
    \item Need for revision surgery (10-15\% lifetime risk)
    \item Anesthesia risks
\end{itemize}

\textbf{Alternatives Discussed}:
\begin{itemize}[leftmargin=*]
    \item Continued conservative management (PT, medications, injections)
    \item Partial knee replacement (if eligible)
    \item No treatment
\end{itemize}

Patient demonstrates understanding, all questions answered, consents to proceed with surgery.

% ===== SECTION 10: SIGNATURES =====
\vspace{2em}

\section*{10. Provider Signatures}

\textbf{Surgeon}:\\[0.5em]
Signature: \rule{6cm}{0.5pt} \quad Date: \rule{3cm}{0.5pt}\\
Name/Credentials: \rule{6cm}{0.5pt}\\[1em]

\textbf{Anesthesiologist}:\\[0.5em]
Signature: \rule{6cm}{0.5pt} \quad Date: \rule{3cm}{0.5pt}\\
Name/Credentials: \rule{6cm}{0.5pt}\\[1em]

\textbf{Patient Consent}:\\[0.5em]
I have reviewed this perioperative care plan. I understand the procedure, risks, benefits, and alternatives. My questions have been answered. I consent to the planned surgery.\\[0.5em]
Signature: \rule{6cm}{0.5pt} \quad Date: \rule{3cm}{0.5pt}\\

\vspace{2em}
\begin{center}
\rule{\textwidth}{1pt}\\
\textbf{End of Perioperative Care Plan}\\
This document contains confidential patient information protected by HIPAA.
\end{center}

\end{document}

% ========== NOTES FOR USERS ==========
%
% This template emphasizes Enhanced Recovery After Surgery (ERAS) principles
% Key ERAS elements: preop carbohydrate loading, minimal fasting, multimodal analgesia,
% early mobilization, early feeding, minimizing tubes/drains, VTE prophylaxis
%
% CUSTOMIZATION:
% - Adjust for specific surgical procedure
% - Modify based on patient comorbidities
% - Update medication protocols per institutional guidelines
% - Adapt ERAS elements based on evidence and surgeon preference
%
% COMPILATION:
% pdflatex perioperative_care_plan.tex

