% Rehabilitation Treatment Plan Template
% For physical therapy, occupational therapy, and rehabilitation services
% Last updated: 2025

\documentclass[11pt,letterpaper]{article}

% Packages
\usepackage[top=1in,bottom=1in,left=1in,right=1in]{geometry}
\usepackage{amsmath,amssymb}
\usepackage[utf8]{inputenc}
\usepackage{graphicx}
\usepackage{array}
\usepackage{longtable}
\usepackage{booktabs}
\usepackage{enumitem}
\usepackage{xcolor}
\usepackage{fancyhdr}
\usepackage{lastpage}
\usepackage{tabularx}
\usepackage{multirow}
\usepackage[most]{tcolorbox}

% Header and footer
\pagestyle{fancy}
\fancyhf{}
\lhead{Rehabilitation Treatment Plan}
\rhead{Page \thepage\ of \pageref{LastPage}}
\lfoot{Date Created: \today}
\rfoot{Confidential Patient Information}

% Title formatting
\usepackage{titlesec}
\titleformat{\section}{\large\bfseries}{\thesection}{1em}{}
\titleformat{\subsection}{\normalsize\bfseries}{\thesubsection}{1em}{}

\begin{document}

% Title
\begin{center}
{\Large\bfseries REHABILITATION TREATMENT PLAN}\\[0.5em]
{\large Physical Therapy | Occupational Therapy | Speech-Language Pathology}\\[0.5em]
\rule{\textwidth}{1pt}
\end{center}

\vspace{1em}

% ===== TREATMENT PLAN HIGHLIGHTS (Foundation Medicine Model) =====
\begin{tcolorbox}[colback=green!5!white,colframe=green!75!black,title=\textbf{TREATMENT PLAN HIGHLIGHTS},fonttitle=\bfseries\large]

\textbf{Key Diagnosis:} [Primary condition requiring rehabilitation - e.g., Post-stroke hemiparesis, Total knee replacement]

\vspace{0.3em}
\textbf{Primary Functional Goals:}
\begin{itemize}[leftmargin=*,itemsep=0pt]
    \item [Goal 1 - e.g., Independent ambulation with assistive device within 8 weeks]
    \item [Goal 2 - e.g., Return to independent ADLs (bathing, dressing) within 12 weeks]
    \item [Goal 3 - e.g., Improve upper extremity strength to 4/5 for functional tasks]
\end{itemize}

\vspace{0.3em}
\textbf{Main Interventions:}
\begin{itemize}[leftmargin=*,itemsep=0pt]
    \item \textit{Physical Therapy:} [Focus - e.g., Gait training, strengthening, balance exercises 3x/week]
    \item \textit{Occupational Therapy:} [Focus - e.g., ADL training, adaptive equipment, 2x/week]
    \item \textit{Home Exercise Program:} [Key exercises - e.g., Daily strengthening and ROM exercises]
\end{itemize}

\vspace{0.3em}
\textbf{Timeline:} [Duration - e.g., Acute phase (4 weeks), Active rehab (8 weeks), Maintenance (ongoing)]

\end{tcolorbox}

\vspace{1em}

% ===== SECTION 1: PATIENT INFORMATION =====
\section*{1. Patient Information}

\textbf{HIPAA Notice}: De-identify per Safe Harbor method. Remove all 18 HIPAA identifiers before sharing.

\vspace{0.5em}

\begin{tabularx}{\textwidth}{|l|X|}
\hline
\textbf{Patient ID} & [De-identified code, e.g., PT-RH-001] \\ \hline
\textbf{Age Range} & [e.g., 65-70 years] \\ \hline
\textbf{Sex} & [Male/Female/Other] \\ \hline
\textbf{Date of Plan} & [Month/Year only] \\ \hline
\textbf{Referring Provider} & [Name, Credentials] \\ \hline
\textbf{Primary Therapist} & [PT/OT/SLP Name, Credentials] \\ \hline
\textbf{Facility} & [Rehabilitation center/Clinic name] \\ \hline
\end{tabularx}

\vspace{1em}

\subsection*{Diagnosis and Medical History}

\begin{itemize}[leftmargin=*]
    \item \textbf{Primary Diagnosis}: [e.g., Right hip fracture status post ORIF] (ICD-10: [code])
    \item \textbf{Secondary Diagnoses}:
    \begin{itemize}
        \item [e.g., Osteoporosis] (ICD-10: [code])
        \item [e.g., Hypertension] (ICD-10: [code])
        \item [Additional relevant conditions]
    \end{itemize}
    \item \textbf{Date of Injury/Surgery}: [Month/Year]
    \item \textbf{Surgical Procedure}: [e.g., Open reduction internal fixation right hip]
    \item \textbf{Precautions/Restrictions}: [e.g., Weight-bearing as tolerated, hip flexion $<$90 degrees]
\end{itemize}

\subsection*{Current Medications}

Medications affecting rehabilitation:
\begin{itemize}[leftmargin=*]
    \item \textbf{Pain Management}: [e.g., Oxycodone 5mg Q6H PRN - may affect alertness]
    \item \textbf{Anticoagulation}: [e.g., Enoxaparin 40mg daily - fall precautions]
    \item \textbf{Other Relevant Medications}: [e.g., Beta-blocker - monitor HR during exercise]
\end{itemize}

\subsection*{Living Situation and Support}

\begin{itemize}[leftmargin=*]
    \item \textbf{Living Environment}: [e.g., Two-story home, bedroom upstairs, 4 steps to entry]
    \item \textbf{Social Support}: [e.g., Lives with spouse, adult children nearby]
    \item \textbf{Prior Functional Level}: [e.g., Independent in all ADLs, community ambulation]
    \item \textbf{Occupation/Activities}: [e.g., Retired teacher, enjoys gardening and walking]
\end{itemize}

% ===== SECTION 2: FUNCTIONAL ASSESSMENT =====
\section*{2. Initial Functional Assessment}

\subsection*{2.1 Functional Independence Measure (FIM) or Similar}

\textbf{Date of Assessment}: [Date]

\begin{tabularx}{\textwidth}{|l|c|c|X|}
\hline
\textbf{Domain} & \textbf{Score} & \textbf{Goal} & \textbf{Notes} \\ \hline
Self-Care & [e.g., 28/42] & [35/42] & Requires assist with lower body dressing, bathing \\ \hline
Sphincter Control & [42/42] & [42/42] & Independent \\ \hline
Transfers & [e.g., 12/21] & [18/21] & Moderate assist bed/chair, toilet \\ \hline
Locomotion & [e.g., 8/14] & [12/14] & Contact guard ambulation 50ft with walker \\ \hline
Communication & [14/14] & [14/14] & Independent \\ \hline
Social Cognition & [21/21] & [21/21] & Independent \\ \hline
\textbf{TOTAL FIM} & \textbf{[125/126]} & \textbf{[142/126]} & \\ \hline
\end{tabularx}

\vspace{0.5em}
\textit{FIM Scoring: 7=Complete Independence, 6=Modified Independence, 5=Supervision, 4=Minimal Assist, 3=Moderate Assist, 2=Maximal Assist, 1=Total Assist}

\subsection*{2.2 Physical Therapy Assessment}

\textbf{Range of Motion}:
\begin{longtable}{|l|c|c|c|}
\hline
\textbf{Joint/Motion} & \textbf{Baseline} & \textbf{Goal} & \textbf{Normal Range} \\ \hline
\endfirsthead
\hline
\textbf{Joint/Motion} & \textbf{Baseline} & \textbf{Goal} & \textbf{Normal Range} \\ \hline
\endhead
Right hip flexion & 70° (pain at end range) & 110° pain-free & 120° \\ \hline
Right hip extension & 5° & 15° & 20° \\ \hline
Right hip abduction & 20° & 35° & 45° \\ \hline
Right knee flexion & 100° & 125° & 130° \\ \hline
Right ankle DF/PF & 5°/35° & 10°/40° & 15°/50° \\ \hline
[Additional joints] & & & \\ \hline
\end{longtable}

\textbf{Muscle Strength (Manual Muscle Testing - MMT)}:
\begin{longtable}{|l|c|c|}
\hline
\textbf{Muscle Group} & \textbf{Baseline} & \textbf{Goal} \\ \hline
\endfirsthead
\hline
\textbf{Muscle Group} & \textbf{Baseline} & \textbf{Goal} \\ \hline
\endhead
Right hip flexors & 3/5 (fair) & 4+/5 (good+) \\ \hline
Right hip extensors & 3/5 (fair) & 4+/5 (good+) \\ \hline
Right hip abductors & 2+/5 (poor+) & 4/5 (good) \\ \hline
Right quadriceps & 4-/5 (good-) & 5/5 (normal) \\ \hline
Right ankle DF/PF & 4/5 / 4/5 & 5/5 / 5/5 \\ \hline
Core stability & Fair & Good \\ \hline
[Additional muscles] & & \\ \hline
\end{longtable}

\textit{MMT Scale: 5=Normal, 4=Good, 3=Fair, 2=Poor, 1=Trace, 0=Zero}

\textbf{Balance Assessment}:
\begin{itemize}[leftmargin=*]
    \item \textbf{Berg Balance Scale}: [e.g., 38/56 - Moderate fall risk]
    \item \textbf{Goal Berg Score}: [e.g., $>$45/56 - Low fall risk]
    \item \textbf{Static Standing Balance}: [e.g., Able to stand 30 sec with walker, not independent]
    \item \textbf{Dynamic Balance}: [e.g., Unable to step over obstacles safely]
    \item \textbf{Single Leg Stance}: [e.g., Unable, requires support]
\end{itemize}

\textbf{Gait Assessment}:
\begin{itemize}[leftmargin=*]
    \item \textbf{Assistive Device}: [e.g., Front-wheeled walker]
    \item \textbf{Weight-Bearing Status}: [e.g., WBAT (weight-bearing as tolerated)]
    \item \textbf{Gait Distance}: [e.g., 50 feet with contact guard, requires 1 rest break]
    \item \textbf{Gait Speed}: [e.g., 0.4 m/s (severely impaired, normal $>$1.0 m/s)]
    \item \textbf{Gait Deviations}: [e.g., Shortened stance phase right, Trendelenburg gait, decreased step length]
    \item \textbf{Stairs}: [e.g., Unable to attempt, 4 steps required for home access]
\end{itemize}

\textbf{Endurance}:
\begin{itemize}[leftmargin=*]
    \item \textbf{6-Minute Walk Test}: [e.g., 150 feet - severely impaired]
    \item \textbf{Goal Distance}: [e.g., 300+ feet]
    \item \textbf{Perceived Exertion}: [e.g., 5/10 after 50 feet]
    \item \textbf{Vital Signs Response}: [e.g., HR increases 85→105, appropriate response]
\end{itemize}

\textbf{Pain Assessment}:
\begin{itemize}[leftmargin=*]
    \item \textbf{Pain Location}: [e.g., Right hip, groin region]
    \item \textbf{Pain at Rest}: [e.g., 2/10]
    \item \textbf{Pain with Activity}: [e.g., 6/10 with weight-bearing, 4/10 with ROM]
    \item \textbf{Pain Impact}: [e.g., Limits therapy participation, improves with rest]
\end{itemize}

\subsection*{2.3 Occupational Therapy Assessment}

\textbf{Activities of Daily Living (ADLs)}:
\begin{longtable}{|l|c|X|}
\hline
\textbf{Activity} & \textbf{Level} & \textbf{Description} \\ \hline
\endfirsthead
\hline
\textbf{Activity} & \textbf{Level} & \textbf{Description} \\ \hline
\endhead
Bathing & Mod A & Requires assist entering/exiting shower, reaching lower extremities \\ \hline
Dressing - Upper Body & I & Independent \\ \hline
Dressing - Lower Body & Mod A & Requires assist donning socks, shoes, pants due to hip precautions \\ \hline
Toileting & Min A & Requires assist with clothing management \\ \hline
Grooming & I & Independent \\ \hline
Feeding & I & Independent \\ \hline
Functional Mobility & CG & Contact guard for bed mobility, transfers \\ \hline
\end{longtable}

\textit{I=Independent, SV=Supervision, CG=Contact Guard, Min A=Minimal Assist, Mod A=Moderate Assist, Max A=Maximal Assist}

\textbf{Instrumental Activities of Daily Living (IADLs)}:
\begin{itemize}[leftmargin=*]
    \item \textbf{Meal Preparation}: Not assessed, not safe for standing tasks currently
    \item \textbf{Housekeeping}: Dependent, unable to perform
    \item \textbf{Laundry}: Dependent
    \item \textbf{Shopping}: Dependent
    \item \textbf{Home Management}: Requires complete assistance
\end{itemize}

\textbf{Upper Extremity Function}:
\begin{itemize}[leftmargin=*]
    \item \textbf{Grip Strength}: Right [kg], Left [kg] (compared to normative data)
    \item \textbf{Coordination}: [e.g., Within normal limits bilaterally]
    \item \textbf{Sensation}: [e.g., Intact to light touch, proprioception]
\end{itemize}

\subsection*{2.4 Cognitive and Perceptual Assessment}

\begin{itemize}[leftmargin=*]
    \item \textbf{Alertness/Orientation}: [e.g., Alert, oriented x3]
    \item \textbf{Memory}: [e.g., Intact for short and long-term]
    \item \textbf{Safety Awareness}: [e.g., Good insight into limitations, follows precautions]
    \item \textbf{Executive Function}: [e.g., Able to problem-solve, sequence tasks appropriately]
    \item \textbf{Visual-Perceptual Skills}: [e.g., Within normal limits]
\end{itemize}

\subsection*{2.5 Environmental Assessment}

\textbf{Home Safety Concerns}:
\begin{itemize}[leftmargin=*]
    \item 4 steps to enter home - needs stair training
    \item Bedroom/bathroom upstairs - may need temporary bedroom on main floor
    \item Shower stall (no tub) - needs shower chair, grab bars
    \item Scatter rugs - fall hazard, recommend removal
    \item Adequate lighting - satisfactory
\end{itemize}

% ===== SECTION 3: REHABILITATION GOALS =====
\section*{3. Rehabilitation Goals (SMART Format)}

\subsection*{3.1 Short-Term Goals (2-4 weeks)}

\textbf{Impairment-Level Goals}:
\begin{enumerate}[leftmargin=*]
    \item \textbf{Range of Motion}: Increase right hip flexion from 70° to 90° pain-free within 2 weeks to improve functional mobility.
    
    \item \textbf{Strength}: Improve right hip abductor strength from 2+/5 to 3+/5 within 3 weeks to reduce Trendelenburg gait.
    
    \item \textbf{Balance}: Increase Berg Balance Scale from 38/56 to 42/56 within 4 weeks to reduce fall risk.
\end{enumerate}

\textbf{Activity-Level Goals}:
\begin{enumerate}[leftmargin=*]
    \item \textbf{Ambulation}: Ambulate 150 feet with front-wheeled walker, supervision level, within 3 weeks.
    
    \item \textbf{Transfers}: Perform bed-to-chair and toilet transfers with supervision (no physical assist) within 2 weeks.
    
    \item \textbf{Stairs}: Ascend/descend 4 stairs with handrail and supervision within 4 weeks for home access.
    
    \item \textbf{Lower Body Dressing}: Don socks and shoes with adaptive equipment (reacher, sock aid) with minimal assist within 3 weeks.
    
    \item \textbf{Bathing}: Shower independently using shower chair and grab bars with setup assistance within 4 weeks.
\end{enumerate}

\subsection*{3.2 Long-Term Goals (6-12 weeks)}

\textbf{Participation-Level Goals}:
\begin{enumerate}[leftmargin=*]
    \item \textbf{Community Ambulation}: Walk independently 300+ feet with assistive device on varied terrain within 8 weeks to enable community outings.
    
    \item \textbf{ADL Independence}: Achieve independence in all basic ADLs (bathing, dressing, toileting, transfers) within 8 weeks for safe home discharge.
    
    \item \textbf{Home Management}: Return to light homemaking tasks (meal prep, laundry) with modified techniques within 12 weeks.
    
    \item \textbf{Recreational Activities}: Resume gardening with adaptive techniques and equipment within 12 weeks.
    
    \item \textbf{Fall Prevention}: Demonstrate safety awareness and fall prevention strategies for independent home functioning within 8 weeks.
\end{enumerate}

\textbf{Discharge Goals}:
\begin{itemize}[leftmargin=*]
    \item Safe discharge home with appropriate DME (durable medical equipment)
    \item Independent or supervision level for all ADLs
    \item Community ambulation with assistive device
    \item Patient and family educated on home exercise program
    \item Fall risk minimized with environmental modifications
\end{itemize}

\subsection*{3.3 Patient-Centered Goals}

Patient's top priorities:
\begin{enumerate}[leftmargin=*]
    \item "I want to go home and not need help from my family"
    \item "I want to be able to go to the grocery store again"
    \item "I want to get back to my garden this spring"
\end{enumerate}

% ===== SECTION 4: TREATMENT INTERVENTIONS =====
\section*{4. Treatment Interventions}

\subsection*{4.1 Physical Therapy Interventions}

\textbf{Frequency}: 3 sessions per week, 45-60 minutes per session, for 8-12 weeks

\textbf{Therapeutic Exercise}:
\begin{itemize}[leftmargin=*]
    \item \textbf{Strengthening}:
    \begin{itemize}
        \item Hip abduction in sidelying with resistance band: 3 sets x 10 reps
        \item Hip extension prone: 3 sets x 10 reps
        \item Quadriceps sets and straight leg raises: 3 sets x 10 reps
        \item Standing hip abduction at parallel bars: 2 sets x 10 reps
        \item Step-ups (2-inch platform progressing to 6-inch): 2 sets x 10 reps
        \item Squats (partial, with walker for support): 2 sets x 10 reps
    \end{itemize}
    
    \item \textbf{Range of Motion}:
    \begin{itemize}
        \item Active-assisted hip flexion supine: 3 sets x 10 reps
        \item Hip flexor stretching (modified, respecting precautions): 3 x 30 sec holds
        \item Ankle pumps and circles: 3 sets x 10 reps
    \end{itemize}
    
    \item \textbf{Core Stabilization}:
    \begin{itemize}
        \item Abdominal bracing: 10 x 10 sec holds
        \item Pelvic tilts: 2 sets x 10 reps
        \item Dead bug progression (modified): 2 sets x 8 reps
    \end{itemize}
\end{itemize}

\textbf{Balance Training}:
\begin{itemize}[leftmargin=*]
    \item Static standing balance exercises at parallel bars
    \item Weight shifting activities (anterior-posterior, medial-lateral)
    \item Tandem stance progression
    \item Single-leg stance (holding support as needed)
    \item Reaching activities outside base of support
    \item Step-over obstacles
\end{itemize}

\textbf{Gait Training}:
\begin{itemize}[leftmargin=*]
    \item Gait training with front-wheeled walker on level surfaces
    \item Focus on step length symmetry, heel strike, push-off
    \item Progress from contact guard to supervision to modified independence
    \item Advance distance as tolerated (goal 300+ feet)
    \item Outdoor gait training on varied terrain (grass, gravel, curbs)
    \item Reduce assistive device as appropriate (walker → cane → no device)
\end{itemize}

\textbf{Stair Training}:
\begin{itemize}[leftmargin=*]
    \item Stair negotiation with handrail (step-to pattern initially)
    \item 4 steps ascending/descending to match home environment
    \item Progress to step-over-step pattern
    \item Practice with carrying objects
\end{itemize}

\textbf{Modalities (as indicated)}:
\begin{itemize}[leftmargin=*]
    \item Ice after therapy sessions for pain management
    \item Electrical stimulation for hip abductor muscle re-education (if indicated)
    \item Ultrasound for soft tissue mobility (if indicated)
\end{itemize}

\textbf{Patient Education}:
\begin{itemize}[leftmargin=*]
    \item Hip precautions education and review
    \item Fall prevention strategies
    \item Proper use of assistive device
    \item Pain management techniques
    \item Activity pacing and energy conservation
\end{itemize}

\subsection*{4.2 Occupational Therapy Interventions}

\textbf{Frequency}: 3 sessions per week, 45 minutes per session, for 6-8 weeks

\textbf{ADL Training}:
\begin{itemize}[leftmargin=*]
    \item \textbf{Bathing}: Practice shower transfers with grab bars and shower chair, long-handled sponge technique
    \item \textbf{Lower Body Dressing}: Training with reacher, sock aid, elastic shoelaces, dressing stick
    \item \textbf{Toileting}: Practice with raised toilet seat and grab bars
    \item \textbf{Bed Mobility}: Log-roll technique, use of bed rail if needed
    \item \textbf{Kitchen Tasks}: Safe standing tolerance, use of walker basket to carry items
\end{itemize}

\textbf{Adaptive Equipment Training}:
\begin{itemize}[leftmargin=*]
    \item Reacher (32-inch) for dressing, picking up objects
    \item Sock aid and dressing stick for lower extremity dressing
    \item Long-handled shoe horn
    \item Long-handled sponge/bath brush
    \item Shower chair with back
    \item Raised toilet seat with arms
    \item Bedside commode (if bedroom upstairs initially)
\end{itemize}

\textbf{Home Management Training}:
\begin{itemize}[leftmargin=*]
    \item Light meal preparation (seated when possible)
    \item Laundry (modified techniques, avoid lifting heavy baskets)
    \item Safe reaching and bending techniques
    \item Organization strategies to minimize unnecessary walking
\end{itemize}

\textbf{Upper Extremity Strengthening}:
\begin{itemize}[leftmargin=*]
    \item Therapeutic putty for grip strength
    \item Weighted exercises for shoulder stability (needed for walker use)
    \item Fine motor coordination activities
\end{itemize}

\textbf{Energy Conservation and Work Simplification}:
\begin{itemize}[leftmargin=*]
    \item Activity pacing strategies
    \item Prioritization of daily tasks
    \item Use of rest breaks
    \item Organization to reduce unnecessary steps
\end{itemize}

\subsection*{4.3 Home Exercise Program (HEP)}

Patient provided with illustrated HEP to perform daily at home:

\begin{longtable}{|p{4cm}|p{4cm}|p{5cm}|}
\hline
\textbf{Exercise} & \textbf{Dosage} & \textbf{Instructions} \\ \hline
\endfirsthead
\hline
\textbf{Exercise} & \textbf{Dosage} & \textbf{Instructions} \\ \hline
\endhead
Ankle pumps & 3 x 10, 3x daily & Seated or lying, point toes up/down \\ \hline
Quadriceps sets & 3 x 10, 2x daily & Tighten thigh muscle, hold 5 sec \\ \hline
Hip abduction sidelying & 2 x 10, 1x daily & Lift top leg, hold 2 sec, lower slowly \\ \hline
Sit-to-stand & 2 x 10, 2x daily & Use walker, stand fully, sit slowly \\ \hline
Standing hip flexion & 2 x 10, 1x daily & Lift knee (respect 90° precaution) \\ \hline
Balance - standing & 3 x 30 sec, 2x daily & Stand at counter, reduce hand support as able \\ \hline
Walking & 10 min, 2-3x daily & With walker, gradually increase distance \\ \hline
\end{longtable}

\textbf{HEP Instructions}:
\begin{itemize}[leftmargin=*]
    \item Perform exercises on non-therapy days
    \item Stop if pain exceeds 4/10
    \item Maintain hip precautions at all times
    \item Progress per therapist instruction only
    \item Record completion in exercise log
\end{itemize}

\subsection*{4.4 Durable Medical Equipment (DME)}

\textbf{Recommended Equipment}:
\begin{itemize}[leftmargin=*]
    \item Front-wheeled walker (standard adult)
    \item Shower chair with back (adjustable height)
    \item Grab bars for shower (2 bars - vertical and horizontal)
    \item Raised toilet seat with arms
    \item Reacher (32-inch)
    \item Sock aid
    \item Long-handled shoe horn
    \item Long-handled sponge
    \item Bedside commode (if needed initially)
    \item Non-slip bath mat
\end{itemize}

% ===== SECTION 5: TREATMENT SCHEDULE =====
\section*{5. Treatment Schedule and Timeline}

\subsection*{Treatment Phases}

\begin{tabularx}{\textwidth}{|l|l|X|}
\hline
\textbf{Phase} & \textbf{Duration} & \textbf{Focus} \\ \hline
Acute/Early & Weeks 1-2 & Pain management, basic mobility, ADL training with equipment, safety \\ \hline
Intermediate & Weeks 3-6 & Strength/ROM progression, advanced balance, stair training, ADL refinement \\ \hline
Advanced & Weeks 7-10 & Community ambulation, IADL training, HEP independence, discharge prep \\ \hline
Transition & Weeks 11-12 & Reduce frequency, monitor independence, finalize home setup \\ \hline
\end{tabularx}

\subsection*{Session Frequency and Duration}

\begin{tabularx}{\textwidth}{|l|X|X|}
\hline
\textbf{Discipline} & \textbf{Frequency} & \textbf{Duration} \\ \hline
Physical Therapy & 3x/week & 45-60 min/session, 8-12 weeks total \\ \hline
Occupational Therapy & 3x/week & 45 min/session, 6-8 weeks total \\ \hline
Home Exercise Program & Daily (non-therapy days) & 30 min/day \\ \hline
\end{tabularx}

\subsection*{Progress Assessments}

\begin{itemize}[leftmargin=*]
    \item \textbf{Weekly}: Informal progress monitoring, pain levels, exercise tolerance
    \item \textbf{Biweekly}: Reassess key impairments (ROM, strength, balance measures)
    \item \textbf{Week 4}: Formal reassessment, FIM score, goal progress review, plan modification if needed
    \item \textbf{Week 8}: Comprehensive reassessment, discharge planning, final goal review
    \item \textbf{Discharge}: Final outcomes documentation, HEP review, follow-up recommendations
\end{itemize}

% ===== SECTION 6: OUTCOME MEASURES =====
\section*{6. Outcome Measures and Monitoring}

\subsection*{Standardized Assessments}

\begin{longtable}{|p{4.5cm}|p{3cm}|p{3cm}|p{3cm}|}
\hline
\textbf{Measure} & \textbf{Baseline} & \textbf{Goal} & \textbf{Frequency} \\ \hline
\endfirsthead
\hline
\textbf{Measure} & \textbf{Baseline} & \textbf{Goal} & \textbf{Frequency} \\ \hline
\endhead
FIM Score & [125/126] & [142/126] & Week 0, 4, 8, discharge \\ \hline
Berg Balance Scale & [38/56] & [$>$45/56] & Week 0, 4, 8, discharge \\ \hline
6-Minute Walk Test & [150 feet] & [$>$300 feet] & Week 0, 4, 8, discharge \\ \hline
Gait Speed & [0.4 m/s] & [$>$0.8 m/s] & Week 0, 4, 8, discharge \\ \hline
Pain (NRS 0-10) & [6/10 with activity] & [$<$3/10] & Each session \\ \hline
ROM - Hip Flexion & [70°] & [110°] & Biweekly \\ \hline
Strength - Hip Abductors & [2+/5] & [4/5] & Biweekly \\ \hline
\end{longtable}

\subsection*{Progress Indicators}

\textbf{Positive Progress}:
\begin{itemize}[leftmargin=*]
    \item Increasing ambulation distance
    \item Reduced level of assistance for ADLs
    \item Improved balance scores
    \item Decreased pain with activity
    \item Increased strength/ROM measurements
    \item Patient confidence and self-efficacy improving
\end{itemize}

\textbf{Barriers to Progress}:
\begin{itemize}[leftmargin=*]
    \item Inadequate pain control
    \item Poor therapy attendance or compliance
    \item Medical complications or setbacks
    \item Psychosocial factors (depression, anxiety, lack of support)
    \item Cognitive impairment affecting learning
\end{itemize}

% ===== SECTION 7: EXPECTED OUTCOMES =====
\section*{7. Expected Outcomes and Prognosis}

\subsection*{Rehabilitation Potential}

\textbf{Overall Prognosis}: [e.g., Good] - Patient is motivated, has good social support, no significant cognitive impairment, and appropriate medical management.

\textbf{Expected Functional Outcome}:
\begin{itemize}[leftmargin=*]
    \item Independent or supervision level for all basic ADLs
    \item Community ambulation with assistive device (walker or cane)
    \item Ability to negotiate stairs for home access
    \item Safe discharge home with DME and environmental modifications
    \item Return to modified IADL participation
\end{itemize}

\subsection*{Timeline for Key Milestones}

\begin{itemize}[leftmargin=*]
    \item \textbf{Week 2}: Transfers with supervision, basic ADLs with minimal assist
    \item \textbf{Week 4}: Ambulation 150 feet with walker/supervision, improved pain control
    \item \textbf{Week 6}: Stairs with handrail/supervision, ADLs mostly independent with equipment
    \item \textbf{Week 8}: Community ambulation 300+ feet, all ADLs independent, ready for discharge
\end{itemize}

% ===== SECTION 8: FOLLOW-UP AND DISCHARGE PLANNING =====
\section*{8. Follow-Up and Discharge Planning}

\subsection*{Discharge Criteria}

Patient ready for discharge when:
\begin{itemize}[leftmargin=*]
    \item Safe for home environment (with or without DME)
    \item Independent or supervision level for ADLs
    \item Patient/caregiver educated on HEP and safety
    \item DME obtained and home modifications completed
    \item Functional goals achieved or plateau reached
\end{itemize}

\subsection*{Discharge Recommendations}

\begin{itemize}[leftmargin=*]
    \item Continue HEP as prescribed, progress as tolerated
    \item Follow up with orthopedic surgeon at [timeframe]
    \item Consider outpatient therapy if continued progress expected
    \item Home health PT/OT if unable to access outpatient services
    \item Transition to community exercise program (e.g., senior center, aquatics)
\end{itemize}

\subsection*{Home Modifications and Safety}

\begin{itemize}[leftmargin=*]
    \item Install grab bars in shower (vertical and horizontal)
    \item Ensure adequate lighting, especially on stairs
    \item Remove scatter rugs and clutter
    \item Consider temporary bedroom on main floor if stairs difficult
    \item Rearrange furniture to create clear pathways
    \item Store frequently used items at accessible heights
\end{itemize}

\subsection*{Follow-Up Communication}

\begin{itemize}[leftmargin=*]
    \item Progress reports sent to referring physician biweekly
    \item Final discharge summary to all providers
    \item Home safety assessment completed
    \item DME delivered and training completed
    \item Emergency contact: Therapy department [phone]
\end{itemize}

% ===== SECTION 9: SAFETY AND PRECAUTIONS =====
\section*{9. Safety Considerations and Precautions}

\subsection*{Medical Precautions}

\begin{itemize}[leftmargin=*]
    \item \textbf{Hip Precautions} (post-ORIF):
    \begin{itemize}
        \item No hip flexion $>$90 degrees for 6-8 weeks
        \item No hip adduction past midline
        \item No internal rotation
        \item Sleep with abduction pillow
        \item Use elevated toilet seat and shower chair
    \end{itemize}
    
    \item \textbf{Weight-Bearing Status}: [e.g., WBAT - Weight-bearing as tolerated]
    
    \item \textbf{Anticoagulation}: On enoxaparin - use fall precautions, report bruising/bleeding
    
    \item \textbf{Pain Management}: Opioid use may cause drowsiness - schedule therapy before pain medication if possible
\end{itemize}

\subsection*{Fall Risk Management}

\textbf{Fall Risk Factors}:
\begin{itemize}[leftmargin=*]
    \item Recent surgery/hospitalization
    \item Impaired balance (Berg 38/56)
    \item Use of walker
    \item Pain medication (opioids)
    \item Environmental hazards at home
\end{itemize}

\textbf{Fall Prevention Strategies}:
\begin{itemize}[leftmargin=*]
    \item Consistent use of walker
    \item Non-slip footwear with closed heel
    \item Call for assistance for transfers initially
    \item Adequate lighting
    \item Avoid carrying items while walking (use walker basket)
    \item Balance training in therapy
    \item Home safety modifications
\end{itemize}

\subsection*{Contraindications to Treatment}

Hold or modify therapy if:
\begin{itemize}[leftmargin=*]
    \item Fever $>$101°F or signs of infection
    \item Uncontrolled pain ($>$7/10)
    \item Excessive swelling, warmth, redness at surgical site
    \item Chest pain, severe shortness of breath
    \item Dizziness, lightheadedness, abnormal vital signs
    \item Patient refusal or excessive fatigue
\end{itemize}

\subsection*{Emergency Procedures}

\begin{itemize}[leftmargin=*]
    \item \textbf{Fall During Therapy}: Assess for injury, vital signs, notify physician, incident report
    \item \textbf{Chest Pain/SOB}: Stop activity, call 911, notify physician
    \item \textbf{Excessive Pain}: Stop activity, apply ice, notify physician, reassess treatment plan
\end{itemize}

% ===== SECTION 10: PROVIDER SIGNATURE =====
\vspace{2em}

\section*{10. Rehabilitation Team Signatures}

\textbf{Physical Therapist}:\\[0.5em]
Signature: \rule{6cm}{0.5pt} \quad Date: \rule{3cm}{0.5pt}\\
Name/Credentials: \rule{6cm}{0.5pt}\\[1em]

\textbf{Occupational Therapist}:\\[0.5em]
Signature: \rule{6cm}{0.5pt} \quad Date: \rule{3cm}{0.5pt}\\
Name/Credentials: \rule{6cm}{0.5pt}\\[1em]

\textbf{Referring Physician Approval}:\\[0.5em]
Signature: \rule{6cm}{0.5pt} \quad Date: \rule{3cm}{0.5pt}\\
Name/Credentials: \rule{6cm}{0.5pt}\\

\vspace{2em}
\begin{center}
\rule{\textwidth}{1pt}\\
\textbf{End of Rehabilitation Treatment Plan}\\
This document contains confidential patient information protected by HIPAA.
\end{center}

\end{document}

% ========== NOTES FOR USERS ==========
%
% CUSTOMIZATION:
% - Replace all bracketed placeholders with patient-specific information
% - Adjust goals based on baseline assessment
% - Modify exercises based on patient tolerance and precautions
% - Update DME recommendations as needed
%
% COMPILATION:
% pdflatex rehabilitation_treatment_plan.tex

