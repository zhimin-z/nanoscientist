% General Medical Treatment Plan Template
% For primary care and chronic disease management
% Last updated: 2025

\documentclass[11pt,letterpaper]{article}

% Packages
\usepackage[top=1in,bottom=1in,left=1in,right=1in]{geometry}
\usepackage{amsmath,amssymb}
\usepackage[utf8]{inputenc}
\usepackage{graphicx}
\usepackage{array}
\usepackage{longtable}
\usepackage{booktabs}
\usepackage{enumitem}
\usepackage{xcolor}
\usepackage{fancyhdr}
\usepackage{lastpage}
\usepackage{tabularx}
\usepackage[most]{tcolorbox}

% Header and footer
\pagestyle{fancy}
\fancyhf{}
\lhead{General Medical Treatment Plan}
\rhead{Page \thepage\ of \pageref{LastPage}}
\lfoot{Date Created: \today}
\rfoot{Confidential Patient Information}

% Title formatting
\usepackage{titlesec}
\titleformat{\section}{\large\bfseries}{\thesection}{1em}{}
\titleformat{\subsection}{\normalsize\bfseries}{\thesubsection}{1em}{}

\begin{document}

% Title
\begin{center}
{\Large\bfseries MEDICAL TREATMENT PLAN}\\[0.5em]
{\large General Medicine \& Chronic Disease Management}\\[0.5em]
\rule{\textwidth}{1pt}
\end{center}

\vspace{1em}

% ===== TREATMENT PLAN HIGHLIGHTS (Foundation Medicine Model) =====
\begin{tcolorbox}[colback=blue!5!white,colframe=blue!75!black,title=\textbf{TREATMENT PLAN HIGHLIGHTS},fonttitle=\bfseries\large]

\textbf{Key Diagnosis:} [Primary diagnosis with ICD-10 code, severity/stage]

\vspace{0.3em}
\textbf{Primary Treatment Goals:}
\begin{itemize}[leftmargin=*,itemsep=0pt]
    \item [Goal 1 - e.g., Reduce HbA1c from 8.5\% to $<$7\% within 3 months]
    \item [Goal 2 - e.g., Achieve blood pressure $<$130/80 mmHg within 8 weeks]
    \item [Goal 3 - e.g., Weight loss of 7-10\% body weight over 6 months]
\end{itemize}

\vspace{0.3em}
\textbf{Main Interventions:}
\begin{itemize}[leftmargin=*,itemsep=0pt]
    \item \textit{Pharmacological:} [Key medications - e.g., Metformin 1000mg BID, Lisinopril 10mg daily]
    \item \textit{Non-pharmacological:} [Lifestyle modifications - e.g., Mediterranean diet, 150 min/week exercise]
    \item \textit{Monitoring:} [Key parameters - e.g., HbA1c every 3 months, home BP daily]
\end{itemize}

\vspace{0.3em}
\textbf{Timeline:} [Duration - e.g., Intensive initiation (4 weeks), Adjustment phase (8 weeks), Maintenance (ongoing)]

\end{tcolorbox}

\vspace{1em}

% ===== SECTION 1: PATIENT INFORMATION =====
\section*{1. Patient Information}

\textbf{HIPAA Notice}: All identifiable information must be removed or de-identified per Safe Harbor method before sharing this document. Remove: name, dates (except year), addresses, phone/fax, email, SSN, medical record numbers, account numbers, photos, and other unique identifiers.

\vspace{0.5em}

\begin{tabularx}{\textwidth}{|l|X|}
\hline
\textbf{Patient ID} & [De-identified code, e.g., PT-001] \\ \hline
\textbf{Age Range} & [e.g., 55-60 years] \\ \hline
\textbf{Sex} & [Male/Female/Other] \\ \hline
\textbf{Race/Ethnicity} & [If relevant to treatment] \\ \hline
\textbf{Date of Plan} & [Month/Year only] \\ \hline
\textbf{Provider} & [Name, MD/DO/NP/PA, Credentials] \\ \hline
\textbf{Facility} & [Healthcare facility name] \\ \hline
\end{tabularx}

\vspace{1em}

\subsection*{Active Medical Conditions}
\begin{itemize}[leftmargin=*]
    \item \textbf{Primary Diagnosis}: [Condition with ICD-10 code]
    \item \textbf{Secondary Diagnoses}:
    \begin{itemize}
        \item [Comorbidity 1 with ICD-10 code]
        \item [Comorbidity 2 with ICD-10 code]
        \item [Additional conditions as needed]
    \end{itemize}
\end{itemize}

\subsection*{Current Medications}
\begin{longtable}{|p{3.5cm}|p{2cm}|p{2cm}|p{5cm}|}
\hline
\textbf{Medication} & \textbf{Dose} & \textbf{Frequency} & \textbf{Indication} \\ \hline
\endfirsthead
\hline
\textbf{Medication} & \textbf{Dose} & \textbf{Frequency} & \textbf{Indication} \\ \hline
\endhead
Medication 1 & [e.g., 10mg] & [e.g., daily] & [Indication] \\ \hline
Medication 2 & [e.g., 50mg] & [e.g., BID] & [Indication] \\ \hline
[Add rows as needed] & & & \\ \hline
\end{longtable}

\subsection*{Allergies}
\begin{itemize}[leftmargin=*]
    \item \textbf{Drug Allergies}: [List medications and reactions, or NKDA]
    \item \textbf{Food/Environmental}: [If relevant to treatment]
\end{itemize}

\subsection*{Baseline Assessment}
\begin{itemize}[leftmargin=*]
    \item \textbf{Functional Status}: [Independent/requires assistance/dependent for ADLs]
    \item \textbf{Cognitive Status}: [Alert and oriented/impairment if present]
    \item \textbf{Social Support}: [Lives alone/with family, support system]
    \item \textbf{Key Baseline Values}: [e.g., HbA1c 8.5\%, BP 145/90, BMI 32, eGFR 55]
\end{itemize}

% ===== SECTION 2: DIAGNOSIS AND ASSESSMENT =====
\section*{2. Diagnosis and Assessment Summary}

\subsection*{Primary Diagnosis}
\textbf{Diagnosis}: [Full diagnosis name]\\
\textbf{ICD-10 Code}: [e.g., E11.9 for Type 2 Diabetes Mellitus without complications]\\
\textbf{Severity}: [Mild/Moderate/Severe or stage classification]\\
\textbf{Duration}: [Time since diagnosis]

\subsection*{Clinical Presentation}
[Describe current symptoms, functional limitations, and impact on quality of life. Include relevant exam findings and diagnostic test results.]

\subsection*{Risk Stratification}
\begin{itemize}[leftmargin=*]
    \item \textbf{Cardiovascular Risk}: [e.g., ASCVD 10-year risk 15\%]
    \item \textbf{Complications Risk}: [e.g., high risk for diabetic nephropathy]
    \item \textbf{Other Risk Factors}: [e.g., fall risk, frailty, polypharmacy]
\end{itemize}

\subsection*{Prognostic Considerations}
[Discuss expected disease course, factors affecting prognosis, and rationale for treatment intensity.]

% ===== SECTION 3: TREATMENT GOALS =====
\section*{3. Treatment Goals (SMART Format)}

\textbf{SMART Criteria}: All goals should be \textbf{S}pecific, \textbf{M}easurable, \textbf{A}chievable, \textbf{R}elevant, and \textbf{T}ime-bound.

\subsection*{Short-Term Goals (1-3 months)}

\begin{enumerate}[leftmargin=*]
    \item \textbf{Goal 1}: [e.g., Reduce HbA1c from 8.5\% to $<$7.5\%]
    \begin{itemize}
        \item \textit{Specific}: Reduce HbA1c by at least 1 percentage point
        \item \textit{Measurable}: HbA1c lab value
        \item \textit{Achievable}: With medication initiation and lifestyle changes
        \item \textit{Relevant}: Reduce microvascular complication risk
        \item \textit{Time-bound}: Achieve within 3 months (next follow-up)
    \end{itemize}
    
    \item \textbf{Goal 2}: [e.g., Decrease systolic blood pressure to $<$130 mmHg]
    \begin{itemize}
        \item \textit{Specific}: Achieve BP $<$130/80 mmHg
        \item \textit{Measurable}: Office and home BP measurements
        \item \textit{Achievable}: With medication optimization
        \item \textit{Relevant}: Reduce cardiovascular event risk
        \item \textit{Time-bound}: Within 8 weeks
    \end{itemize}
    
    \item \textbf{Goal 3}: [Additional short-term goal]
\end{enumerate}

\subsection*{Long-Term Goals (6-12 months)}

\begin{enumerate}[leftmargin=*]
    \item \textbf{Goal 1}: [e.g., Maintain HbA1c $<$7\% and prevent diabetic complications]
    \begin{itemize}
        \item \textit{Success criteria}: HbA1c $<$7\%, no new retinopathy/nephropathy/neuropathy
        \item \textit{Timeline}: Ongoing, assessed every 3-6 months
    \end{itemize}
    
    \item \textbf{Goal 2}: [e.g., Weight loss of 15 pounds (7\% body weight)]
    \begin{itemize}
        \item \textit{Success criteria}: BMI reduction from 32 to $<$30
        \item \textit{Timeline}: 6-12 months at 1-2 lbs/week
    \end{itemize}
    
    \item \textbf{Goal 3}: [e.g., Achieve LDL cholesterol $<$70 mg/dL]
    
    \item \textbf{Goal 4}: [Additional long-term goal as needed]
\end{enumerate}

\subsection*{Patient-Centered Goals}
\begin{itemize}[leftmargin=*]
    \item \textbf{Patient Priority 1}: [e.g., "Feel more energetic throughout the day"]
    \item \textbf{Patient Priority 2}: [e.g., "Avoid insulin injections if possible"]
    \item \textbf{Patient Priority 3}: [e.g., "Continue working full-time"]
\end{itemize}

% ===== SECTION 4: INTERVENTIONS =====
\section*{4. Interventions}

\subsection*{4.1 Pharmacological Interventions}

\begin{longtable}{|p{3cm}|p{2cm}|p{2cm}|p{6.5cm}|}
\hline
\textbf{Medication} & \textbf{Dose} & \textbf{Frequency} & \textbf{Instructions \& Rationale} \\ \hline
\endfirsthead
\hline
\textbf{Medication} & \textbf{Dose} & \textbf{Frequency} & \textbf{Instructions \& Rationale} \\ \hline
\endhead

[e.g., Metformin] & 500mg & BID & \textbf{Start:} Take with meals to reduce GI upset. \textbf{Titration:} Increase to 1000mg BID after 2 weeks if tolerated. \textbf{Target:} 2000mg daily. \textbf{Rationale:} First-line for T2DM, reduces hepatic glucose production. \\ \hline

[e.g., Lisinopril] & 10mg & Daily & \textbf{Instructions:} Take in morning. Monitor BP at home. \textbf{Titration:} May increase to 20mg if BP not at goal in 4 weeks. \textbf{Rationale:} ACE inhibitor for HTN and renal protection in diabetes. \\ \hline

[Additional medications] & & & \\ \hline
\end{longtable}

\textbf{Medication Safety Considerations}:
\begin{itemize}[leftmargin=*]
    \item \textbf{Drug Interactions}: [List relevant interactions to monitor]
    \item \textbf{Adverse Effects to Monitor}: [e.g., metformin - GI upset, lactic acidosis; lisinopril - cough, hyperkalemia, angioedema]
    \item \textbf{Contraindications}: [e.g., metformin if eGFR $<$30]
    \item \textbf{Pregnancy Category}: [If relevant to patient]
\end{itemize}

\subsection*{4.2 Non-Pharmacological Interventions}

\textbf{Lifestyle Modifications}:
\begin{itemize}[leftmargin=*]
    \item \textbf{Diet}:
    \begin{itemize}
        \item Mediterranean or DASH diet pattern
        \item Carbohydrate counting: 45-60g per meal
        \item Reduce saturated fat $<$7\% of calories
        \item Sodium restriction $<$2300mg daily
        \item Referral to registered dietitian
    \end{itemize}
    
    \item \textbf{Exercise}:
    \begin{itemize}
        \item Aerobic exercise: 150 minutes/week moderate intensity (e.g., brisk walking 30 min 5x/week)
        \item Resistance training: 2-3 sessions/week
        \item Reduce sedentary time, stand/move every 30 minutes
    \end{itemize}
    
    \item \textbf{Smoking Cessation}: [If applicable]
    \begin{itemize}
        \item Nicotine replacement therapy (patch, gum, lozenge)
        \item Consider varenicline or bupropion
        \item Behavioral counseling: 1-800-QUIT-NOW
        \item Target quit date: [specific date within 1 month]
    \end{itemize}
    
    \item \textbf{Weight Management}:
    \begin{itemize}
        \item Target: 7-10\% body weight loss over 6 months
        \item Caloric deficit: 500-750 kcal/day
        \item Weekly self-weighing and food diary
        \item Consider weight loss program or app
    \end{itemize}
    
    \item \textbf{Sleep Hygiene}:
    \begin{itemize}
        \item Target 7-9 hours nightly
        \item Consistent sleep schedule
        \item Screen for sleep apnea if indicated
    \end{itemize}
    
    \item \textbf{Stress Management}:
    \begin{itemize}
        \item Mindfulness or meditation practice
        \item Stress reduction techniques
        \item Adequate social support
    \end{itemize}
\end{itemize}

\textbf{Self-Management and Monitoring}:
\begin{itemize}[leftmargin=*]
    \item \textbf{Blood Glucose Monitoring}: [Frequency, e.g., fasting and 2hr post-prandial 3x/week]
    \item \textbf{Home Blood Pressure}: [Frequency, e.g., daily in AM, record in log]
    \item \textbf{Weight Tracking}: [e.g., weekly on same day/time]
    \item \textbf{Symptom Diary}: [Track relevant symptoms]
    \item \textbf{Medication Adherence}: [Pill box, reminder app]
\end{itemize}

\subsection*{4.3 Procedural and Referral Interventions}

\begin{itemize}[leftmargin=*]
    \item \textbf{Specialist Referrals}:
    \begin{itemize}
        \item [e.g., Endocrinology consultation for diabetes management]
        \item [e.g., Ophthalmology for annual dilated eye exam]
        \item [e.g., Podiatry for diabetic foot exam]
        \item [e.g., Nephrology if eGFR $<$30 or proteinuria]
    \end{itemize}
    
    \item \textbf{Diagnostic Testing Schedule}:
    \begin{itemize}
        \item [e.g., HbA1c every 3 months until at goal, then every 6 months]
        \item [e.g., Lipid panel annually]
        \item [e.g., Urine albumin-to-creatinine ratio annually]
        \item [e.g., Comprehensive metabolic panel every 6 months]
    \end{itemize}
    
    \item \textbf{Preventive Care}:
    \begin{itemize}
        \item Influenza vaccine annually
        \item Pneumococcal vaccines (PCV20 or PCV15+PPSV23)
        \item COVID-19 vaccination per current guidelines
        \item Age-appropriate cancer screenings
        \item [Other preventive measures as indicated]
    \end{itemize}
\end{itemize}

% ===== SECTION 5: TIMELINE AND SCHEDULE =====
\section*{5. Timeline and Schedule}

\subsection*{Treatment Phases}

\begin{tabularx}{\textwidth}{|l|X|X|}
\hline
\textbf{Phase} & \textbf{Timeframe} & \textbf{Focus} \\ \hline
Intensive Initiation & Weeks 1-4 & Medication titration, lifestyle education, baseline monitoring \\ \hline
Adjustment & Weeks 5-12 & Optimize medications, reinforce lifestyle changes, assess goal progress \\ \hline
Maintenance & Months 4-12 & Sustain improvements, prevent complications, long-term adherence \\ \hline
Ongoing & $>$12 months & Chronic disease management, annual assessments, update goals \\ \hline
\end{tabularx}

\subsection*{Appointment Schedule}

\begin{tabularx}{\textwidth}{|l|X|X|}
\hline
\textbf{Timepoint} & \textbf{Visit Type} & \textbf{Key Activities} \\ \hline
Week 2 & Phone/telehealth & Check medication tolerance, answer questions \\ \hline
Week 4 & Office visit & Medication adjustment, BP check, labs, review monitoring \\ \hline
Week 8 & Office visit & Assess progress toward goals, reinforce lifestyle \\ \hline
Month 3 & Office visit & HbA1c, comprehensive assessment, goal evaluation \\ \hline
Month 6 & Office visit & Reassess all goals, update plan, labs \\ \hline
Month 12 & Annual exam & Comprehensive evaluation, preventive care, specialty referrals \\ \hline
Ongoing & Every 3-6 months & Per chronic disease management protocol \\ \hline
\end{tabularx}

\subsection*{Milestone Assessments}

\begin{itemize}[leftmargin=*]
    \item \textbf{Month 1}: Medication tolerance, lifestyle initiation, home monitoring established
    \item \textbf{Month 3}: HbA1c $<$7.5\%, BP $<$130/80, 3-5 lb weight loss
    \item \textbf{Month 6}: HbA1c $<$7\%, sustained BP control, 8-10 lb weight loss
    \item \textbf{Month 12}: All long-term goals achieved or revised, complication screening complete
\end{itemize}

% ===== SECTION 6: MONITORING PARAMETERS =====
\section*{6. Monitoring Parameters}

\subsection*{Clinical Outcomes to Track}

\begin{longtable}{|p{4cm}|p{3cm}|p{3cm}|p{4cm}|}
\hline
\textbf{Parameter} & \textbf{Baseline} & \textbf{Target} & \textbf{Frequency} \\ \hline
\endfirsthead
\hline
\textbf{Parameter} & \textbf{Baseline} & \textbf{Target} & \textbf{Frequency} \\ \hline
\endhead

HbA1c & [e.g., 8.5\%] & $<$7\% & Every 3 months until stable, then every 6 months \\ \hline
Fasting Glucose & [e.g., 165 mg/dL] & 80-130 mg/dL & Home monitoring per schedule \\ \hline
Blood Pressure & [e.g., 145/90] & $<$130/80 mmHg & Daily home, every office visit \\ \hline
Weight/BMI & [e.g., 210 lb, BMI 32] & 195 lb, BMI $<$30 & Weekly at home, every visit \\ \hline
LDL Cholesterol & [e.g., 135 mg/dL] & $<$70 mg/dL & Every 6-12 months \\ \hline
eGFR & [e.g., 55 mL/min] & Stable, $>$45 & Every 6 months \\ \hline
Urine ACR & [e.g., normal] & $<$30 mg/g & Annually \\ \hline
[Add additional parameters] & & & \\ \hline
\end{longtable}

\subsection*{Assessment Tools and Scales}

\begin{itemize}[leftmargin=*]
    \item \textbf{Diabetes Distress Scale}: [Assess emotional burden of diabetes management]
    \item \textbf{SF-12 or PROMIS}: [Quality of life assessment]
    \item \textbf{Medication Adherence}: [Morisky scale or refill tracking]
    \item \textbf{[Other relevant scales]}: [e.g., PHQ-2 for depression screening]
\end{itemize}

\subsection*{Safety Monitoring}

\begin{itemize}[leftmargin=*]
    \item \textbf{Hypoglycemia}: Frequency of blood glucose $<$70 mg/dL, symptoms
    \item \textbf{Medication Adverse Effects}: GI upset, cough, dizziness, other symptoms
    \item \textbf{Hyperkalemia}: Potassium level if on ACE inhibitor/ARB
    \item \textbf{Renal Function}: Monitor eGFR for metformin safety, ACE/ARB effects
\end{itemize}

\subsection*{Thresholds for Intervention}

\begin{itemize}[leftmargin=*]
    \item \textbf{Urgent}: Blood glucose $>$300 or $<$50, BP $>$180/110, chest pain, severe symptoms
    \item \textbf{Escalate Treatment}: No improvement in HbA1c after 3 months, BP above goal after 8 weeks
    \item \textbf{Modify Plan}: Intolerable side effects, patient preference change, new comorbidities
\end{itemize}

% ===== SECTION 7: EXPECTED OUTCOMES =====
\section*{7. Expected Outcomes and Prognosis}

\textbf{Anticipated Treatment Response}: With adherence, expect HbA1c reduction of 1-1.5\%, BP reduction of 10-15 mmHg, and 5-10\% weight loss over 6 months. Improvements visible at 4-8 weeks (BP, glucose), with HbA1c changes by 3 months.

\vspace{0.5em}
\textbf{Long-Term Benefits}: Reduced complication risk (cardiovascular events, retinopathy, nephropathy), improved quality of life, maintained independence and functional status.

% ===== SECTION 8: FOLLOW-UP PLAN =====
\section*{8. Follow-Up Plan}

\subsection*{Scheduled Appointments}

\begin{itemize}[leftmargin=*]
    \item \textbf{Next Visit}: [Date/timeframe - e.g., 4 weeks from today]
    \item \textbf{Visit Purpose}: [Medication adjustment, lab review, goal assessment]
    \item \textbf{Ongoing Schedule}: See Appointment Schedule in Section 5
\end{itemize}

\subsection*{Communication Plan}

\begin{itemize}[leftmargin=*]
    \item \textbf{Between-Visit Contact}: Phone call at 2 weeks to assess medication tolerance
    \item \textbf{Lab Results}: Will call with results within 3-5 business days
    \item \textbf{Questions}: Call office at [phone], patient portal messaging
    \item \textbf{Prescription Refills}: Via patient portal or pharmacy automated refill
\end{itemize}

\subsection*{Emergency Procedures}

\textbf{Call 911 immediately for}:
\begin{itemize}[leftmargin=*]
    \item Chest pain, shortness of breath, or stroke symptoms
    \item Severe hypoglycemia with confusion or loss of consciousness
    \item Severe allergic reaction (angioedema, anaphylaxis)
\end{itemize}

\textbf{Call office same day for}:
\begin{itemize}[leftmargin=*]
    \item Blood glucose consistently $>$300 or $<$60 mg/dL
    \item Blood pressure $>$180/110 mmHg
    \item Persistent severe medication side effects
    \item Fever, infection, or acute illness (may need medication adjustment)
\end{itemize}

\subsection*{Transition Planning}

\begin{itemize}[leftmargin=*]
    \item \textbf{If Hospitalized}: Provide this treatment plan to hospital team, resume medications on discharge
    \item \textbf{Specialist Co-Management}: Share plan with all specialists, coordinate medication changes
    \item \textbf{Future Considerations}: [e.g., may need insulin if oral medications insufficient]
\end{itemize}

% ===== SECTION 9: PATIENT EDUCATION =====
\section*{9. Patient Education and Self-Management}

\textbf{Key Education Topics}: Disease understanding, complication risks, treatment rationale, self-monitoring techniques (glucose, BP), medication administration, diet/nutrition basics, exercise safety, sick day management.

\vspace{0.5em}
\textbf{Critical Warning Signs}:
\begin{itemize}[leftmargin=*,itemsep=0pt]
    \item \textit{Emergency (call 911)}: Chest pain, severe hypoglycemia with confusion, stroke symptoms
    \item \textit{Call office same day}: Glucose $>$300 or $<$60 mg/dL, BP $>$180/110, severe medication side effects
    \item \textit{Urgent evaluation}: Diabetic foot wounds, severe hyperglycemia with symptoms
\end{itemize}

\vspace{0.5em}
\textbf{Support Resources}: DSMES referral, registered dietitian, educational materials, support groups, tracking technology, financial assistance programs as needed.

% ===== SECTION 10: RISK MITIGATION AND SAFETY =====
\section*{10. Risk Mitigation and Safety}

\textbf{Key Medication Safety Concerns}:
\begin{itemize}[leftmargin=*,itemsep=0pt]
    \item \textit{Metformin}: Monitor eGFR every 6 months; hold if eGFR $<$30, during acute illness, or 48 hours before contrast
    \item \textit{ACE inhibitor}: Check K+ and creatinine at 1-2 weeks, then every 6 months; hold during dehydration/AKI
    \item \textit{Hypoglycemia}: Low risk without insulin/sulfonylureas; educate on recognition and 15-15 rule
\end{itemize}

\vspace{0.5em}
\textbf{Complication Prevention}: Annual eye exam, foot exam, and urine ACR; aspirin if ASCVD risk $>$10\%; BP and glucose control reduces cardiovascular, retinopathy, nephropathy, and neuropathy risks.

\vspace{0.5em}
\textbf{Emergency Actions}: Severe hypoglycemia ($<$50, confusion) - glucagon then 911; chest pain/stroke - call 911; hyperglycemia $>$300 with symptoms - hydrate and call office; severe medication side effects - stop medication, call same day.

% ===== SECTION 11: PROVIDER SIGNATURE =====
\vspace{2em}

\section*{11. Provider Signature and Attestation}

I have reviewed this treatment plan with the patient. The patient demonstrates understanding of the diagnosis, treatment rationale, goals, interventions, self-management requirements, warning signs, and when to seek emergency care. The patient agrees to this treatment plan and has had the opportunity to ask questions. Shared decision-making was employed, and patient preferences were incorporated.

\vspace{1em}

\begin{tabular}{ll}
Provider Signature: & \rule{7cm}{0.5pt} \\[1em]
Provider Name/Credentials: & \rule{7cm}{0.5pt} \\[1em]
Date: & \rule{4cm}{0.5pt} \\[2em]
\end{tabular}

\subsection*{Patient Acknowledgment (Optional)}

I have reviewed this treatment plan with my healthcare provider. I understand my diagnosis, treatment goals, medications, lifestyle recommendations, self-monitoring requirements, and when to seek medical attention. I agree to follow this plan and contact my provider with questions or concerns.

\vspace{1em}

\begin{tabular}{ll}
Patient/Representative Signature: & \rule{7cm}{0.5pt} \\[1em]
Date: & \rule{4cm}{0.5pt} \\
\end{tabular}

\vspace{2em}
\begin{center}
\rule{\textwidth}{1pt}\\
\textbf{End of Treatment Plan}\\
This document contains confidential patient information protected by HIPAA.
\end{center}

\end{document}

% ========== NOTES FOR USERS ==========
%
% CUSTOMIZATION INSTRUCTIONS:
% 1. Replace all bracketed placeholders [like this] with patient-specific information
% 2. Remove or add sections as appropriate for the clinical condition
% 3. Ensure all SMART goals meet criteria (Specific, Measurable, Achievable, Relevant, Time-bound)
% 4. Include evidence-based interventions per current clinical guidelines
% 5. De-identify all protected health information before sharing
%
% COMPILATION:
% pdflatex general_medical_treatment_plan.tex
%
% VALIDATION:
% Run check_completeness.py and validate_treatment_plan.py before finalizing

