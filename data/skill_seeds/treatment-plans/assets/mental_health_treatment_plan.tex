% Mental Health Treatment Plan Template
% For psychiatric and behavioral health treatment
% Last updated: 2025

\documentclass[11pt,letterpaper]{article}

% Packages
\usepackage[top=1in,bottom=1in,left=1in,right=1in]{geometry}
\usepackage{amsmath,amssymb}
\usepackage[utf8]{inputenc}
\usepackage{graphicx}
\usepackage{array}
\usepackage{longtable}
\usepackage{booktabs}
\usepackage{enumitem}
\usepackage{xcolor}
\usepackage{fancyhdr}
\usepackage{lastpage}
\usepackage{tabularx}
\usepackage[most]{tcolorbox}

% Header and footer
\pagestyle{fancy}
\fancyhf{}
\lhead{Mental Health Treatment Plan}
\rhead{Page \thepage\ of \pageref{LastPage}}
\lfoot{Date Created: \today}
\rfoot{Confidential Patient Information}

% Title formatting
\usepackage{titlesec}
\titleformat{\section}{\large\bfseries}{\thesection}{1em}{}
\titleformat{\subsection}{\normalsize\bfseries}{\thesubsection}{1em}{}

\begin{document}

% Title
\begin{center}
{\Large\bfseries MENTAL HEALTH TREATMENT PLAN}\\[0.5em]
{\large Psychiatric \& Behavioral Health Services}\\[0.5em]
\rule{\textwidth}{1pt}
\end{center}

\vspace{1em}

% ===== TREATMENT PLAN HIGHLIGHTS (Foundation Medicine Model) =====
\begin{tcolorbox}[colback=purple!5!white,colframe=purple!75!black,title=\textbf{TREATMENT PLAN HIGHLIGHTS},fonttitle=\bfseries\large]

\textbf{Key Diagnosis:} [Primary psychiatric diagnosis - e.g., Major Depressive Disorder, moderate (DSM-5 296.32)]

\vspace{0.3em}
\textbf{Primary Treatment Goals:}
\begin{itemize}[leftmargin=*,itemsep=0pt]
    \item [Goal 1 - e.g., Reduce PHQ-9 score from 18 to $<$10 within 12 weeks]
    \item [Goal 2 - e.g., Return to work full-time within 3 months]
    \item [Goal 3 - e.g., Develop 3 effective coping strategies for stress management]
\end{itemize}

\vspace{0.3em}
\textbf{Main Interventions:}
\begin{itemize}[leftmargin=*,itemsep=0pt]
    \item \textit{Psychotherapy:} [Modality - e.g., Cognitive Behavioral Therapy (CBT) weekly for 16 sessions]
    \item \textit{Medication:} [Key medications - e.g., Sertraline 50mg daily, titrate to 100mg]
    \item \textit{Safety:} [Crisis plan in place, emergency contacts established]
\end{itemize}

\vspace{0.3em}
\textbf{Timeline:} [Duration - e.g., Acute treatment (12 weeks), Continuation (4-6 months), Maintenance (ongoing)]

\end{tcolorbox}

\vspace{1em}

% ===== SECTION 1: PATIENT INFORMATION =====
\section*{1. Patient Information}

\textbf{HIPAA Notice}: De-identify all protected health information per Safe Harbor method before sharing.

\vspace{0.5em}

\begin{tabularx}{\textwidth}{|l|X|}
\hline
\textbf{Patient ID} & [De-identified code, e.g., MH-001] \\ \hline
\textbf{Age Range} & [e.g., 30-35 years] \\ \hline
\textbf{Sex} & [Male/Female/Other] \\ \hline
\textbf{Gender Identity} & [If relevant and disclosed] \\ \hline
\textbf{Pronouns} & [Patient's preferred pronouns] \\ \hline
\textbf{Date of Plan} & [Month/Year only] \\ \hline
\textbf{Treating Provider} & [Psychiatrist/Psychologist/LCSW/NP Name, Credentials] \\ \hline
\textbf{Treatment Setting} & [Outpatient/IOP/PHP/Inpatient] \\ \hline
\textbf{Facility} & [Mental health center/clinic name] \\ \hline
\end{tabularx}

\vspace{1em}

\subsection*{Presenting Problem}

\textbf{Chief Complaint}: [Patient's own words, e.g., "I've been feeling really down and can't get motivated to do anything"]

\textbf{History of Present Illness}:
[Detailed description of current symptoms, onset, duration, severity, precipitating factors, impact on functioning. Example: Patient reports depressed mood, anhedonia, fatigue, and difficulty concentrating for past 3 months, following job loss. Symptoms have progressively worsened, now affecting ability to complete daily tasks and maintain social relationships.]

\subsection*{Psychiatric History}

\begin{itemize}[leftmargin=*]
    \item \textbf{Previous Psychiatric Diagnoses}: [e.g., Major Depressive Disorder, diagnosed 5 years ago]
    \item \textbf{Previous Treatment}:
    \begin{itemize}
        \item Psychotherapy: [e.g., CBT for 6 months in 2020, helpful]
        \item Medications: [e.g., Sertraline 100mg 2020-2021, discontinued due to side effects]
        \item Hospitalizations: [e.g., One psychiatric hospitalization in 2019 for suicidal ideation]
    \end{itemize}
    \item \textbf{Family Psychiatric History}: [e.g., Mother with depression, paternal uncle with bipolar disorder]
\end{itemize}

\subsection*{Substance Use History}

\begin{itemize}[leftmargin=*]
    \item \textbf{Alcohol}: [e.g., Social use, 2-3 drinks per week, denies binge drinking]
    \item \textbf{Tobacco}: [e.g., Non-smoker]
    \item \textbf{Cannabis}: [e.g., Previously daily use, quit 6 months ago]
    \item \textbf{Other Substances}: [e.g., Denies other illicit drug use]
    \item \textbf{Substance Use Disorder}: [e.g., Cannabis use disorder, in remission]
\end{itemize}

\subsection*{Medical History}

\begin{itemize}[leftmargin=*]
    \item \textbf{Chronic Medical Conditions}: [e.g., Hypothyroidism, well-controlled on levothyroxine]
    \item \textbf{Current Medications}: [e.g., Levothyroxine 100mcg daily]
    \item \textbf{Allergies}: [NKDA or list medication allergies and reactions]
\end{itemize}

\subsection*{Social History and Support}

\begin{itemize}[leftmargin=*]
    \item \textbf{Living Situation}: [e.g., Lives alone in apartment, safe housing]
    \item \textbf{Employment}: [e.g., Recently unemployed (3 months), previously worked as accountant]
    \item \textbf{Education}: [e.g., Bachelor's degree in accounting]
    \item \textbf{Marital/Relationship Status}: [e.g., Single, not in relationship]
    \item \textbf{Social Support}: [e.g., Close relationship with sister, few friends, isolated recently]
    \item \textbf{Financial Stressors}: [e.g., Unemployment causing financial strain]
    \item \textbf{Legal Issues}: [e.g., None]
    \item \textbf{Trauma History}: [e.g., Reports childhood emotional abuse, no recent trauma]
\end{itemize}

% ===== SECTION 2: PSYCHIATRIC ASSESSMENT =====
\section*{2. Psychiatric Assessment and Diagnosis}

\subsection*{Mental Status Examination}

\begin{itemize}[leftmargin=*]
    \item \textbf{Appearance}: [e.g., Casually dressed, fair grooming, appropriate for season]
    \item \textbf{Behavior}: [e.g., Cooperative, fair eye contact, psychomotor retardation noted]
    \item \textbf{Speech}: [e.g., Soft volume, slow rate, decreased spontaneity]
    \item \textbf{Mood}: [e.g., "Depressed and hopeless" - patient's own words]
    \item \textbf{Affect}: [e.g., Constricted, dysphoric, congruent with mood]
    \item \textbf{Thought Process}: [e.g., Linear, goal-directed, no tangentiality or loose associations]
    \item \textbf{Thought Content}:
    \begin{itemize}
        \item Suicidal ideation: [e.g., Passive SI present ("wish I wouldn't wake up"), denies active SI/plan/intent]
        \item Homicidal ideation: [e.g., Denied]
        \item Delusions: [e.g., None identified]
        \item Obsessions/compulsions: [e.g., None]
    \end{itemize}
    \item \textbf{Perceptions}: [e.g., No hallucinations (auditory, visual, tactile) reported or observed]
    \item \textbf{Cognition}:
    \begin{itemize}
        \item Orientation: [e.g., Oriented to person, place, time, situation]
        \item Memory: [e.g., Intact for recent and remote events]
        \item Concentration: [e.g., Impaired, difficulty with serial 7s]
        \item Insight: [e.g., Fair - recognizes need for treatment]
        \item Judgment: [e.g., Fair to good - makes reasonable decisions]
    \end{itemize}
\end{itemize}

\subsection*{Diagnostic Assessment}

\textbf{Primary Diagnosis}: [e.g., Major Depressive Disorder, Recurrent Episode, Moderate]\\
\textbf{DSM-5 Code}: [e.g., F33.1]

\textbf{DSM-5 Criteria Met}:
\begin{itemize}[leftmargin=*]
    \item Depressed mood most of the day, nearly every day (patient report, observed affect)
    \item Markedly diminished interest or pleasure in activities (anhedonia)
    \item Significant weight loss (10 lbs in 2 months)
    \item Insomnia nearly every night (difficulty falling and staying asleep)
    \item Fatigue and loss of energy nearly every day
    \item Feelings of worthlessness and guilt
    \item Diminished ability to think and concentrate
    \item Duration: 3 months
    \item Significant distress and impairment in occupational and social functioning
\end{itemize}

\textbf{Secondary Diagnoses}:
\begin{itemize}[leftmargin=*]
    \item [e.g., Cannabis Use Disorder, Mild, In Sustained Remission] (DSM-5: F12.11)
    \item [e.g., Unspecified Anxiety Disorder] (DSM-5: F41.9)
\end{itemize}

\subsection*{Symptom Severity Assessment}

\begin{tabularx}{\textwidth}{|l|c|c|X|}
\hline
\textbf{Assessment Tool} & \textbf{Score} & \textbf{Interpretation} & \textbf{Notes} \\ \hline
PHQ-9 (Depression) & 18/27 & Moderately severe depression & Target $<$10 for remission \\ \hline
GAD-7 (Anxiety) & 12/21 & Moderate anxiety & Target $<$5 \\ \hline
PCL-5 (PTSD) & N/A & Not administered & Consider if trauma symptoms emerge \\ \hline
C-SSRS (Suicide Risk) & Level 3 & Passive SI, no intent/plan & Requires safety planning \\ \hline
AUDIT (Alcohol) & 3/40 & Low risk & No current concern \\ \hline
\end{tabularx}

\subsection*{Functional Impairment}

\textbf{Impact on Daily Functioning}:
\begin{itemize}[leftmargin=*]
    \item \textbf{Occupational}: Unable to work currently, difficulty with job search due to lack of motivation
    \item \textbf{Social}: Withdrawn from friends, decreased social activities, isolating at home
    \item \textbf{Self-Care}: Difficulty maintaining hygiene, skipping meals, irregular sleep
    \item \textbf{Relationships}: Strained relationships due to irritability and withdrawal
    \item \textbf{Physical Health}: Decreased exercise, poor nutrition
\end{itemize}

\subsection*{Risk Assessment}

\textbf{Suicide Risk}: [e.g., Low to Moderate]
\begin{itemize}[leftmargin=*]
    \item \textit{Risk Factors}: Depression, unemployment, social isolation, passive SI, previous suicide attempt (2019)
    \item \textit{Protective Factors}: Engaged in treatment, close relationship with sister, denies current intent/plan, future-oriented (wants to get better)
    \item \textit{Current Status}: Passive SI only, no active ideation, plan, or intent. Contracts for safety.
\end{itemize}

\textbf{Homicide/Violence Risk}: [e.g., Low] - No homicidal ideation, no history of violence

% ===== SECTION 3: TREATMENT GOALS =====
\section*{3. Treatment Goals (SMART Format)}

\subsection*{3.1 Short-Term Goals (4-8 weeks)}

\textbf{Symptom Reduction Goals}:
\begin{enumerate}[leftmargin=*]
    \item \textbf{Depression}: Reduce PHQ-9 score from 18 to $<$10 (minimal depression) within 8 weeks through medication and psychotherapy.
    \begin{itemize}
        \item \textit{Measurable}: PHQ-9 assessment every 2 weeks
        \item \textit{Achievable}: With SSRI and weekly CBT
        \item \textit{Time-bound}: 8 weeks
    \end{itemize}
    
    \item \textbf{Sleep}: Improve sleep to 6-7 hours nightly with no more than 1 awakening within 4 weeks through sleep hygiene and possible medication adjustment.
    
    \item \textbf{Anxiety}: Reduce GAD-7 score from 12 to $<$8 within 6 weeks using CBT anxiety management techniques.
    
    \item \textbf{Suicide Risk}: Eliminate passive suicidal ideation, maintain safety contract, implement crisis plan within 2 weeks.
\end{enumerate}

\textbf{Functional Goals}:
\begin{enumerate}[leftmargin=*]
    \item \textbf{Self-Care}: Establish daily self-care routine (shower, meals, sleep schedule) with 80\% compliance within 3 weeks.
    
    \item \textbf{Social Engagement}: Re-engage in 1-2 social activities per week (phone calls with friends, sister visits) within 4 weeks.
    
    \item \textbf{Coping Skills}: Learn and practice 3 new coping skills for managing depressive symptoms within 4 weeks.
\end{enumerate}

\subsection*{3.2 Long-Term Goals (3-6 months)}

\textbf{Recovery-Oriented Goals}:
\begin{enumerate}[leftmargin=*]
    \item \textbf{Remission}: Achieve depression remission with PHQ-9 score $<$5 and sustained improved mood within 12-16 weeks.
    
    \item \textbf{Return to Work}: Develop job search plan, practice interview skills, secure employment or engage in meaningful volunteer work within 3-4 months.
    
    \item \textbf{Relationship Building}: Rebuild and strengthen social connections, increase social support network by adding 2-3 regular social contacts within 3 months.
    
    \item \textbf{Quality of Life}: Re-engage in previously enjoyed activities (hobbies, exercise, leisure) at least 3x per week within 3 months.
    
    \item \textbf{Resilience}: Develop sustainable wellness routine including regular sleep, exercise, healthy diet, and stress management practices within 4 months.
    
    \item \textbf{Relapse Prevention}: Identify early warning signs of depression, develop relapse prevention plan, maintain treatment gains within 6 months.
\end{enumerate}

\subsection*{3.3 Patient-Identified Goals}

\begin{itemize}[leftmargin=*]
    \item \textbf{Priority 1}: "I want to feel like myself again and have energy to do things"
    \item \textbf{Priority 2}: "I want to find a new job and feel confident in interviews"
    \item \textbf{Priority 3}: "I want to stop feeling guilty all the time"
    \item \textbf{Priority 4}: "I want to enjoy spending time with my friends and family again"
\end{itemize}

% ===== SECTION 4: TREATMENT INTERVENTIONS =====
\section*{4. Treatment Interventions}

\subsection*{4.1 Psychopharmacology}

\textbf{Medication Plan}:

\begin{longtable}{|p{3cm}|p{2cm}|p{2cm}|p{6.5cm}|}
\hline
\textbf{Medication} & \textbf{Dose} & \textbf{Frequency} & \textbf{Rationale \& Instructions} \\ \hline
\endfirsthead
\hline
\textbf{Medication} & \textbf{Dose} & \textbf{Frequency} & \textbf{Rationale \& Instructions} \\ \hline
\endhead

Escitalopram (Lexapro) & 10mg & Daily (morning) & \textbf{Rationale}: First-line SSRI for major depression. \textbf{Start}: 10mg daily. \textbf{Titration}: May increase to 20mg after 4 weeks if partial response. \textbf{Expected}: 2-4 weeks for initial response, 6-8 weeks for full effect. \textbf{Monitor}: Mood, anxiety, suicidal ideation, side effects. \\ \hline

Trazodone & 50mg & QHS PRN & \textbf{Rationale}: For insomnia, sedating antidepressant. \textbf{Start}: 50mg at bedtime as needed. \textbf{Titration}: May increase to 100mg if ineffective. \textbf{Instructions}: Take 30 min before bed. May cause morning grogginess - reduce dose if bothersome. \\ \hline

[Continue current medications] & & & \\ \hline
Levothyroxine & 100mcg & Daily & \textbf{Continue}: Hypothyroidism management. Monitor TSH every 6-12 months. \\ \hline
\end{longtable}

\textbf{Medication Safety and Monitoring}:
\begin{itemize}[leftmargin=*]
    \item \textbf{Common Side Effects}: Nausea (take with food), headache, insomnia or drowsiness, sexual dysfunction (discuss if bothersome)
    \item \textbf{Serious Side Effects} (rare): Serotonin syndrome (agitation, confusion, rapid heart rate, high fever - seek emergency care), increased suicidal thoughts (especially first 1-2 weeks - monitor closely)
    \item \textbf{Drug Interactions}: Avoid other serotonergic agents, NSAIDs (increased bleeding risk)
    \item \textbf{Adherence Plan}: Set daily reminder alarm, use pill box, refill prescriptions on time
    \item \textbf{Follow-up}: Psychiatry visit week 2 (phone), week 4 (in-person), week 8, then monthly
\end{itemize}

\textbf{Response Timeline}:
\begin{itemize}[leftmargin=*]
    \item Week 1-2: May notice side effects before benefits, monitor suicide risk closely
    \item Week 2-4: Early improvement in sleep, appetite, energy possible
    \item Week 4-6: Mood improvement, decreased anxiety expected
    \item Week 6-8: Full therapeutic effect, reassess dose if partial response
    \item Week 12+: Continued improvement, consider maintenance therapy
\end{itemize}

\subsection*{4.2 Psychotherapy}

\textbf{Therapy Modality}: Cognitive Behavioral Therapy (CBT) for Depression

\textbf{Frequency}: Weekly 50-minute sessions for 12-16 weeks, then biweekly as symptoms improve

\textbf{Treatment Framework}:

\textbf{Weeks 1-4: Assessment and Behavioral Activation}
\begin{itemize}[leftmargin=*]
    \item Establish therapeutic alliance and treatment goals
    \item Psychoeducation: Depression, treatment options, CBT model
    \item Activity monitoring and identifying mood-behavior connections
    \item Behavioral activation: Schedule pleasant and meaningful activities
    \item Develop daily structure and routine
    \item Suicide risk assessment and safety planning
\end{itemize}

\textbf{Weeks 5-8: Cognitive Restructuring}
\begin{itemize}[leftmargin=*]
    \item Identify automatic negative thoughts
    \item Challenge cognitive distortions (all-or-nothing thinking, overgeneralization, catastrophizing)
    \item Develop balanced, realistic thoughts
    \item Address guilt and worthlessness cognitions
    \item Problem-solving skills training
\end{itemize}

\textbf{Weeks 9-12: Skill Building and Application}
\begin{itemize}[leftmargin=*]
    \item Assertiveness and communication skills
    \item Interpersonal effectiveness
    \item Stress management and relaxation techniques
    \item Values clarification and goal-setting (career, relationships)
    \item Address employment/job search anxiety
\end{itemize}

\textbf{Weeks 13-16: Relapse Prevention and Maintenance}
\begin{itemize}[leftmargin=*]
    \item Identify early warning signs of depression
    \item Develop personalized relapse prevention plan
    \item Review and consolidate skills learned
    \item Plan for ongoing self-care and wellness
    \item Discuss transition to maintenance phase or termination
\end{itemize}

\textbf{Specific CBT Techniques}:
\begin{itemize}[leftmargin=*]
    \item Thought records (identify situations, thoughts, emotions, behaviors)
    \item Behavioral experiments (test negative predictions)
    \item Activity scheduling (increase rewarding activities)
    \item Graded task assignment (break large tasks into manageable steps)
    \item Cognitive continuum (evaluate black-and-white thinking)
    \item Core belief work (address underlying schemas)
\end{itemize}

\textbf{Homework Assignments}:
\begin{itemize}[leftmargin=*]
    \item Weekly mood and activity logs
    \item Thought records (3-column or 7-column)
    \item Behavioral activation: Complete 2-3 scheduled activities
    \item Reading: CBT self-help materials (e.g., "Feeling Good" by David Burns)
    \item Skills practice between sessions
\end{itemize}

\subsection*{4.3 Adjunctive Interventions}

\textbf{Case Management}:
\begin{itemize}[leftmargin=*]
    \item Assist with unemployment benefits and financial resources
    \item Connect with vocational rehabilitation services
    \item Coordinate care with primary care provider
    \item Insurance and medication assistance navigation
\end{itemize}

\textbf{Lifestyle Interventions}:
\begin{itemize}[leftmargin=*]
    \item \textbf{Exercise}: Goal of 30 minutes moderate exercise 5x/week (walking, yoga, biking)
    \item \textbf{Sleep Hygiene}: Consistent sleep schedule (11 PM - 7 AM), limit screen time 1 hour before bed, avoid caffeine after 2 PM, bedroom for sleep only
    \item \textbf{Nutrition}: Regular balanced meals, minimize processed foods, stay hydrated
    \item \textbf{Substance Use}: Continue cannabis abstinence, limit alcohol to 1-2 drinks/week max
    \item \textbf{Light Exposure}: Morning sunlight or light box 30 min daily (if seasonal pattern)
\end{itemize}

\textbf{Social Support Enhancement}:
\begin{itemize}[leftmargin=*]
    \item Increase contact with sister (supportive relationship)
    \item Consider depression support group (online or in-person)
    \item Re-engage with friend group gradually
    \item Volunteer opportunities for meaningful engagement
\end{itemize}

\textbf{Family/Collateral Sessions}:
\begin{itemize}[leftmargin=*]
    \item Offer to include sister in 1-2 sessions (with patient consent) for psychoeducation and support
    \item Educate family on depression, how to help, what to avoid (enabling, criticism)
\end{itemize}

% ===== SECTION 5: TREATMENT SCHEDULE =====
\section*{5. Treatment Schedule and Timeline}

\subsection*{Treatment Phases}

\begin{tabularx}{\textwidth}{|l|l|X|}
\hline
\textbf{Phase} & \textbf{Duration} & \textbf{Focus} \\ \hline
Acute Treatment & Weeks 1-8 & Symptom reduction, medication titration, behavioral activation, safety \\ \hline
Continuation & Weeks 9-16 & Cognitive restructuring, skill building, functional recovery \\ \hline
Maintenance & Months 4-12 & Relapse prevention, sustained wellness, reduce visit frequency \\ \hline
\end{tabularx}

\subsection*{Appointment Schedule}

\begin{tabularx}{\textwidth}{|l|l|X|}
\hline
\textbf{Provider} & \textbf{Frequency} & \textbf{Notes} \\ \hline
Psychiatry & Week 2 (phone), 4, 8, then monthly & Medication management, side effect monitoring \\ \hline
Psychotherapy (CBT) & Weekly weeks 1-12, biweekly weeks 13-16 & 50-minute sessions \\ \hline
PHQ-9/GAD-7 Assessment & Every 2 weeks & Track symptom severity \\ \hline
Case Management & As needed & Resources, benefits, vocational support \\ \hline
\end{tabularx}

\subsection*{Milestones and Reassessment}

\begin{itemize}[leftmargin=*]
    \item \textbf{Week 2}: Medication tolerance check, safety assessment, initial behavioral activation
    \item \textbf{Week 4}: PHQ-9 reassessment, medication dose adjustment if needed, CBT engagement
    \item \textbf{Week 8}: Comprehensive reassessment, PHQ-9 target $<$10, functional improvement expected
    \item \textbf{Week 12}: PHQ-9 target $<$5, relapse prevention planning initiated
    \item \textbf{Week 16}: Treatment goal review, transition to maintenance or taper frequency
\end{itemize}

% ===== SECTION 6: MONITORING AND OUTCOMES =====
\section*{6. Monitoring Parameters and Outcomes}

\subsection*{Symptom Tracking}

\begin{longtable}{|p{4cm}|p{2.5cm}|p{2.5cm}|p{4.5cm}|}
\hline
\textbf{Measure} & \textbf{Baseline} & \textbf{Target} & \textbf{Frequency} \\ \hline
\endfirsthead
\hline
\textbf{Measure} & \textbf{Baseline} & \textbf{Target} & \textbf{Frequency} \\ \hline
\endhead
PHQ-9 (Depression) & 18/27 & $<$5 (remission) & Every 2 weeks \\ \hline
GAD-7 (Anxiety) & 12/21 & $<$5 & Every 2 weeks \\ \hline
C-SSRS (Suicide Risk) & Level 3 (passive SI) & Level 0 (no SI) & Each session initially, then monthly \\ \hline
Sleep Quality & 4-5 hrs, fragmented & 6-7 hrs, consolidated & Weekly self-report \\ \hline
Social Activities & 0-1/week & 3-4/week & Weekly log \\ \hline
Exercise & 0 days/week & 5 days/week & Weekly log \\ \hline
Therapy Homework & -- & 80\% completion & Each session \\ \hline
Medication Adherence & -- & $>$90\% & Each psychiatry visit \\ \hline
\end{longtable}

\subsection*{Functional Outcome Tracking}

\begin{itemize}[leftmargin=*]
    \item \textbf{Self-Care}: Daily routine checklist (shower, meals, sleep, medications)
    \item \textbf{Social Functioning}: Number of social interactions per week
    \item \textbf{Occupational}: Job applications submitted, interviews attended, volunteer hours
    \item \textbf{Quality of Life}: Engagement in hobbies, pleasurable activities
    \item \textbf{Overall Functioning}: GAF or WHODAS score at baseline, 8 weeks, discharge
\end{itemize}

\subsection*{Safety Monitoring}

\begin{itemize}[leftmargin=*]
    \item Suicidal ideation assessment at every contact (especially weeks 1-4)
    \item Medication side effects and tolerability
    \item Substance use (alcohol, cannabis) - weekly check-ins
    \item Worsening symptoms or breakthrough depression
    \item Medication adherence
\end{itemize}

% ===== SECTION 7: CRISIS AND SAFETY PLANNING =====
\section*{7. Crisis Management and Safety Planning}

\subsection*{Safety Plan (Based on Stanley-Brown Model)}

\textbf{Step 1: Warning Signs}
\begin{itemize}[leftmargin=*]
    \item Thoughts: "I'm worthless," "Things will never get better," "I'm a burden"
    \item Feelings: Hopelessness, overwhelming sadness, numbness
    \item Behaviors: Isolating for days, not eating, excessive sleeping
    \item Situations: Financial stress, rejection, conflict with family
\end{itemize}

\textbf{Step 2: Internal Coping Strategies} (things I can do on my own)
\begin{itemize}[leftmargin=*]
    \item Go for a walk outside
    \item Listen to favorite music playlist
    \item Take a warm shower
    \item Deep breathing exercises (5-10 minutes)
    \item Read CBT thought records
    \item Write in journal
\end{itemize}

\textbf{Step 3: Social Contacts for Distraction}
\begin{itemize}[leftmargin=*]
    \item Sister: [phone number]
    \item Close friend: [phone number]
    \item Former coworker: [phone number]
\end{itemize}

\textbf{Step 4: People I Can Ask for Help}
\begin{itemize}[leftmargin=*]
    \item Sister: [phone number] - can talk about feelings, will listen without judgment
    \item Therapist: [phone number] - call for emergency appointment
    \item Psychiatrist: [phone number] - after-hours answering service
\end{itemize}

\textbf{Step 5: Professionals and Agencies to Contact}
\begin{itemize}[leftmargin=*]
    \item Therapist: [clinic phone]
    \item Psychiatrist on-call: [after-hours number]
    \item Crisis Line: 988 Suicide \& Crisis Lifeline (call or text 988)
    \item Crisis Text Line: Text HOME to 741741
    \item Local crisis center: [local crisis services phone]
\end{itemize}

\textbf{Step 6: Reduce Access to Lethal Means}
\begin{itemize}[leftmargin=*]
    \item No firearms in home
    \item Medications: Sister holds extra medication supply, patient has only 1-week supply at home
    \item Remove other potential means from immediate environment
\end{itemize}

\textbf{One Thing That Is Most Important to Me}:
\begin{itemize}[leftmargin=*]
    \item [e.g., "My relationship with my sister - I don't want to hurt her"]
\end{itemize}

\subsection*{Emergency Procedures}

\textbf{Patient to seek immediate care (Emergency Department or call 911) if}:
\begin{itemize}[leftmargin=*]
    \item Active suicidal ideation with plan and intent
    \item Unable to maintain safety despite using crisis plan
    \item Acute psychosis (hallucinations, delusions, disorganized behavior)
    \item Severe agitation or aggression toward others
    \item Substance intoxication/overdose
\end{itemize}

\textbf{Provider to intervene if}:
\begin{itemize}[leftmargin=*]
    \item Increased suicide risk (passive → active SI, plan development)
    \item Significant worsening of depression or emergence of psychotic symptoms
    \item Non-adherence with safety plan
    \item Relapse in substance use
    \item Actions: Increase visit frequency, consider higher level of care (IOP/PHP/inpatient), medication adjustment, collateral contact with family
\end{itemize}

% ===== SECTION 8: PATIENT EDUCATION =====
\section*{8. Patient Education and Psychoeducation}

\subsection*{Understanding Depression}

Education provided on:
\begin{itemize}[leftmargin=*]
    \item \textbf{What is Depression}: Biological illness, not weakness or character flaw
    \item \textbf{Neurobiology}: Serotonin, norepinephrine, brain circuits involved
    \item \textbf{Course}: Episodic illness, high recurrence rate, importance of treatment adherence
    \item \textbf{Treatment}: Evidence for medication + therapy combination
\end{itemize}

\subsection*{Medication Education}

\begin{itemize}[leftmargin=*]
    \item How SSRIs work (increase serotonin availability)
    \item Timeline for response (2-4 weeks initial, 6-8 weeks full effect)
    \item Common side effects and management
    \item Importance of daily adherence (not "as needed")
    \item Not addictive, but need to taper when discontinuing
    \item Maintenance treatment (continue 6-12 months after remission)
\end{itemize}

\subsection*{Therapy Skills and Homework}

\begin{itemize}[leftmargin=*]
    \item CBT model: Thoughts → Feelings → Behaviors (interconnected)
    \item Behavioral activation: Activity improves mood (not the reverse)
    \item Cognitive distortions: Common thinking errors in depression
    \item Thought challenging: Evidence for/against, alternative perspectives
    \item Skills practice between sessions is essential
\end{itemize}

\subsection*{Self-Management Strategies}

\begin{itemize}[leftmargin=*]
    \item Recognize early warning signs of depression
    \item When to call provider (worsening symptoms, suicidal thoughts)
    \item Lifestyle factors: sleep, exercise, nutrition, substance use
    \item Stress management and self-care
    \item Building and maintaining social connections
\end{itemize}

\subsection*{Resources Provided}

\begin{itemize}[leftmargin=*]
    \item Crisis hotline numbers (988, Crisis Text Line)
    \item CBT self-help books: "Feeling Good" by David Burns, "Mind Over Mood"
    \item Meditation apps: Headspace, Calm, Insight Timer
    \item Exercise resources: Local trails, gyms, online yoga
    \item NAMI (National Alliance on Mental Illness) support groups
    \item Depression and Bipolar Support Alliance (DBSA)
\end{itemize}

% ===== SECTION 9: FOLLOW-UP AND DISCHARGE =====
\section*{9. Follow-Up and Discharge Planning}

\subsection*{Continuation and Maintenance Treatment}

\textbf{After Acute Treatment (if goals achieved)}:
\begin{itemize}[leftmargin=*]
    \item Continue medication for 6-12 months minimum after remission
    \item Taper therapy to biweekly, then monthly "booster" sessions
    \item Regular symptom monitoring (monthly PHQ-9)
    \item Psychiatry visits every 2-3 months for medication management
\end{itemize}

\subsection*{Relapse Prevention}

\begin{itemize}[leftmargin=*]
    \item \textbf{Early Warning Signs}: [Patient-specific list from treatment]
    \item \textbf{Action Plan}: If warning signs emerge, resume weekly therapy, contact psychiatrist
    \item \textbf{Protective Factors}: Maintain exercise, sleep, social connections, continue medication
    \item \textbf{Ongoing Skills Practice}: Continue thought records, behavioral activation as needed
\end{itemize}

\subsection*{Discharge Criteria}

Ready for discharge when:
\begin{itemize}[leftmargin=*]
    \item PHQ-9 $<$5 sustained for 4+ weeks
    \item No suicidal ideation
    \item Functional recovery (working or engaged in meaningful activities, social connections restored)
    \item Mastery of CBT skills and relapse prevention plan
    \item Stable on medication regimen
    \item Patient and provider agree discharge is appropriate
\end{itemize}

\subsection*{Discharge Recommendations}

\begin{itemize}[leftmargin=*]
    \item Continue antidepressant for 6-12 months, then discuss tapering with psychiatrist
    \item Monthly "check-in" sessions available if needed
    \item Return to treatment if early warning signs emerge
    \item Continue healthy lifestyle practices
    \item Stay connected with support system
    \item Annual depression screening with primary care provider
\end{itemize}

% ===== SECTION 10: INFORMED CONSENT =====
\section*{10. Informed Consent and Collaboration}

\subsection*{Treatment Consent}

The following have been discussed with the patient:
\begin{itemize}[leftmargin=*]
    \item Diagnosis, symptoms, and prognosis
    \item Treatment options (medication, therapy, combination, no treatment)
    \item Risks and benefits of recommended treatment
    \item Expected timeline for improvement
    \item Potential side effects of medication
    \item Alternatives to proposed treatment
    \item Importance of adherence and therapy homework
    \item Right to refuse or discontinue treatment
    \item Limits of confidentiality (harm to self/others, abuse)
\end{itemize}

Patient demonstrates understanding and agrees to treatment plan. Questions answered satisfactorily. Patient has opportunity for shared decision-making and treatment preferences incorporated.

\subsection*{Collaborative Treatment Agreement}

\textbf{Provider Responsibilities}:
\begin{itemize}[leftmargin=*]
    \item Provide evidence-based treatment
    \item Monitor progress and adjust treatment as needed
    \item Maintain availability for emergencies (or provide backup coverage)
    \item Respect patient autonomy and preferences
\end{itemize}

\textbf{Patient Responsibilities}:
\begin{itemize}[leftmargin=*]
    \item Attend scheduled appointments
    \item Take medications as prescribed
    \item Complete therapy homework
    \item Communicate openly about symptoms and concerns
    \item Contact provider if symptoms worsen or suicidal thoughts emerge
    \item Follow safety plan
\end{itemize}

% ===== SECTION 11: SIGNATURES =====
\vspace{2em}

\section*{11. Provider Signature and Attestation}

I have reviewed this treatment plan with the patient. The patient demonstrates understanding of the diagnosis, treatment recommendations, risks and benefits, and alternatives. The patient has been involved in shared decision-making. Safety planning has been completed. The patient agrees to this treatment plan.

\vspace{1em}

\begin{tabular}{ll}
Provider Signature: & \rule{7cm}{0.5pt} \\[1em]
Provider Name/Credentials: & \rule{7cm}{0.5pt} \\[1em]
Date: & \rule{4cm}{0.5pt} \\[2em]
\end{tabular}

\subsection*{Patient Acknowledgment}

I have reviewed this treatment plan with my mental health provider. I understand my diagnosis, treatment goals, and the recommended interventions. My questions have been answered. I agree to participate in this treatment plan and will contact my provider if I have concerns or my symptoms worsen.

\vspace{1em}

\begin{tabular}{ll}
Patient Signature: & \rule{7cm}{0.5pt} \\[1em]
Date: & \rule{4cm}{0.5pt} \\
\end{tabular}

\vspace{2em}
\begin{center}
\rule{\textwidth}{1pt}\\
\textbf{End of Mental Health Treatment Plan}\\
This document contains confidential patient information protected by HIPAA and 42 CFR Part 2.
\end{center}

\end{document}

% ========== NOTES FOR USERS ==========
%
% CUSTOMIZATION:
% - Replace all bracketed placeholders with patient-specific information
% - Adjust CBT framework based on presenting problem (can use DBT, ACT, IPT instead)
% - Modify safety plan collaboratively with patient
% - Select appropriate medications based on diagnosis and patient factors
%
% IMPORTANT:
% - Complete thorough suicide risk assessment
% - Document safety planning
% - Ensure crisis resources are accurate and accessible
% - Maintain 42 CFR Part 2 confidentiality for substance use information
%
% COMPILATION:
% pdflatex mental_health_treatment_plan.tex

